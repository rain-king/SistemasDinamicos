\documentclass[11pt]{article}

    \usepackage[breakable]{tcolorbox}
    \usepackage{parskip} % Stop auto-indenting (to mimic markdown behaviour)
    
    \usepackage{iftex}
    \ifPDFTeX
    	\usepackage[T1]{fontenc}
    	\usepackage{mathpazo}
    \else
    	\usepackage{fontspec}
    \fi

    % Basic figure setup, for now with no caption control since it's done
    % automatically by Pandoc (which extracts ![](path) syntax from Markdown).
    \usepackage{graphicx}
    % Maintain compatibility with old templates. Remove in nbconvert 6.0
    \let\Oldincludegraphics\includegraphics
    % Ensure that by default, figures have no caption (until we provide a
    % proper Figure object with a Caption API and a way to capture that
    % in the conversion process - todo).
    \usepackage{caption}
    \DeclareCaptionFormat{nocaption}{}
    \captionsetup{format=nocaption,aboveskip=0pt,belowskip=0pt}

    \usepackage{float}
    \floatplacement{figure}{H} % forces figures to be placed at the correct location
    \usepackage{xcolor} % Allow colors to be defined
    \usepackage{enumerate} % Needed for markdown enumerations to work
    \usepackage{geometry} % Used to adjust the document margins
    \usepackage{amsmath} % Equations
    \usepackage{amssymb} % Equations
    \usepackage{textcomp} % defines textquotesingle
    % Hack from http://tex.stackexchange.com/a/47451/13684:
    \AtBeginDocument{%
        \def\PYZsq{\textquotesingle}% Upright quotes in Pygmentized code
    }
    \usepackage{upquote} % Upright quotes for verbatim code
    \usepackage{eurosym} % defines \euro
    \usepackage[mathletters]{ucs} % Extended unicode (utf-8) support
    \usepackage{fancyvrb} % verbatim replacement that allows latex
    \usepackage{grffile} % extends the file name processing of package graphics 
                         % to support a larger range
    \makeatletter % fix for old versions of grffile with XeLaTeX
    \@ifpackagelater{grffile}{2019/11/01}
    {
      % Do nothing on new versions
    }
    {
      \def\Gread@@xetex#1{%
        \IfFileExists{"\Gin@base".bb}%
        {\Gread@eps{\Gin@base.bb}}%
        {\Gread@@xetex@aux#1}%
      }
    }
    \makeatother
    \usepackage[Export]{adjustbox} % Used to constrain images to a maximum size
    \adjustboxset{max size={0.9\linewidth}{0.9\paperheight}}

    % The hyperref package gives us a pdf with properly built
    % internal navigation ('pdf bookmarks' for the table of contents,
    % internal cross-reference links, web links for URLs, etc.)
    \usepackage{hyperref}
    % The default LaTeX title has an obnoxious amount of whitespace. By default,
    % titling removes some of it. It also provides customization options.
    \usepackage{titling}
    \usepackage{longtable} % longtable support required by pandoc >1.10
    \usepackage{booktabs}  % table support for pandoc > 1.12.2
    \usepackage[inline]{enumitem} % IRkernel/repr support (it uses the enumerate* environment)
    \usepackage[normalem]{ulem} % ulem is needed to support strikethroughs (\sout)
                                % normalem makes italics be italics, not underlines
    \usepackage{mathrsfs}
    

    
    % Colors for the hyperref package
    \definecolor{urlcolor}{rgb}{0,.145,.698}
    \definecolor{linkcolor}{rgb}{.71,0.21,0.01}
    \definecolor{citecolor}{rgb}{.12,.54,.11}

    % ANSI colors
    \definecolor{ansi-black}{HTML}{3E424D}
    \definecolor{ansi-black-intense}{HTML}{282C36}
    \definecolor{ansi-red}{HTML}{E75C58}
    \definecolor{ansi-red-intense}{HTML}{B22B31}
    \definecolor{ansi-green}{HTML}{00A250}
    \definecolor{ansi-green-intense}{HTML}{007427}
    \definecolor{ansi-yellow}{HTML}{DDB62B}
    \definecolor{ansi-yellow-intense}{HTML}{B27D12}
    \definecolor{ansi-blue}{HTML}{208FFB}
    \definecolor{ansi-blue-intense}{HTML}{0065CA}
    \definecolor{ansi-magenta}{HTML}{D160C4}
    \definecolor{ansi-magenta-intense}{HTML}{A03196}
    \definecolor{ansi-cyan}{HTML}{60C6C8}
    \definecolor{ansi-cyan-intense}{HTML}{258F8F}
    \definecolor{ansi-white}{HTML}{C5C1B4}
    \definecolor{ansi-white-intense}{HTML}{A1A6B2}
    \definecolor{ansi-default-inverse-fg}{HTML}{FFFFFF}
    \definecolor{ansi-default-inverse-bg}{HTML}{000000}

    % common color for the border for error outputs.
    \definecolor{outerrorbackground}{HTML}{FFDFDF}

    % commands and environments needed by pandoc snippets
    % extracted from the output of `pandoc -s`
    \providecommand{\tightlist}{%
      \setlength{\itemsep}{0pt}\setlength{\parskip}{0pt}}
    \DefineVerbatimEnvironment{Highlighting}{Verbatim}{commandchars=\\\{\}}
    % Add ',fontsize=\small' for more characters per line
    \newenvironment{Shaded}{}{}
    \newcommand{\KeywordTok}[1]{\textcolor[rgb]{0.00,0.44,0.13}{\textbf{{#1}}}}
    \newcommand{\DataTypeTok}[1]{\textcolor[rgb]{0.56,0.13,0.00}{{#1}}}
    \newcommand{\DecValTok}[1]{\textcolor[rgb]{0.25,0.63,0.44}{{#1}}}
    \newcommand{\BaseNTok}[1]{\textcolor[rgb]{0.25,0.63,0.44}{{#1}}}
    \newcommand{\FloatTok}[1]{\textcolor[rgb]{0.25,0.63,0.44}{{#1}}}
    \newcommand{\CharTok}[1]{\textcolor[rgb]{0.25,0.44,0.63}{{#1}}}
    \newcommand{\StringTok}[1]{\textcolor[rgb]{0.25,0.44,0.63}{{#1}}}
    \newcommand{\CommentTok}[1]{\textcolor[rgb]{0.38,0.63,0.69}{\textit{{#1}}}}
    \newcommand{\OtherTok}[1]{\textcolor[rgb]{0.00,0.44,0.13}{{#1}}}
    \newcommand{\AlertTok}[1]{\textcolor[rgb]{1.00,0.00,0.00}{\textbf{{#1}}}}
    \newcommand{\FunctionTok}[1]{\textcolor[rgb]{0.02,0.16,0.49}{{#1}}}
    \newcommand{\RegionMarkerTok}[1]{{#1}}
    \newcommand{\ErrorTok}[1]{\textcolor[rgb]{1.00,0.00,0.00}{\textbf{{#1}}}}
    \newcommand{\NormalTok}[1]{{#1}}
    
    % Additional commands for more recent versions of Pandoc
    \newcommand{\ConstantTok}[1]{\textcolor[rgb]{0.53,0.00,0.00}{{#1}}}
    \newcommand{\SpecialCharTok}[1]{\textcolor[rgb]{0.25,0.44,0.63}{{#1}}}
    \newcommand{\VerbatimStringTok}[1]{\textcolor[rgb]{0.25,0.44,0.63}{{#1}}}
    \newcommand{\SpecialStringTok}[1]{\textcolor[rgb]{0.73,0.40,0.53}{{#1}}}
    \newcommand{\ImportTok}[1]{{#1}}
    \newcommand{\DocumentationTok}[1]{\textcolor[rgb]{0.73,0.13,0.13}{\textit{{#1}}}}
    \newcommand{\AnnotationTok}[1]{\textcolor[rgb]{0.38,0.63,0.69}{\textbf{\textit{{#1}}}}}
    \newcommand{\CommentVarTok}[1]{\textcolor[rgb]{0.38,0.63,0.69}{\textbf{\textit{{#1}}}}}
    \newcommand{\VariableTok}[1]{\textcolor[rgb]{0.10,0.09,0.49}{{#1}}}
    \newcommand{\ControlFlowTok}[1]{\textcolor[rgb]{0.00,0.44,0.13}{\textbf{{#1}}}}
    \newcommand{\OperatorTok}[1]{\textcolor[rgb]{0.40,0.40,0.40}{{#1}}}
    \newcommand{\BuiltInTok}[1]{{#1}}
    \newcommand{\ExtensionTok}[1]{{#1}}
    \newcommand{\PreprocessorTok}[1]{\textcolor[rgb]{0.74,0.48,0.00}{{#1}}}
    \newcommand{\AttributeTok}[1]{\textcolor[rgb]{0.49,0.56,0.16}{{#1}}}
    \newcommand{\InformationTok}[1]{\textcolor[rgb]{0.38,0.63,0.69}{\textbf{\textit{{#1}}}}}
    \newcommand{\WarningTok}[1]{\textcolor[rgb]{0.38,0.63,0.69}{\textbf{\textit{{#1}}}}}
    
    
    % Define a nice break command that doesn't care if a line doesn't already
    % exist.
    \def\br{\hspace*{\fill} \\* }
    % Math Jax compatibility definitions
    \def\gt{>}
    \def\lt{<}
    \let\Oldtex\TeX
    \let\Oldlatex\LaTeX
    \renewcommand{\TeX}{\textrm{\Oldtex}}
    \renewcommand{\LaTeX}{\textrm{\Oldlatex}}
    % Document parameters
    % Document title
    \title{Tarea2}
    
    
    
    
    
% Pygments definitions
\makeatletter
\def\PY@reset{\let\PY@it=\relax \let\PY@bf=\relax%
    \let\PY@ul=\relax \let\PY@tc=\relax%
    \let\PY@bc=\relax \let\PY@ff=\relax}
\def\PY@tok#1{\csname PY@tok@#1\endcsname}
\def\PY@toks#1+{\ifx\relax#1\empty\else%
    \PY@tok{#1}\expandafter\PY@toks\fi}
\def\PY@do#1{\PY@bc{\PY@tc{\PY@ul{%
    \PY@it{\PY@bf{\PY@ff{#1}}}}}}}
\def\PY#1#2{\PY@reset\PY@toks#1+\relax+\PY@do{#2}}

\@namedef{PY@tok@w}{\def\PY@tc##1{\textcolor[rgb]{0.73,0.73,0.73}{##1}}}
\@namedef{PY@tok@c}{\let\PY@it=\textit\def\PY@tc##1{\textcolor[rgb]{0.25,0.50,0.50}{##1}}}
\@namedef{PY@tok@cp}{\def\PY@tc##1{\textcolor[rgb]{0.74,0.48,0.00}{##1}}}
\@namedef{PY@tok@k}{\let\PY@bf=\textbf\def\PY@tc##1{\textcolor[rgb]{0.00,0.50,0.00}{##1}}}
\@namedef{PY@tok@kp}{\def\PY@tc##1{\textcolor[rgb]{0.00,0.50,0.00}{##1}}}
\@namedef{PY@tok@kt}{\def\PY@tc##1{\textcolor[rgb]{0.69,0.00,0.25}{##1}}}
\@namedef{PY@tok@o}{\def\PY@tc##1{\textcolor[rgb]{0.40,0.40,0.40}{##1}}}
\@namedef{PY@tok@ow}{\let\PY@bf=\textbf\def\PY@tc##1{\textcolor[rgb]{0.67,0.13,1.00}{##1}}}
\@namedef{PY@tok@nb}{\def\PY@tc##1{\textcolor[rgb]{0.00,0.50,0.00}{##1}}}
\@namedef{PY@tok@nf}{\def\PY@tc##1{\textcolor[rgb]{0.00,0.00,1.00}{##1}}}
\@namedef{PY@tok@nc}{\let\PY@bf=\textbf\def\PY@tc##1{\textcolor[rgb]{0.00,0.00,1.00}{##1}}}
\@namedef{PY@tok@nn}{\let\PY@bf=\textbf\def\PY@tc##1{\textcolor[rgb]{0.00,0.00,1.00}{##1}}}
\@namedef{PY@tok@ne}{\let\PY@bf=\textbf\def\PY@tc##1{\textcolor[rgb]{0.82,0.25,0.23}{##1}}}
\@namedef{PY@tok@nv}{\def\PY@tc##1{\textcolor[rgb]{0.10,0.09,0.49}{##1}}}
\@namedef{PY@tok@no}{\def\PY@tc##1{\textcolor[rgb]{0.53,0.00,0.00}{##1}}}
\@namedef{PY@tok@nl}{\def\PY@tc##1{\textcolor[rgb]{0.63,0.63,0.00}{##1}}}
\@namedef{PY@tok@ni}{\let\PY@bf=\textbf\def\PY@tc##1{\textcolor[rgb]{0.60,0.60,0.60}{##1}}}
\@namedef{PY@tok@na}{\def\PY@tc##1{\textcolor[rgb]{0.49,0.56,0.16}{##1}}}
\@namedef{PY@tok@nt}{\let\PY@bf=\textbf\def\PY@tc##1{\textcolor[rgb]{0.00,0.50,0.00}{##1}}}
\@namedef{PY@tok@nd}{\def\PY@tc##1{\textcolor[rgb]{0.67,0.13,1.00}{##1}}}
\@namedef{PY@tok@s}{\def\PY@tc##1{\textcolor[rgb]{0.73,0.13,0.13}{##1}}}
\@namedef{PY@tok@sd}{\let\PY@it=\textit\def\PY@tc##1{\textcolor[rgb]{0.73,0.13,0.13}{##1}}}
\@namedef{PY@tok@si}{\let\PY@bf=\textbf\def\PY@tc##1{\textcolor[rgb]{0.73,0.40,0.53}{##1}}}
\@namedef{PY@tok@se}{\let\PY@bf=\textbf\def\PY@tc##1{\textcolor[rgb]{0.73,0.40,0.13}{##1}}}
\@namedef{PY@tok@sr}{\def\PY@tc##1{\textcolor[rgb]{0.73,0.40,0.53}{##1}}}
\@namedef{PY@tok@ss}{\def\PY@tc##1{\textcolor[rgb]{0.10,0.09,0.49}{##1}}}
\@namedef{PY@tok@sx}{\def\PY@tc##1{\textcolor[rgb]{0.00,0.50,0.00}{##1}}}
\@namedef{PY@tok@m}{\def\PY@tc##1{\textcolor[rgb]{0.40,0.40,0.40}{##1}}}
\@namedef{PY@tok@gh}{\let\PY@bf=\textbf\def\PY@tc##1{\textcolor[rgb]{0.00,0.00,0.50}{##1}}}
\@namedef{PY@tok@gu}{\let\PY@bf=\textbf\def\PY@tc##1{\textcolor[rgb]{0.50,0.00,0.50}{##1}}}
\@namedef{PY@tok@gd}{\def\PY@tc##1{\textcolor[rgb]{0.63,0.00,0.00}{##1}}}
\@namedef{PY@tok@gi}{\def\PY@tc##1{\textcolor[rgb]{0.00,0.63,0.00}{##1}}}
\@namedef{PY@tok@gr}{\def\PY@tc##1{\textcolor[rgb]{1.00,0.00,0.00}{##1}}}
\@namedef{PY@tok@ge}{\let\PY@it=\textit}
\@namedef{PY@tok@gs}{\let\PY@bf=\textbf}
\@namedef{PY@tok@gp}{\let\PY@bf=\textbf\def\PY@tc##1{\textcolor[rgb]{0.00,0.00,0.50}{##1}}}
\@namedef{PY@tok@go}{\def\PY@tc##1{\textcolor[rgb]{0.53,0.53,0.53}{##1}}}
\@namedef{PY@tok@gt}{\def\PY@tc##1{\textcolor[rgb]{0.00,0.27,0.87}{##1}}}
\@namedef{PY@tok@err}{\def\PY@bc##1{{\setlength{\fboxsep}{\string -\fboxrule}\fcolorbox[rgb]{1.00,0.00,0.00}{1,1,1}{\strut ##1}}}}
\@namedef{PY@tok@kc}{\let\PY@bf=\textbf\def\PY@tc##1{\textcolor[rgb]{0.00,0.50,0.00}{##1}}}
\@namedef{PY@tok@kd}{\let\PY@bf=\textbf\def\PY@tc##1{\textcolor[rgb]{0.00,0.50,0.00}{##1}}}
\@namedef{PY@tok@kn}{\let\PY@bf=\textbf\def\PY@tc##1{\textcolor[rgb]{0.00,0.50,0.00}{##1}}}
\@namedef{PY@tok@kr}{\let\PY@bf=\textbf\def\PY@tc##1{\textcolor[rgb]{0.00,0.50,0.00}{##1}}}
\@namedef{PY@tok@bp}{\def\PY@tc##1{\textcolor[rgb]{0.00,0.50,0.00}{##1}}}
\@namedef{PY@tok@fm}{\def\PY@tc##1{\textcolor[rgb]{0.00,0.00,1.00}{##1}}}
\@namedef{PY@tok@vc}{\def\PY@tc##1{\textcolor[rgb]{0.10,0.09,0.49}{##1}}}
\@namedef{PY@tok@vg}{\def\PY@tc##1{\textcolor[rgb]{0.10,0.09,0.49}{##1}}}
\@namedef{PY@tok@vi}{\def\PY@tc##1{\textcolor[rgb]{0.10,0.09,0.49}{##1}}}
\@namedef{PY@tok@vm}{\def\PY@tc##1{\textcolor[rgb]{0.10,0.09,0.49}{##1}}}
\@namedef{PY@tok@sa}{\def\PY@tc##1{\textcolor[rgb]{0.73,0.13,0.13}{##1}}}
\@namedef{PY@tok@sb}{\def\PY@tc##1{\textcolor[rgb]{0.73,0.13,0.13}{##1}}}
\@namedef{PY@tok@sc}{\def\PY@tc##1{\textcolor[rgb]{0.73,0.13,0.13}{##1}}}
\@namedef{PY@tok@dl}{\def\PY@tc##1{\textcolor[rgb]{0.73,0.13,0.13}{##1}}}
\@namedef{PY@tok@s2}{\def\PY@tc##1{\textcolor[rgb]{0.73,0.13,0.13}{##1}}}
\@namedef{PY@tok@sh}{\def\PY@tc##1{\textcolor[rgb]{0.73,0.13,0.13}{##1}}}
\@namedef{PY@tok@s1}{\def\PY@tc##1{\textcolor[rgb]{0.73,0.13,0.13}{##1}}}
\@namedef{PY@tok@mb}{\def\PY@tc##1{\textcolor[rgb]{0.40,0.40,0.40}{##1}}}
\@namedef{PY@tok@mf}{\def\PY@tc##1{\textcolor[rgb]{0.40,0.40,0.40}{##1}}}
\@namedef{PY@tok@mh}{\def\PY@tc##1{\textcolor[rgb]{0.40,0.40,0.40}{##1}}}
\@namedef{PY@tok@mi}{\def\PY@tc##1{\textcolor[rgb]{0.40,0.40,0.40}{##1}}}
\@namedef{PY@tok@il}{\def\PY@tc##1{\textcolor[rgb]{0.40,0.40,0.40}{##1}}}
\@namedef{PY@tok@mo}{\def\PY@tc##1{\textcolor[rgb]{0.40,0.40,0.40}{##1}}}
\@namedef{PY@tok@ch}{\let\PY@it=\textit\def\PY@tc##1{\textcolor[rgb]{0.25,0.50,0.50}{##1}}}
\@namedef{PY@tok@cm}{\let\PY@it=\textit\def\PY@tc##1{\textcolor[rgb]{0.25,0.50,0.50}{##1}}}
\@namedef{PY@tok@cpf}{\let\PY@it=\textit\def\PY@tc##1{\textcolor[rgb]{0.25,0.50,0.50}{##1}}}
\@namedef{PY@tok@c1}{\let\PY@it=\textit\def\PY@tc##1{\textcolor[rgb]{0.25,0.50,0.50}{##1}}}
\@namedef{PY@tok@cs}{\let\PY@it=\textit\def\PY@tc##1{\textcolor[rgb]{0.25,0.50,0.50}{##1}}}

\def\PYZbs{\char`\\}
\def\PYZus{\char`\_}
\def\PYZob{\char`\{}
\def\PYZcb{\char`\}}
\def\PYZca{\char`\^}
\def\PYZam{\char`\&}
\def\PYZlt{\char`\<}
\def\PYZgt{\char`\>}
\def\PYZsh{\char`\#}
\def\PYZpc{\char`\%}
\def\PYZdl{\char`\$}
\def\PYZhy{\char`\-}
\def\PYZsq{\char`\'}
\def\PYZdq{\char`\"}
\def\PYZti{\char`\~}
% for compatibility with earlier versions
\def\PYZat{@}
\def\PYZlb{[}
\def\PYZrb{]}
\makeatother


    % For linebreaks inside Verbatim environment from package fancyvrb. 
    \makeatletter
        \newbox\Wrappedcontinuationbox 
        \newbox\Wrappedvisiblespacebox 
        \newcommand*\Wrappedvisiblespace {\textcolor{red}{\textvisiblespace}} 
        \newcommand*\Wrappedcontinuationsymbol {\textcolor{red}{\llap{\tiny$\m@th\hookrightarrow$}}} 
        \newcommand*\Wrappedcontinuationindent {3ex } 
        \newcommand*\Wrappedafterbreak {\kern\Wrappedcontinuationindent\copy\Wrappedcontinuationbox} 
        % Take advantage of the already applied Pygments mark-up to insert 
        % potential linebreaks for TeX processing. 
        %        {, <, #, %, $, ' and ": go to next line. 
        %        _, }, ^, &, >, - and ~: stay at end of broken line. 
        % Use of \textquotesingle for straight quote. 
        \newcommand*\Wrappedbreaksatspecials {% 
            \def\PYGZus{\discretionary{\char`\_}{\Wrappedafterbreak}{\char`\_}}% 
            \def\PYGZob{\discretionary{}{\Wrappedafterbreak\char`\{}{\char`\{}}% 
            \def\PYGZcb{\discretionary{\char`\}}{\Wrappedafterbreak}{\char`\}}}% 
            \def\PYGZca{\discretionary{\char`\^}{\Wrappedafterbreak}{\char`\^}}% 
            \def\PYGZam{\discretionary{\char`\&}{\Wrappedafterbreak}{\char`\&}}% 
            \def\PYGZlt{\discretionary{}{\Wrappedafterbreak\char`\<}{\char`\<}}% 
            \def\PYGZgt{\discretionary{\char`\>}{\Wrappedafterbreak}{\char`\>}}% 
            \def\PYGZsh{\discretionary{}{\Wrappedafterbreak\char`\#}{\char`\#}}% 
            \def\PYGZpc{\discretionary{}{\Wrappedafterbreak\char`\%}{\char`\%}}% 
            \def\PYGZdl{\discretionary{}{\Wrappedafterbreak\char`\$}{\char`\$}}% 
            \def\PYGZhy{\discretionary{\char`\-}{\Wrappedafterbreak}{\char`\-}}% 
            \def\PYGZsq{\discretionary{}{\Wrappedafterbreak\textquotesingle}{\textquotesingle}}% 
            \def\PYGZdq{\discretionary{}{\Wrappedafterbreak\char`\"}{\char`\"}}% 
            \def\PYGZti{\discretionary{\char`\~}{\Wrappedafterbreak}{\char`\~}}% 
        } 
        % Some characters . , ; ? ! / are not pygmentized. 
        % This macro makes them "active" and they will insert potential linebreaks 
        \newcommand*\Wrappedbreaksatpunct {% 
            \lccode`\~`\.\lowercase{\def~}{\discretionary{\hbox{\char`\.}}{\Wrappedafterbreak}{\hbox{\char`\.}}}% 
            \lccode`\~`\,\lowercase{\def~}{\discretionary{\hbox{\char`\,}}{\Wrappedafterbreak}{\hbox{\char`\,}}}% 
            \lccode`\~`\;\lowercase{\def~}{\discretionary{\hbox{\char`\;}}{\Wrappedafterbreak}{\hbox{\char`\;}}}% 
            \lccode`\~`\:\lowercase{\def~}{\discretionary{\hbox{\char`\:}}{\Wrappedafterbreak}{\hbox{\char`\:}}}% 
            \lccode`\~`\?\lowercase{\def~}{\discretionary{\hbox{\char`\?}}{\Wrappedafterbreak}{\hbox{\char`\?}}}% 
            \lccode`\~`\!\lowercase{\def~}{\discretionary{\hbox{\char`\!}}{\Wrappedafterbreak}{\hbox{\char`\!}}}% 
            \lccode`\~`\/\lowercase{\def~}{\discretionary{\hbox{\char`\/}}{\Wrappedafterbreak}{\hbox{\char`\/}}}% 
            \catcode`\.\active
            \catcode`\,\active 
            \catcode`\;\active
            \catcode`\:\active
            \catcode`\?\active
            \catcode`\!\active
            \catcode`\/\active 
            \lccode`\~`\~ 	
        }
    \makeatother

    \let\OriginalVerbatim=\Verbatim
    \makeatletter
    \renewcommand{\Verbatim}[1][1]{%
        %\parskip\z@skip
        \sbox\Wrappedcontinuationbox {\Wrappedcontinuationsymbol}%
        \sbox\Wrappedvisiblespacebox {\FV@SetupFont\Wrappedvisiblespace}%
        \def\FancyVerbFormatLine ##1{\hsize\linewidth
            \vtop{\raggedright\hyphenpenalty\z@\exhyphenpenalty\z@
                \doublehyphendemerits\z@\finalhyphendemerits\z@
                \strut ##1\strut}%
        }%
        % If the linebreak is at a space, the latter will be displayed as visible
        % space at end of first line, and a continuation symbol starts next line.
        % Stretch/shrink are however usually zero for typewriter font.
        \def\FV@Space {%
            \nobreak\hskip\z@ plus\fontdimen3\font minus\fontdimen4\font
            \discretionary{\copy\Wrappedvisiblespacebox}{\Wrappedafterbreak}
            {\kern\fontdimen2\font}%
        }%
        
        % Allow breaks at special characters using \PYG... macros.
        \Wrappedbreaksatspecials
        % Breaks at punctuation characters . , ; ? ! and / need catcode=\active 	
        \OriginalVerbatim[#1,codes*=\Wrappedbreaksatpunct]%
    }
    \makeatother

    % Exact colors from NB
    \definecolor{incolor}{HTML}{303F9F}
    \definecolor{outcolor}{HTML}{D84315}
    \definecolor{cellborder}{HTML}{CFCFCF}
    \definecolor{cellbackground}{HTML}{F7F7F7}
    
    % prompt
    \makeatletter
    \newcommand{\boxspacing}{\kern\kvtcb@left@rule\kern\kvtcb@boxsep}
    \makeatother
    \newcommand{\prompt}[4]{
        {\ttfamily\llap{{\color{#2}[#3]:\hspace{3pt}#4}}\vspace{-\baselineskip}}
    }
    

    
    % Prevent overflowing lines due to hard-to-break entities
    \sloppy 
    % Setup hyperref package
    \hypersetup{
      breaklinks=true,  % so long urls are correctly broken across lines
      colorlinks=true,
      urlcolor=urlcolor,
      linkcolor=linkcolor,
      citecolor=citecolor,
      }
    % Slightly bigger margins than the latex defaults
    
    \geometry{verbose,tmargin=1in,bmargin=1in,lmargin=1in,rmargin=1in}
    
    

\begin{document}
    
%    \maketitle
    
    

    
    \hypertarget{ejercicio-1-experimento-con-luz}{%
\section*{Ejercicio 1: Experimento con
Luz}\label{ejercicio-1-experimento-con-luz}}

\hypertarget{descripciuxf3n-del-experimento-y-notaciuxf3n}{%
\subsection*{Descripción del Experimento y
Notación}\label{descripciuxf3n-del-experimento-y-notaciuxf3n}}

Se mide la intensidad de la luz \(I_d\) al inicio de la capa de
profundidad \(d\) de cierta agua. La longitud de cada capa es la misma.
La distancia del inicio de la capa \(d\) a la superficie, es \(d\) veces
la longitud de las capas. Y como es usual \(\Delta I_d := I_{d+1}-I_d\);
y \(\hat I_d\) representa una aproximación a \(I_d\).

Los resultados del experimento para esta agua son los siguientes.

    \begin{tcolorbox}[breakable, size=fbox, boxrule=1pt, pad at break*=1mm,colback=cellbackground, colframe=cellborder]
\prompt{In}{incolor}{38}{\boxspacing}
\begin{Verbatim}[commandchars=\\\{\}]
\PY{k+kn}{import} \PY{n+nn}{matplotlib}\PY{n+nn}{.}\PY{n+nn}{pyplot} \PY{k}{as} \PY{n+nn}{plt}
\PY{k+kn}{import} \PY{n+nn}{matplotlib} \PY{k}{as} \PY{n+nn}{mpl}
\PY{n}{mpl}\PY{o}{.}\PY{n}{rcParams}\PY{p}{[}\PY{l+s+s1}{\PYZsq{}}\PY{l+s+s1}{figure.figsize}\PY{l+s+s1}{\PYZsq{}}\PY{p}{]} \PY{o}{=} \PY{p}{[}\PY{l+m+mi}{10}\PY{p}{,}\PY{l+m+mi}{8}\PY{p}{]}
\PY{n}{mpl}\PY{o}{.}\PY{n}{rcParams}\PY{p}{[}\PY{l+s+s1}{\PYZsq{}}\PY{l+s+s1}{font.size}\PY{l+s+s1}{\PYZsq{}}\PY{p}{]} \PY{o}{=} \PY{l+m+mi}{14}
\PY{k+kn}{import} \PY{n+nn}{pandas} \PY{k}{as} \PY{n+nn}{pd}
\PY{n}{pd}\PY{o}{.}\PY{n}{set\PYZus{}option}\PY{p}{(}\PY{l+s+s1}{\PYZsq{}}\PY{l+s+s1}{display.precision}\PY{l+s+s1}{\PYZsq{}}\PY{p}{,} \PY{l+m+mi}{3}\PY{p}{)}
\PY{n}{pd}\PY{o}{.}\PY{n}{set\PYZus{}option}\PY{p}{(}\PY{l+s+s2}{\PYZdq{}}\PY{l+s+s2}{display.latex.repr}\PY{l+s+s2}{\PYZdq{}}\PY{p}{,} \PY{k+kc}{False}\PY{p}{)}

\PY{c+c1}{\PYZsh{}datos = pd.read\PYZus{}clipboard(header=[0])}
\PY{n}{agua1} \PY{o}{=} \PY{p}{\PYZob{}}\PY{l+s+s1}{\PYZsq{}}\PY{l+s+s1}{\PYZdl{}d\PYZdl{}}\PY{l+s+s1}{\PYZsq{}}\PY{p}{:} \PY{p}{\PYZob{}}\PY{l+m+mi}{0}\PY{p}{:} \PY{l+m+mi}{0}\PY{p}{,} \PY{l+m+mi}{1}\PY{p}{:} \PY{l+m+mi}{1}\PY{p}{,} \PY{l+m+mi}{2}\PY{p}{:} \PY{l+m+mi}{2}\PY{p}{,} \PY{l+m+mi}{3}\PY{p}{:} \PY{l+m+mi}{3}\PY{p}{,} \PY{l+m+mi}{4}\PY{p}{:} \PY{l+m+mi}{4}\PY{p}{,} \PY{l+m+mi}{5}\PY{p}{:} \PY{l+m+mi}{5}\PY{p}{,} \PY{l+m+mi}{6}\PY{p}{:} \PY{l+m+mi}{6}\PY{p}{,} \PY{l+m+mi}{7}\PY{p}{:} \PY{l+m+mi}{7}\PY{p}{,} \PY{l+m+mi}{8}\PY{p}{:} \PY{l+m+mi}{8}\PY{p}{\PYZcb{}}\PY{p}{,}
         \PY{l+s+s1}{\PYZsq{}}\PY{l+s+s1}{\PYZdl{}I\PYZus{}d\PYZdl{}}\PY{l+s+s1}{\PYZsq{}}\PY{p}{:} \PY{p}{\PYZob{}}\PY{l+m+mi}{0}\PY{p}{:} \PY{l+m+mf}{0.842}\PY{p}{,} \PY{l+m+mi}{1}\PY{p}{:} \PY{l+m+mf}{0.639}\PY{p}{,} \PY{l+m+mi}{2}\PY{p}{:} \PY{l+m+mf}{0.459}\PY{p}{,} \PY{l+m+mi}{3}\PY{p}{:} \PY{l+m+mf}{0.348}\PY{p}{,} \PY{l+m+mi}{4}\PY{p}{:} \PY{l+m+mf}{0.263}\PY{p}{,} \PY{l+m+mi}{5}\PY{p}{:} \PY{l+m+mf}{0.202}\PY{p}{,} \PY{l+m+mi}{6}\PY{p}{:} \PY{l+m+mf}{0.154}\PY{p}{,} \PY{l+m+mi}{7}\PY{p}{:} \PY{l+m+mf}{0.114}\PY{p}{,} \PY{l+m+mi}{8}\PY{p}{:} \PY{l+m+mf}{0.085}\PY{p}{\PYZcb{}}\PY{p}{,}
         \PY{l+s+s1}{\PYZsq{}}\PY{l+s+s1}{\PYZdl{}}\PY{l+s+se}{\PYZbs{}\PYZbs{}}\PY{l+s+s1}{Delta I\PYZus{}d\PYZdl{}}\PY{l+s+s1}{\PYZsq{}}\PY{p}{:} \PY{p}{\PYZob{}}\PY{l+m+mi}{0}\PY{p}{:} \PY{o}{\PYZhy{}}\PY{l+m+mf}{0.203}\PY{p}{,} \PY{l+m+mi}{1}\PY{p}{:} \PY{o}{\PYZhy{}}\PY{l+m+mf}{0.18}\PY{p}{,} \PY{l+m+mi}{2}\PY{p}{:} \PY{o}{\PYZhy{}}\PY{l+m+mf}{0.111}\PY{p}{,} \PY{l+m+mi}{3}\PY{p}{:} \PY{o}{\PYZhy{}}\PY{l+m+mf}{0.085}\PY{p}{,} \PY{l+m+mi}{4}\PY{p}{:} \PY{o}{\PYZhy{}}\PY{l+m+mf}{0.061}\PY{p}{,} \PY{l+m+mi}{5}\PY{p}{:} \PY{o}{\PYZhy{}}\PY{l+m+mf}{0.048}\PY{p}{,} \PY{l+m+mi}{6}\PY{p}{:} \PY{o}{\PYZhy{}}\PY{l+m+mf}{0.04}\PY{p}{,} \PY{l+m+mi}{7}\PY{p}{:} \PY{o}{\PYZhy{}}\PY{l+m+mf}{0.029}\PY{p}{,} \PY{l+m+mi}{8}\PY{p}{:} \PY{k+kc}{None}\PY{p}{\PYZcb{}}\PY{p}{\PYZcb{}}
\PY{n}{datos} \PY{o}{=} \PY{n}{pd}\PY{o}{.}\PY{n}{DataFrame}\PY{p}{(}\PY{n}{agua1}\PY{p}{)}

\PY{n}{display}\PY{p}{(}\PY{n}{datos}\PY{o}{.}\PY{n}{plot}\PY{p}{(}\PY{n}{x}\PY{o}{=}\PY{l+s+s2}{\PYZdq{}}\PY{l+s+s2}{\PYZdl{}d\PYZdl{}}\PY{l+s+s2}{\PYZdq{}}\PY{p}{,} \PY{n}{y}\PY{o}{=}\PY{l+s+s2}{\PYZdq{}}\PY{l+s+s2}{\PYZdl{}I\PYZus{}d\PYZdl{}}\PY{l+s+s2}{\PYZdq{}}\PY{p}{,} \PY{n}{marker}\PY{o}{=}\PY{l+s+s2}{\PYZdq{}}\PY{l+s+s2}{o}\PY{l+s+s2}{\PYZdq{}}\PY{p}{,} \PY{n}{color}\PY{o}{=}\PY{l+s+s2}{\PYZdq{}}\PY{l+s+s2}{g}\PY{l+s+s2}{\PYZdq{}}\PY{p}{)}\PY{p}{)}
\PY{n}{display}\PY{p}{(}\PY{n}{datos}\PY{o}{.}\PY{n}{plot}\PY{p}{(}\PY{n}{x}\PY{o}{=}\PY{l+s+s2}{\PYZdq{}}\PY{l+s+s2}{\PYZdl{}I\PYZus{}d\PYZdl{}}\PY{l+s+s2}{\PYZdq{}}\PY{p}{,} \PY{n}{y}\PY{o}{=}\PY{l+s+s2}{\PYZdq{}}\PY{l+s+s2}{\PYZdl{}}\PY{l+s+se}{\PYZbs{}\PYZbs{}}\PY{l+s+s2}{Delta I\PYZus{}d\PYZdl{}}\PY{l+s+s2}{\PYZdq{}}\PY{p}{,}  \PY{n}{marker}\PY{o}{=}\PY{l+s+s2}{\PYZdq{}}\PY{l+s+s2}{o}\PY{l+s+s2}{\PYZdq{}}\PY{p}{,} \PY{n}{color}\PY{o}{=}\PY{l+s+s2}{\PYZdq{}}\PY{l+s+s2}{g}\PY{l+s+s2}{\PYZdq{}}\PY{p}{)}\PY{p}{)}

\PY{c+c1}{\PYZsh{}print(datos.to\PYZus{}latex(index=False, escape=False))}
\end{Verbatim}
\end{tcolorbox}

    
    \begin{Verbatim}[commandchars=\\\{\}]
<AxesSubplot:xlabel='\$d\$'>
    \end{Verbatim}

    
    
    \begin{Verbatim}[commandchars=\\\{\}]
<AxesSubplot:xlabel='\$I\_d\$'>
    \end{Verbatim}

    
    \begin{center}
    \adjustimage{max size={0.9\linewidth}{0.9\paperheight}}{output_1_2.png}
    \end{center}
    { \hspace*{\fill} \\}
    
    \begin{center}
    \adjustimage{max size={0.9\linewidth}{0.9\paperheight}}{output_1_3.png}
    \end{center}
    { \hspace*{\fill} \\}
    
    \begin{tabular}{rrr}
\toprule
 $d$ &  $I_d$ &  $\Delta I_d$ \\
\midrule
   0 &  0.842 &        -0.203 \\
   1 &  0.639 &        -0.180 \\
   2 &  0.459 &        -0.111 \\
   3 &  0.348 &        -0.085 \\
   4 &  0.263 &        -0.061 \\
   5 &  0.202 &        -0.048 \\
   6 &  0.154 &        -0.040 \\
   7 &  0.114 &        -0.029 \\
   8 &  0.085 &           NaN \\
\bottomrule
\end{tabular}

    \hypertarget{ecuaciuxf3n-dinuxe1mica-y-soluciuxf3n}{%
\subsection*{Ecuación Dinámica y
Solución}\label{ecuaciuxf3n-dinuxe1mica-y-soluciuxf3n}}

La segunda gráfica de diferencias contra intensidades sugiere un modelo
lineal. Debe de considerarse con cuidado cuál es la ecuación dinámica de
estos datos pues los datos del eje \(I_d\) han sido re-ordenados. (El
punto con \(d=0\) en realidad será aquel con mayor \(I_d\), pues se
observa de los datos que la intensidad disminuye conforme la capa \(d\)
es más profunda.)

La hipótesis a usar para este modelo es que una fracción
\(f:1\geq f \geq 0\) de la intensidad de luz se pierde al pasar a la
siguiente capa. De manera que la diferencia de intensidad de la luz de
una capa a la siguiente está dada por la ecuación dinámica

\[\Delta \hat I_d = -f I_d,\] que tiene solución

\[\hat I_d = (1-f)^{d}I_0,\]

donde \(d\) está extendido del número de capas en enteros no negativos a
cualquier valor no negativo. Se interpretan los valores no enteros de
\(d\) como la distancia a la superficie en múltiplos del grosor de las
capas.

    \hypertarget{ajuste-lineal}{%
\subsection*{Ajuste lineal}\label{ajuste-lineal}}

Se usa una librería de ajuste polinomial y se usa el coeficiente de
\(x\) del resultado para grado 1 como pendiente para modelar la
diferencia de intensidad respecto de la intensidad.

    \begin{tcolorbox}[breakable, size=fbox, boxrule=1pt, pad at break*=1mm,colback=cellbackground, colframe=cellborder]
\prompt{In}{incolor}{2}{\boxspacing}
\begin{Verbatim}[commandchars=\\\{\}]
\PY{k+kn}{from} \PY{n+nn}{numpy} \PY{k+kn}{import} \PY{n}{polyfit} \PY{k}{as} \PY{n}{pf}
\PY{p}{(}\PY{n}{m}\PY{p}{,}\PY{n}{b}\PY{p}{)} \PY{o}{=} \PY{n+nb}{list}\PY{p}{(}\PY{n}{pf}\PY{p}{(}\PY{n}{x}\PY{o}{=}\PY{n}{datos}\PY{o}{.}\PY{n}{iloc}\PY{p}{[}\PY{p}{:}\PY{o}{\PYZhy{}}\PY{l+m+mi}{1}\PY{p}{,}\PY{l+m+mi}{1}\PY{p}{]}\PY{o}{.}\PY{n}{to\PYZus{}numpy}\PY{p}{(}\PY{p}{)}\PY{p}{,} \PY{n}{y}\PY{o}{=}\PY{n}{datos}\PY{o}{.}\PY{n}{iloc}\PY{p}{[}\PY{p}{:}\PY{o}{\PYZhy{}}\PY{l+m+mi}{1}\PY{p}{,}\PY{l+m+mi}{2}\PY{p}{]}\PY{o}{.}\PY{n}{to\PYZus{}numpy}\PY{p}{(}\PY{p}{)}\PY{p}{,} \PY{n}{deg}\PY{o}{=}\PY{l+m+mi}{1}\PY{p}{)}\PY{p}{)}
\PY{p}{(}\PY{n}{m}\PY{p}{,}\PY{n}{b}\PY{p}{)} \PY{c+c1}{\PYZsh{} m es \PYZhy{}f}
\end{Verbatim}
\end{tcolorbox}

            \begin{tcolorbox}[breakable, size=fbox, boxrule=.5pt, pad at break*=1mm, opacityfill=0]
\prompt{Out}{outcolor}{2}{\boxspacing}
\begin{Verbatim}[commandchars=\\\{\}]
(-0.2545728251931048, 0.0015080631135462607)
\end{Verbatim}
\end{tcolorbox}
        
    Como hemos dicho al inicio de esta sección en realidad los puntos
\((I_d, \Delta I_d)\) cerca de intensidad 0 son datos de las capas más
profundas. Si \(I_d\) llegase a \(0\) en algún \(d\), las siguientes
\(I_{d+k}\) no deberían de poder disminuir a valores negativos. Por lo
que el valor de \(\Delta I_d\) cuando \(I_d=0\) debe de ser \(0\), e
ignoramos que la regresión lineal nos de una intersección distinta de
\((0,0)\).

Tomamos entonces \(f=0.2546\), y añadimos las aproximaciones a la tabla
de ``pandas''.

    \hypertarget{comparaciuxf3n-de-soluciuxf3n-con-datos-observados}{%
\subsection*{Comparación de Solución con Datos
Observados}\label{comparaciuxf3n-de-soluciuxf3n-con-datos-observados}}

    \begin{tcolorbox}[breakable, size=fbox, boxrule=1pt, pad at break*=1mm,colback=cellbackground, colframe=cellborder]
\prompt{In}{incolor}{40}{\boxspacing}
\begin{Verbatim}[commandchars=\\\{\}]
\PY{n}{ajuste\PYZus{}lineal} \PY{o}{=} \PY{k}{lambda} \PY{n}{x}\PY{p}{:} \PY{n}{m}\PY{o}{*}\PY{n}{x}
\PY{n}{solucion} \PY{o}{=} \PY{k}{lambda} \PY{n}{d}\PY{p}{:} \PY{p}{(}\PY{l+m+mi}{1}\PY{o}{+}\PY{n}{m}\PY{p}{)}\PY{o}{*}\PY{o}{*}\PY{n}{d} \PY{o}{*} \PY{n}{datos}\PY{p}{[}\PY{l+s+s2}{\PYZdq{}}\PY{l+s+s2}{\PYZdl{}I\PYZus{}d\PYZdl{}}\PY{l+s+s2}{\PYZdq{}}\PY{p}{]}\PY{p}{[}\PY{l+m+mi}{0}\PY{p}{]}
\PY{n}{datos}\PY{p}{[}\PY{l+s+s2}{\PYZdq{}}\PY{l+s+s2}{\PYZdl{}}\PY{l+s+se}{\PYZbs{}\PYZbs{}}\PY{l+s+s2}{hat I\PYZus{}d\PYZdl{}}\PY{l+s+s2}{\PYZdq{}}\PY{p}{]} \PY{o}{=} \PY{n}{datos}\PY{p}{[}\PY{p}{[}\PY{l+s+s2}{\PYZdq{}}\PY{l+s+s2}{\PYZdl{}d\PYZdl{}}\PY{l+s+s2}{\PYZdq{}}\PY{p}{]}\PY{p}{]}\PY{o}{.}\PY{n}{applymap}\PY{p}{(}\PY{n}{solucion}\PY{p}{)}
\PY{n}{datos}\PY{p}{[}\PY{l+s+s2}{\PYZdq{}}\PY{l+s+s2}{\PYZdl{}}\PY{l+s+se}{\PYZbs{}\PYZbs{}}\PY{l+s+s2}{Delta }\PY{l+s+se}{\PYZbs{}\PYZbs{}}\PY{l+s+s2}{hat I\PYZus{}d\PYZdl{}}\PY{l+s+s2}{\PYZdq{}}\PY{p}{]} \PY{o}{=} \PY{n}{datos}\PY{p}{[}\PY{p}{[}\PY{l+s+s2}{\PYZdq{}}\PY{l+s+s2}{\PYZdl{}I\PYZus{}d\PYZdl{}}\PY{l+s+s2}{\PYZdq{}}\PY{p}{]}\PY{p}{]}\PY{o}{.}\PY{n}{applymap}\PY{p}{(}\PY{n}{ajuste\PYZus{}lineal}\PY{p}{)}
\PY{c+c1}{\PYZsh{}print(datos.to\PYZus{}latex(index=False, escape=False))}
\end{Verbatim}
\end{tcolorbox}

    \begin{tabular}{rrrrr}
\toprule
 $d$ &  $I_d$ &  $\Delta I_d$ &  $\hat I_d$ &  $\Delta \hat I_d$ \\
\midrule
   0 &  0.842 &        -0.203 &       0.842 &             -0.214 \\
   1 &  0.639 &        -0.180 &       0.628 &             -0.163 \\
   2 &  0.459 &        -0.111 &       0.468 &             -0.117 \\
   3 &  0.348 &        -0.085 &       0.349 &             -0.089 \\
   4 &  0.263 &        -0.061 &       0.260 &             -0.067 \\
   5 &  0.202 &        -0.048 &       0.194 &             -0.051 \\
   6 &  0.154 &        -0.040 &       0.144 &             -0.039 \\
   7 &  0.114 &        -0.029 &       0.108 &             -0.029 \\
   8 &  0.085 &           NaN &       0.080 &             -0.022 \\
\bottomrule
\end{tabular}

    Se mantiene el color verde para los datos originales, y se usa azúl para
la solución del sistema dinámico que aproxima los datos.

    \begin{tcolorbox}[breakable, size=fbox, boxrule=1pt, pad at break*=1mm,colback=cellbackground, colframe=cellborder]
\prompt{In}{incolor}{41}{\boxspacing}
\begin{Verbatim}[commandchars=\\\{\}]
\PY{n}{display}\PY{p}{(}\PY{n}{datos}\PY{o}{.}\PY{n}{plot}\PY{p}{(}\PY{n}{x}\PY{o}{=}\PY{l+s+s2}{\PYZdq{}}\PY{l+s+s2}{\PYZdl{}I\PYZus{}d\PYZdl{}}\PY{l+s+s2}{\PYZdq{}}\PY{p}{,} \PY{n}{y}\PY{o}{=}\PY{p}{[}\PY{l+s+s2}{\PYZdq{}}\PY{l+s+s2}{\PYZdl{}}\PY{l+s+se}{\PYZbs{}\PYZbs{}}\PY{l+s+s2}{Delta I\PYZus{}d\PYZdl{}}\PY{l+s+s2}{\PYZdq{}}\PY{p}{,}\PY{l+s+s2}{\PYZdq{}}\PY{l+s+s2}{\PYZdl{}}\PY{l+s+se}{\PYZbs{}\PYZbs{}}\PY{l+s+s2}{Delta }\PY{l+s+se}{\PYZbs{}\PYZbs{}}\PY{l+s+s2}{hat I\PYZus{}d\PYZdl{}}\PY{l+s+s2}{\PYZdq{}}\PY{p}{]}\PY{p}{,}  \PY{n}{marker}\PY{o}{=}\PY{l+s+s2}{\PYZdq{}}\PY{l+s+s2}{o}\PY{l+s+s2}{\PYZdq{}}\PY{p}{,} \PY{n}{color}\PY{o}{=}\PY{p}{[}\PY{l+s+s2}{\PYZdq{}}\PY{l+s+s2}{g}\PY{l+s+s2}{\PYZdq{}}\PY{p}{,}\PY{l+s+s2}{\PYZdq{}}\PY{l+s+s2}{b}\PY{l+s+s2}{\PYZdq{}}\PY{p}{]}\PY{p}{,}
       \PY{n}{title}\PY{o}{=}\PY{l+s+s2}{\PYZdq{}}\PY{l+s+s2}{Comparación de Diferencias respecto Intensidad en Datos y en la Solución}\PY{l+s+s2}{\PYZdq{}}\PY{p}{)}\PY{p}{)}

\PY{n}{display}\PY{p}{(}\PY{n}{datos}\PY{o}{.}\PY{n}{plot}\PY{p}{(}\PY{n}{y}\PY{o}{=}\PY{p}{[}\PY{l+s+s2}{\PYZdq{}}\PY{l+s+s2}{\PYZdl{}I\PYZus{}d\PYZdl{}}\PY{l+s+s2}{\PYZdq{}}\PY{p}{,}\PY{l+s+s2}{\PYZdq{}}\PY{l+s+s2}{\PYZdl{}}\PY{l+s+se}{\PYZbs{}\PYZbs{}}\PY{l+s+s2}{hat I\PYZus{}d\PYZdl{}}\PY{l+s+s2}{\PYZdq{}}\PY{p}{]}\PY{p}{,}  \PY{n}{marker}\PY{o}{=}\PY{l+s+s2}{\PYZdq{}}\PY{l+s+s2}{o}\PY{l+s+s2}{\PYZdq{}}\PY{p}{,} \PY{n}{color}\PY{o}{=}\PY{p}{[}\PY{l+s+s2}{\PYZdq{}}\PY{l+s+s2}{g}\PY{l+s+s2}{\PYZdq{}}\PY{p}{,}\PY{l+s+s2}{\PYZdq{}}\PY{l+s+s2}{b}\PY{l+s+s2}{\PYZdq{}}\PY{p}{]}\PY{p}{,}
       \PY{n}{title}\PY{o}{=}\PY{l+s+s2}{\PYZdq{}}\PY{l+s+s2}{Comparación de Datos y la Solución}\PY{l+s+s2}{\PYZdq{}}\PY{p}{)}\PY{p}{)}
\end{Verbatim}
\end{tcolorbox}

    
    \begin{Verbatim}[commandchars=\\\{\}]
<AxesSubplot:title=\{'center':'Comparación de Diferencias respecto Intensidad en Datos y en la Solución'\}, xlabel='\$I\_d\$'>
    \end{Verbatim}

    
    
    \begin{Verbatim}[commandchars=\\\{\}]
<AxesSubplot:title=\{'center':'Comparación de Datos y la Solución'\}>
    \end{Verbatim}

    
    \begin{center}
    \adjustimage{max size={0.9\linewidth}{0.9\paperheight}}{output_11_2.png}
    \end{center}
    { \hspace*{\fill} \\}
    
    \begin{center}
    \adjustimage{max size={0.9\linewidth}{0.9\paperheight}}{output_11_3.png}
    \end{center}
    { \hspace*{\fill} \\}
    
    Podemos observar que nuestra aproximación está un poco por debajo de los
datos en los últimos puntos, pero en algunos otros por encima con un
pequeño margen.

    \hypertarget{comparaciuxf3n-del-agua-anterior-con-el-agua-vista-en-clase}{%
\subsection*{Comparación del Agua anterior con el Agua vista en
Clase}\label{comparaciuxf3n-del-agua-anterior-con-el-agua-vista-en-clase}}

    Llamemosle ``Agua \#1'' al agua a la que le corresponden los datos
anteriores de este ejercicio y ``Agua \#2'' al agua con los datos vistos
en clase. Para el Agua \#2 los datos del experimento nombraremos las
intensidades en el grosor \(d\) por \(I_d\). Donde las capas \(d\) son
del mismo grosor que en el Agua \#1 pero la opacidad podría ser
distinta.

    Para determinar que una es más opaca que la otra no es suficiente ver
los cambios en las capas correspondientes, pues el cambio es
proporcional a la intensidad al inicio de la capa y estas son distintas.
Pero con la solución del sistema dinámico para el agua \#1 podemos
comparar sustituyendo el valor inicial \(I'_0\) por \(I_0\) en

\[\hat I_d = (1-f)^{d}I_0.\]

Pues esto aproximaría cuanto sería la intensidad en el agua \#1 si se
empezara con la misma intensidad de luz en la superficie que en el agua
\#2. Nombremos \(\hat I'_d\) a tales aproximaciones; la intensidad de la
luz en la capa \(d\) en el agua \#1 si la intensidad de luz inicial
fuera la misma que en el agua \#2.

    \begin{tcolorbox}[breakable, size=fbox, boxrule=1pt, pad at break*=1mm,colback=cellbackground, colframe=cellborder]
\prompt{In}{incolor}{42}{\boxspacing}
\begin{Verbatim}[commandchars=\\\{\}]
\PY{n}{agua2} \PY{o}{=} \PY{p}{\PYZob{}}\PY{l+s+s1}{\PYZsq{}}\PY{l+s+s1}{\PYZdl{}d\PYZdl{}}\PY{l+s+s1}{\PYZsq{}}\PY{p}{:} \PY{p}{\PYZob{}}\PY{l+m+mi}{0}\PY{p}{:} \PY{l+m+mi}{0}\PY{p}{,} \PY{l+m+mi}{1}\PY{p}{:} \PY{l+m+mi}{1}\PY{p}{,} \PY{l+m+mi}{2}\PY{p}{:} \PY{l+m+mi}{2}\PY{p}{,} \PY{l+m+mi}{3}\PY{p}{:} \PY{l+m+mi}{3}\PY{p}{,} \PY{l+m+mi}{4}\PY{p}{:} \PY{l+m+mi}{4}\PY{p}{,} \PY{l+m+mi}{5}\PY{p}{:} \PY{l+m+mi}{5}\PY{p}{,} \PY{l+m+mi}{6}\PY{p}{:} \PY{l+m+mi}{6}\PY{p}{,} \PY{l+m+mi}{7}\PY{p}{:} \PY{l+m+mi}{7}\PY{p}{,} \PY{l+m+mi}{8}\PY{p}{:} \PY{l+m+mi}{8}\PY{p}{,} \PY{l+m+mi}{9}\PY{p}{:} \PY{l+m+mi}{9}\PY{p}{\PYZcb{}}\PY{p}{,}
         \PY{l+s+s1}{\PYZsq{}}\PY{l+s+s1}{\PYZdl{}I}\PY{l+s+se}{\PYZbs{}\PYZsq{}}\PY{l+s+s1}{\PYZus{}d\PYZdl{}}\PY{l+s+s1}{\PYZsq{}}\PY{p}{:} \PY{p}{\PYZob{}}\PY{l+m+mi}{0}\PY{p}{:} \PY{l+m+mf}{0.4}\PY{p}{,} \PY{l+m+mi}{1}\PY{p}{:} \PY{l+m+mf}{0.33}\PY{p}{,} \PY{l+m+mi}{2}\PY{p}{:} \PY{l+m+mf}{0.27}\PY{p}{,} \PY{l+m+mi}{3}\PY{p}{:} \PY{l+m+mf}{0.216}\PY{p}{,} \PY{l+m+mi}{4}\PY{p}{:} \PY{l+m+mf}{0.17}\PY{p}{,} \PY{l+m+mi}{5}\PY{p}{:} \PY{l+m+mf}{0.14}\PY{p}{,} \PY{l+m+mi}{6}\PY{p}{:} \PY{l+m+mf}{0.124}\PY{p}{,} \PY{l+m+mi}{7}\PY{p}{:} \PY{l+m+mf}{0.098}\PY{p}{,} \PY{l+m+mi}{8}\PY{p}{:} \PY{l+m+mf}{0.082}\PY{p}{,} \PY{l+m+mi}{9}\PY{p}{:} \PY{l+m+mf}{0.065}\PY{p}{\PYZcb{}}\PY{p}{,}
         \PY{l+s+s1}{\PYZsq{}}\PY{l+s+s1}{\PYZdl{}}\PY{l+s+se}{\PYZbs{}\PYZbs{}}\PY{l+s+s1}{Delta I}\PY{l+s+se}{\PYZbs{}\PYZsq{}}\PY{l+s+s1}{\PYZus{}d\PYZdl{}}\PY{l+s+s1}{\PYZsq{}}\PY{p}{:} \PY{p}{\PYZob{}}\PY{l+m+mi}{0}\PY{p}{:} \PY{o}{\PYZhy{}}\PY{l+m+mf}{0.07}\PY{p}{,} \PY{l+m+mi}{1}\PY{p}{:} \PY{o}{\PYZhy{}}\PY{l+m+mf}{0.06}\PY{p}{,} \PY{l+m+mi}{2}\PY{p}{:} \PY{o}{\PYZhy{}}\PY{l+m+mf}{0.054}\PY{p}{,} \PY{l+m+mi}{3}\PY{p}{:} \PY{o}{\PYZhy{}}\PY{l+m+mf}{0.046}\PY{p}{,} \PY{l+m+mi}{4}\PY{p}{:} \PY{o}{\PYZhy{}}\PY{l+m+mf}{0.03}\PY{p}{,} \PY{l+m+mi}{5}\PY{p}{:} \PY{o}{\PYZhy{}}\PY{l+m+mf}{0.016}\PY{p}{,} \PY{l+m+mi}{6}\PY{p}{:} \PY{o}{\PYZhy{}}\PY{l+m+mf}{0.026}\PY{p}{,} \PY{l+m+mi}{7}\PY{p}{:} \PY{o}{\PYZhy{}}\PY{l+m+mf}{0.016}\PY{p}{,} \PY{l+m+mi}{8}\PY{p}{:} \PY{o}{\PYZhy{}}\PY{l+m+mf}{0.017}\PY{p}{,} \PY{l+m+mi}{9}\PY{p}{:} \PY{k+kc}{None}\PY{p}{\PYZcb{}}\PY{p}{\PYZcb{}}
\PY{n}{datos2} \PY{o}{=} \PY{n}{pd}\PY{o}{.}\PY{n}{DataFrame}\PY{p}{(}\PY{n}{agua2}\PY{p}{)}
\PY{n}{datos2} \PY{o}{=} \PY{n}{datos}\PY{o}{.}\PY{n}{merge}\PY{p}{(}\PY{n}{datos2}\PY{p}{,} \PY{n}{how}\PY{o}{=}\PY{l+s+s1}{\PYZsq{}}\PY{l+s+s1}{outer}\PY{l+s+s1}{\PYZsq{}}\PY{p}{)}
\PY{n}{datos2}\PY{p}{[}\PY{l+s+s2}{\PYZdq{}}\PY{l+s+s2}{\PYZdl{}}\PY{l+s+s2}{\PYZbs{}}\PY{l+s+s2}{hat I}\PY{l+s+se}{\PYZbs{}\PYZsq{}}\PY{l+s+s2}{\PYZus{}d\PYZdl{}}\PY{l+s+s2}{\PYZdq{}}\PY{p}{]} \PY{o}{=} \PY{n}{datos2}\PY{o}{.}\PY{n}{iloc}\PY{p}{[}\PY{p}{:}\PY{p}{,}\PY{p}{[}\PY{l+m+mi}{0}\PY{p}{]}\PY{p}{]}\PY{o}{.}\PY{n}{applymap}\PY{p}{(}\PY{k}{lambda} \PY{n}{d}\PY{p}{:} \PY{p}{(}\PY{l+m+mi}{1}\PY{o}{+}\PY{n}{m}\PY{p}{)}\PY{o}{*}\PY{o}{*}\PY{n}{d}\PY{o}{*}\PY{n}{datos2}\PY{o}{.}\PY{n}{iloc}\PY{p}{[}\PY{l+m+mi}{0}\PY{p}{,}\PY{l+m+mi}{5}\PY{p}{]}\PY{p}{)}
\PY{c+c1}{\PYZsh{}print(datos2.iloc[:,[5,7]].to\PYZus{}latex(index=False, escape=False)) }
\end{Verbatim}
\end{tcolorbox}

    \begin{tabular}{rr}
\toprule
 $I'_d$ &  $\hat I'_d$ \\
\midrule
  0.400 &        0.400 \\
  0.330 &        0.298 \\
  0.270 &        0.222 \\
  0.216 &        0.166 \\
  0.170 &        0.124 \\
  0.140 &        0.092 \\
  0.124 &        0.069 \\
  0.098 &        0.051 \\
  0.082 &        0.038 \\
  0.065 &        0.028 \\
\bottomrule
\end{tabular}

    \begin{tcolorbox}[breakable, size=fbox, boxrule=1pt, pad at break*=1mm,colback=cellbackground, colframe=cellborder]
\prompt{In}{incolor}{43}{\boxspacing}
\begin{Verbatim}[commandchars=\\\{\}]
\PY{n}{datos2}\PY{o}{.}\PY{n}{plot}\PY{p}{(}\PY{n}{x}\PY{o}{=}\PY{l+s+s2}{\PYZdq{}}\PY{l+s+s2}{\PYZdl{}d\PYZdl{}}\PY{l+s+s2}{\PYZdq{}}\PY{p}{,} \PY{n}{y}\PY{o}{=}\PY{p}{[}\PY{l+s+s1}{\PYZsq{}}\PY{l+s+s1}{\PYZdl{}I}\PY{l+s+se}{\PYZbs{}\PYZsq{}}\PY{l+s+s1}{\PYZus{}d\PYZdl{}}\PY{l+s+s1}{\PYZsq{}}\PY{p}{,}\PY{l+s+s1}{\PYZsq{}}\PY{l+s+s1}{\PYZdl{}}\PY{l+s+s1}{\PYZbs{}}\PY{l+s+s1}{hat I}\PY{l+s+se}{\PYZbs{}\PYZsq{}}\PY{l+s+s1}{\PYZus{}d\PYZdl{}}\PY{l+s+s1}{\PYZsq{}}\PY{p}{]}\PY{p}{,} \PY{n}{marker}\PY{o}{=}\PY{l+s+s2}{\PYZdq{}}\PY{l+s+s2}{o}\PY{l+s+s2}{\PYZdq{}}\PY{p}{,} \PY{n}{color}\PY{o}{=}\PY{p}{[}\PY{l+s+s2}{\PYZdq{}}\PY{l+s+s2}{g}\PY{l+s+s2}{\PYZdq{}}\PY{p}{,}\PY{l+s+s2}{\PYZdq{}}\PY{l+s+s2}{b}\PY{l+s+s2}{\PYZdq{}}\PY{p}{]}\PY{p}{)}
\end{Verbatim}
\end{tcolorbox}

            \begin{tcolorbox}[breakable, size=fbox, boxrule=.5pt, pad at break*=1mm, opacityfill=0]
\prompt{Out}{outcolor}{43}{\boxspacing}
\begin{Verbatim}[commandchars=\\\{\}]
<AxesSubplot:xlabel='\$d\$'>
\end{Verbatim}
\end{tcolorbox}
        
    \begin{center}
    \adjustimage{max size={0.9\linewidth}{0.9\paperheight}}{output_18_1.png}
    \end{center}
    { \hspace*{\fill} \\}
    
    Como la estimación es mucho menor usando la fracción del agua \#1 para
estimar la del agua \#2; quiere decir que la fracción de intensidad de
luz que pierde el agua \#1 es mayor que la que pierde el agua \#2. Es
decir, el agua \#1 (de esta tarea) es menos clara que la del agua \#2
(que vimos en clase).

    \hypertarget{ejercicio-2-tiempos-de-duplicado-y-reducciuxf3n-a-la-mitad}{%
\section*{Ejercicio 2: Tiempos de Duplicado y Reducción a la
Mitad}\label{ejercicio-2-tiempos-de-duplicado-y-reducciuxf3n-a-la-mitad}}

Sea \(y(t)=A\cdot B^{kt}\) para \(A,B,k >0\). El tiempo \(t_r\) (si
existe) con \(r> 0\) para que \(y\) alcance \(r\) veces su valor
partiendo del tiempo \(t_0\) inicial está dado por la ecuación

\[A\cdot B^{k(t_r+t_0)} = rA\cdot B^{k t_0},\] \[ B^{k t_r} = r.\]

Por lo que es independiente del valor de \(A\). Si \(\log\) es el
logaritmo natural

\[ B^{k t_r} = \exp(k t_r\log B) =r,\]

entonces si \(r > 0\)

\[k t_r \log B = \log r,\] \[t_r = \frac{\log r}{k \log B}.\]

Entonces
\[t_{1/r} = \frac{\log 1/r}{k \log B} = \frac{\log 1-\log r}{k \log B} = -\frac{\log r}{k \log B},\]
\[t_{1/r} = t_{r}.\]

Un \(t_r\) negativo interpretado como tiempo es que hace \(|t_r|\) de
\(t_0\) se alcanzó \(r\) veces \(y(t_0)\).

En particular cuando \(B>1\) y \(r=2\) se tiene \(t_2\) el tiempo de
duplicado. Cuando \(B<1\) y \(r=1/2\) se tiene el tiempo \(t_{1/2}\) de
reducción a la mitad.

    \hypertarget{a-y-100-cdot-0.8t}{%
\subsection*{\texorpdfstring{a)
\(y = 100 \cdot 0.8^t\)}{a) y = 100 \textbackslash cdot 0.8\^{}t}}\label{a-y-100-cdot-0.8t}}

Como \(0.8<1\) solo hay tiempo de reducción a la mitad:
\[t_{1/2} = -\frac{\log 2}{\log 0.8}.\]

    \begin{tcolorbox}[breakable, size=fbox, boxrule=1pt, pad at break*=1mm,colback=cellbackground, colframe=cellborder]
\prompt{In}{incolor}{44}{\boxspacing}
\begin{Verbatim}[commandchars=\\\{\}]
\PY{k+kn}{from} \PY{n+nn}{math} \PY{k+kn}{import} \PY{n}{log}\PY{p}{,}\PY{n}{exp}
\PY{o}{\PYZhy{}}\PY{n}{log}\PY{p}{(}\PY{l+m+mi}{2}\PY{p}{)}\PY{o}{/}\PY{n}{log}\PY{p}{(}\PY{l+m+mf}{0.8}\PY{p}{)}
\end{Verbatim}
\end{tcolorbox}

            \begin{tcolorbox}[breakable, size=fbox, boxrule=.5pt, pad at break*=1mm, opacityfill=0]
\prompt{Out}{outcolor}{44}{\boxspacing}
\begin{Verbatim}[commandchars=\\\{\}]
3.1062837195053903
\end{Verbatim}
\end{tcolorbox}
        
    \hypertarget{b-y4cdot-53t}{%
\subsection*{\texorpdfstring{b)
\(y=4\cdot 5^{3t}\)}{b) y=4\textbackslash cdot 5\^{}\{3t\}}}\label{b-y4cdot-53t}}

Tiempo de duplicado: \[t_2= \frac{\log 2}{3\log 5}.\]

    \begin{tcolorbox}[breakable, size=fbox, boxrule=1pt, pad at break*=1mm,colback=cellbackground, colframe=cellborder]
\prompt{In}{incolor}{45}{\boxspacing}
\begin{Verbatim}[commandchars=\\\{\}]
\PY{n}{log}\PY{p}{(}\PY{l+m+mi}{2}\PY{p}{)}\PY{o}{/}\PY{p}{(}\PY{l+m+mi}{3}\PY{o}{*}\PY{n}{log}\PY{p}{(}\PY{l+m+mi}{5}\PY{p}{)}\PY{p}{)}
\end{Verbatim}
\end{tcolorbox}

            \begin{tcolorbox}[breakable, size=fbox, boxrule=.5pt, pad at break*=1mm, opacityfill=0]
\prompt{Out}{outcolor}{45}{\boxspacing}
\begin{Verbatim}[commandchars=\\\{\}]
0.14355885269113103
\end{Verbatim}
\end{tcolorbox}
        
    \hypertarget{c-y10cdot-0.82t}{%
\subsection*{\texorpdfstring{c)
\(y=10\cdot 0.8^{2t}\)}{c) y=10\textbackslash cdot 0.8\^{}\{2t\}}}\label{c-y10cdot-0.82t}}

Tiempo de reducción a la mitad:
\[t_{1/2} = -\frac{\log 2}{2 \log 0.8}.\]

    \begin{tcolorbox}[breakable, size=fbox, boxrule=1pt, pad at break*=1mm,colback=cellbackground, colframe=cellborder]
\prompt{In}{incolor}{46}{\boxspacing}
\begin{Verbatim}[commandchars=\\\{\}]
\PY{o}{\PYZhy{}}\PY{n}{log}\PY{p}{(}\PY{l+m+mi}{2}\PY{p}{)}\PY{o}{/}\PY{p}{(}\PY{l+m+mi}{2}\PY{o}{*}\PY{n}{log}\PY{p}{(}\PY{l+m+mf}{0.8}\PY{p}{)}\PY{p}{)}
\end{Verbatim}
\end{tcolorbox}

            \begin{tcolorbox}[breakable, size=fbox, boxrule=.5pt, pad at break*=1mm, opacityfill=0]
\prompt{Out}{outcolor}{46}{\boxspacing}
\begin{Verbatim}[commandchars=\\\{\}]
1.5531418597526951
\end{Verbatim}
\end{tcolorbox}
        
    \hypertarget{d-y0.010.1-t}{%
\subsection*{\texorpdfstring{d)
\(y=0.01^{0.1 t}\)}{d) y=0.01\^{}\{0.1 t\}}}\label{d-y0.010.1-t}}

Tiempo de reducción a la mitad: \[-\frac{\log 2}{0.1 \log 0.01}.\]

    \begin{tcolorbox}[breakable, size=fbox, boxrule=1pt, pad at break*=1mm,colback=cellbackground, colframe=cellborder]
\prompt{In}{incolor}{47}{\boxspacing}
\begin{Verbatim}[commandchars=\\\{\}]
\PY{o}{\PYZhy{}}\PY{n}{log}\PY{p}{(}\PY{l+m+mi}{2}\PY{p}{)}\PY{o}{/}\PY{p}{(}\PY{l+m+mf}{0.1}\PY{o}{*}\PY{n}{log}\PY{p}{(}\PY{l+m+mf}{0.01}\PY{p}{)}\PY{p}{)}
\end{Verbatim}
\end{tcolorbox}

            \begin{tcolorbox}[breakable, size=fbox, boxrule=.5pt, pad at break*=1mm, opacityfill=0]
\prompt{Out}{outcolor}{47}{\boxspacing}
\begin{Verbatim}[commandchars=\\\{\}]
1.505149978319906
\end{Verbatim}
\end{tcolorbox}
        
    \hypertarget{ejercicio-3}{%
\section*{Ejercicio 3}\label{ejercicio-3}}

Dada una inversión inicial \(A_0\) con una tasa de interés compuesta
\(r\), a los \(t\) años se tiene \[A_t = A_0 (1+r)^t.\] Las unidades
representan dólares.

\hypertarget{a-encontrar-a_10-si-a_01-r0.06.}{%
\subsection*{\texorpdfstring{a) Encontrar \(A_{10}\) si \(A_0=1\),
\(r=0.06\).}{a) Encontrar A\_\{10\} si A\_0=1, r=0.06.}}\label{a-encontrar-a_10-si-a_01-r0.06.}}

\[A_{10} = 1\cdot (1+0.06)^{10} = 1.06^{10}.\]

    \begin{tcolorbox}[breakable, size=fbox, boxrule=1pt, pad at break*=1mm,colback=cellbackground, colframe=cellborder]
\prompt{In}{incolor}{48}{\boxspacing}
\begin{Verbatim}[commandchars=\\\{\}]
\PY{l+m+mf}{1.06}\PY{o}{*}\PY{o}{*}\PY{l+m+mi}{10}
\end{Verbatim}
\end{tcolorbox}

            \begin{tcolorbox}[breakable, size=fbox, boxrule=.5pt, pad at break*=1mm, opacityfill=0]
\prompt{Out}{outcolor}{48}{\boxspacing}
\begin{Verbatim}[commandchars=\\\{\}]
1.7908476965428546
\end{Verbatim}
\end{tcolorbox}
        
    \hypertarget{b-para-que-valor-de-r-se-tendruxeda-a_82-si-a_01}{%
\subsection*{\texorpdfstring{b) ¿Para que valor de \(r\) se tendría
\(A_8=2\) si
\(A_0=1\)?}{b) ¿Para que valor de r se tendría A\_8=2 si A\_0=1?}}\label{b-para-que-valor-de-r-se-tendruxeda-a_82-si-a_01}}

Se busca \(r\) tal que \[2 = (1+r)^8,\] \[2^{1/8} = 1+r,\]
\[r = 2^{1/8}-1.\]

    \begin{tcolorbox}[breakable, size=fbox, boxrule=1pt, pad at break*=1mm,colback=cellbackground, colframe=cellborder]
\prompt{In}{incolor}{49}{\boxspacing}
\begin{Verbatim}[commandchars=\\\{\}]
\PY{l+m+mi}{2}\PY{o}{*}\PY{o}{*}\PY{p}{(}\PY{l+m+mi}{1}\PY{o}{/}\PY{l+m+mi}{8}\PY{p}{)}\PY{o}{\PYZhy{}}\PY{l+m+mi}{1}
\end{Verbatim}
\end{tcolorbox}

            \begin{tcolorbox}[breakable, size=fbox, boxrule=.5pt, pad at break*=1mm, opacityfill=0]
\prompt{Out}{outcolor}{49}{\boxspacing}
\begin{Verbatim}[commandchars=\\\{\}]
0.09050773266525769
\end{Verbatim}
\end{tcolorbox}
        
    \hypertarget{c-verificar-si-una-inversiuxf3n-se-duplica-a-una-tasa-de-interuxe9s-de-r-se-duplica-en-72r-auxf1os-para-r468912.}{%
\subsection*{\texorpdfstring{c) Verificar si una inversión se duplica a
una tasa de interés de \(R\%\) se duplica en \(72/R\) años; para
\(R=4,6,8,9,12\).}{c) Verificar si una inversión se duplica a una tasa de interés de R\textbackslash\% se duplica en 72/R años; para R=4,6,8,9,12.}}\label{c-verificar-si-una-inversiuxf3n-se-duplica-a-una-tasa-de-interuxe9s-de-r-se-duplica-en-72r-auxf1os-para-r468912.}}

    Sabemos que el tiempo que buscamos viene de la ecuación
\(y=A\cdot B^kt\) como en el ejercicio 2; \(B=1+R/100\), \(k=1\) y la
inversión inicial es \(A>0\). Y es
\(t_2=\frac{\log 2}{\log B} = \frac{\log 2}{\log (1+R/100)}\) (sin
depender de la inversión inicial mientras no sea 0).

Así que verificamos comprobando contra el valor real.

    \begin{tcolorbox}[breakable, size=fbox, boxrule=1pt, pad at break*=1mm,colback=cellbackground, colframe=cellborder]
\prompt{In}{incolor}{50}{\boxspacing}
\begin{Verbatim}[commandchars=\\\{\}]
\PY{n}{datos3c} \PY{o}{=} \PY{n}{pd}\PY{o}{.}\PY{n}{DataFrame}\PY{p}{(}\PY{n+nb}{map}\PY{p}{(}\PY{k}{lambda} \PY{n}{R}\PY{p}{:} \PY{p}{(}\PY{n}{R}\PY{p}{,} \PY{n}{log}\PY{p}{(}\PY{l+m+mi}{2}\PY{p}{)}\PY{o}{/}\PY{p}{(}\PY{n}{log}\PY{p}{(}\PY{l+m+mi}{1}\PY{o}{+}\PY{n}{R}\PY{o}{/}\PY{l+m+mi}{100}\PY{p}{)}\PY{p}{)}\PY{p}{,} \PY{l+m+mi}{72}\PY{o}{/}\PY{n}{R}\PY{p}{)}\PY{p}{,} \PY{p}{(}\PY{l+m+mi}{4}\PY{p}{,}\PY{l+m+mi}{6}\PY{p}{,}\PY{l+m+mi}{8}\PY{p}{,}\PY{l+m+mi}{9}\PY{p}{,}\PY{l+m+mi}{12}\PY{p}{)}\PY{p}{)}\PY{p}{,}
             \PY{n}{columns} \PY{o}{=} \PY{p}{[}\PY{l+s+s2}{\PYZdq{}}\PY{l+s+s2}{\PYZdl{}R\PYZdl{}}\PY{l+s+s2}{\PYZdq{}}\PY{p}{,} \PY{l+s+s2}{\PYZdq{}}\PY{l+s+s2}{\PYZdl{}t\PYZus{}2\PYZdl{}}\PY{l+s+s2}{\PYZdq{}}\PY{p}{,} \PY{l+s+s2}{\PYZdq{}}\PY{l+s+s2}{\PYZdl{}72/R\PYZdl{}}\PY{l+s+s2}{\PYZdq{}}\PY{p}{]}\PY{p}{)}
\PY{c+c1}{\PYZsh{}print(datos3c.to\PYZus{}latex(index=False, escape=False)) }
\end{Verbatim}
\end{tcolorbox}

    \begin{tabular}{rrr}
\toprule
 $R$ &  $t_2$ &  $72/R$ \\
\midrule
   4 & 17.673 &    18.0 \\
   6 & 11.896 &    12.0 \\
   8 &  9.006 &     9.0 \\
   9 &  8.043 &     8.0 \\
  12 &  6.116 &     6.0 \\
\bottomrule
\end{tabular}

    Se puede ver que la aproximación \(72/R\) es bastante cercana al tiempo
de duplicado verdadero.

    \hypertarget{ejercicio-4}{%
\section*{Ejercicio 4}\label{ejercicio-4}}

\hypertarget{a-una-cuxe9lula-de-la-bacteria-dicha-tiene-masa-2times-10-11-gramos.-si-en-un-cultivo-de-tal-bacteria-en-el-tiempo-0-hay-107-cuxe9lulas-cuuxe1l-es-la-masa-de-bacterias-en-el-cultivo-en-el-tiempo-0}{%
\subsection*{\texorpdfstring{a) Una célula de la bacteria dicha tiene
masa \(2\times 10^{-11}\) gramos. Si en un cultivo de tal bacteria en el
tiempo \(0\) hay \(10^7\) células, ¿cuál es la masa de bacterias en el
cultivo en el tiempo
\(0\)?}{a) Una célula de la bacteria dicha tiene masa 2\textbackslash times 10\^{}\{-11\} gramos. Si en un cultivo de tal bacteria en el tiempo 0 hay 10\^{}7 células, ¿cuál es la masa de bacterias en el cultivo en el tiempo 0?}}\label{a-una-cuxe9lula-de-la-bacteria-dicha-tiene-masa-2times-10-11-gramos.-si-en-un-cultivo-de-tal-bacteria-en-el-tiempo-0-hay-107-cuxe9lulas-cuuxe1l-es-la-masa-de-bacterias-en-el-cultivo-en-el-tiempo-0}}

    Como hay \(10^7\) células de masa \(2\times 10^{-11}\) gramos se tiene
una masa de \(2\times 10^{7-11} = 2\times 10^{-4}\) gramos de bacterias
en el cultivo.

    \hypertarget{b-supongamos-que-para-esta-bacteria-el-tiempo-de-duplicaciuxf3n-es-de-30-minutos.-cuuxe1nto-tiempo-en-minutos-tomaruxeda-para-que-el-cultivo-del-inciso-a-llegue-a-una-masa-de-un-gramo}{%
\subsection*{b) Supongamos que para esta bacteria el tiempo de
duplicación es de 30 minutos. ¿Cuánto tiempo en minutos tomaría para que
el cultivo del inciso a) llegue a una masa de un
gramo?}\label{b-supongamos-que-para-esta-bacteria-el-tiempo-de-duplicaciuxf3n-es-de-30-minutos.-cuuxe1nto-tiempo-en-minutos-tomaruxeda-para-que-el-cultivo-del-inciso-a-llegue-a-una-masa-de-un-gramo}}

    Sea \(t\) el tiempo en intervalos de 30 minutos, es decir \(t\) denota
el tiempo \(t\times 30\) minutos. Sea \(M_t\) la masa en el tiempo \(t\)
en gramos. Tenemos que \[M_{t+1} = 2 M_t,\] y para esta ecuación
dinámica conocemos que la solución es \[M_t = 2^t M_0.\]

    Recordemos que \(M_0=2\times10^{−4}\). Para que \(M_t\) sea \(1\) gramo,
\[1 = 2^t M_0,\] \[t = -\frac{\log M_0}{\log 2}.\] O \(t\times 30\)
minutos.

    \begin{tcolorbox}[breakable, size=fbox, boxrule=1pt, pad at break*=1mm,colback=cellbackground, colframe=cellborder]
\prompt{In}{incolor}{51}{\boxspacing}
\begin{Verbatim}[commandchars=\\\{\}]
\PY{p}{(}\PY{o}{\PYZhy{}}\PY{n}{log}\PY{p}{(}\PY{l+m+mi}{2}\PY{o}{*}\PY{l+m+mi}{10}\PY{o}{*}\PY{o}{*}\PY{p}{(}\PY{o}{\PYZhy{}}\PY{l+m+mi}{4}\PY{p}{)}\PY{p}{)}\PY{o}{/}\PY{n}{log}\PY{p}{(}\PY{l+m+mi}{2}\PY{p}{)}\PY{p}{)}\PY{o}{*}\PY{l+m+mi}{30}
\end{Verbatim}
\end{tcolorbox}

            \begin{tcolorbox}[breakable, size=fbox, boxrule=.5pt, pad at break*=1mm, opacityfill=0]
\prompt{Out}{outcolor}{51}{\boxspacing}
\begin{Verbatim}[commandchars=\\\{\}]
368.6313713864835
\end{Verbatim}
\end{tcolorbox}
        
    Es decir tomaría \(368\) minutos y poco más de medio minuto para llegar
a un gramo.

    \hypertarget{c}{%
\subsection*{c)}\label{c}}

\hypertarget{cuuxe1ntos-minutos-para-que-llegue-m_t-a-ser-6times-1027-la-masa-en-gramos-de-la-tierra-cuuxe1ntas-horas}{%
\subsubsection*{\texorpdfstring{¿Cuántos minutos para que llegue \(M_t\)
a ser \(6\times 10^{27}\) (la masa en gramos de la Tierra)? ¿Cuántas
horas?}{¿Cuántos minutos para que llegue M\_t a ser 6\textbackslash times 10\^{}\{27\} (la masa en gramos de la Tierra)? ¿Cuántas horas?}}\label{cuuxe1ntos-minutos-para-que-llegue-m_t-a-ser-6times-1027-la-masa-en-gramos-de-la-tierra-cuuxe1ntas-horas}}

\(t\) debe ser tal que \[6\times 10^{27} = 2^t M_0,\]
\[t = \frac{\log(6\times 10^{27})-\log M_0}{\log 2}.\]

    \begin{tcolorbox}[breakable, size=fbox, boxrule=1pt, pad at break*=1mm,colback=cellbackground, colframe=cellborder]
\prompt{In}{incolor}{52}{\boxspacing}
\begin{Verbatim}[commandchars=\\\{\}]
\PY{n}{t} \PY{o}{=} \PY{p}{(}\PY{n}{log}\PY{p}{(}\PY{l+m+mi}{6}\PY{o}{*}\PY{l+m+mi}{10}\PY{o}{*}\PY{o}{*}\PY{l+m+mi}{27}\PY{p}{)}\PY{o}{\PYZhy{}}\PY{n}{log}\PY{p}{(}\PY{l+m+mi}{2}\PY{o}{*}\PY{l+m+mi}{10}\PY{o}{*}\PY{o}{*}\PY{o}{\PYZhy{}}\PY{l+m+mi}{4}\PY{p}{)}\PY{p}{)}\PY{o}{/}\PY{n}{log}\PY{p}{(}\PY{l+m+mi}{2}\PY{p}{)}
\PY{n+nb}{print}\PY{p}{(}\PY{l+s+s2}{\PYZdq{}}\PY{l+s+s2}{En minutos: }\PY{l+s+s2}{\PYZdq{}}\PY{p}{,} \PY{n}{t}\PY{o}{*}\PY{l+m+mi}{30}\PY{p}{)}
\PY{n+nb}{print}\PY{p}{(}\PY{l+s+s2}{\PYZdq{}}\PY{l+s+s2}{En horas: }\PY{l+s+s2}{\PYZdq{}}\PY{p}{,} \PY{n}{t}\PY{o}{/}\PY{l+m+mi}{2}\PY{p}{)}
\end{Verbatim}
\end{tcolorbox}

    \begin{Verbatim}[commandchars=\\\{\}]
En minutos:  3136.942003266882
En horas:  52.2823667211147
    \end{Verbatim}

    \hypertarget{por-quuxe9-no-estuxe1mos-preocupados-por-esto-en-el-laboratorio-por-quuxe9-no-ha-sucedido-esto-ya-en-la-naturaleza}{%
\subsubsection*{¿Por qué no estámos preocupados por esto en el
laboratorio? ¿Por qué no ha sucedido esto ya en la
naturaleza?}\label{por-quuxe9-no-estuxe1mos-preocupados-por-esto-en-el-laboratorio-por-quuxe9-no-ha-sucedido-esto-ya-en-la-naturaleza}}

El modelo dónde las diferencias de densidad relativas a la propia
densidad de la población bacteriana crece linealmente, tiene como
supuesto que las condiciones son similares en todo el intervalo de
tiempo. Una de las condiciones es que la densidad este debajo de cierto
nivel para que el metabolismo máximo de las bacterias consuma una
cantidad de nutrimentos menor o igual a la que hay en su entorno; por lo
tanto se estarían dividiendo a velocidad constante.

No en todas partes del laboratorio habría suficientes nutrimentos para
las bacterias ni mucho menos en toda la superficie de la tierra (que
además es solo una fracción de la masa total de la Tierra). Además el
modelo no toma en cuenta depredadores o meramente amenazas que corten
sus nutrimentos o directamente las maten.

\hypertarget{explique-por-quuxe9-no-es-una-buena-idea-extrapolar-los-resultados-mucho-muxe1s-alluxe1-del-punto-final-de-la-recolecciuxf3n-de-datos.}{%
\subsubsection*{Explique por qué no es una buena idea extrapolar los
resultados mucho más allá del punto final de la recolección de
datos.}\label{explique-por-quuxe9-no-es-una-buena-idea-extrapolar-los-resultados-mucho-muxe1s-alluxe1-del-punto-final-de-la-recolecciuxf3n-de-datos.}}

Porque es desconocido que si hubieramos seguido recolectando datos las
condiciones iban a ser las mismas. Y la variación pudo haber sido mayor,
descartando que un solo ajuste lineal pueda aproximar bien todos los
datos de muestras más grandes.

Pero incluso en el mejor de los casos: Cuando hay un modelo lineal que
mejor aproxima una parte del fenómeno; del cuál se puede obtener un
modelo para el fenómeno que nos interesa (Como obtener una solución del
sistema dinámico en un modelo de crecimiento.) El error de medición es
prácticamente inevitable. Si está dado tal error de medición por la
misma variable aleatoria para cada medición, un mejor modelo surge de
una muestra más grande en relación con el teorema del límite central.

Si pensamos en estos modelos lineales como rectas, sabemos que la
distancia de rectas no paralelas incrementa conforme se aleja de la
intersección de estas. Al aplicar regresión lineal y obtener un vector
\(x_0\) y una transformación lineal \(T\) se tiene una aproximación
\(Tx+x_0\) que minimiza la distancia a la muestra. Pero conforme crece
la norma de \(x\) nos alejamos de \(x_0\) que es un punto que está cerca
de la muestra, es decir nos alejamos de la intersección con el mejor
modelo y crecería el error. (Resultado de Espacios Vectoriales Normados
de Dimensión Finita.)

    \hypertarget{ejercicio-5-modelo-para-el-crecimiento-del-moho}{%
\section*{Ejercicio 5: Modelo para el Crecimiento del
Moho}\label{ejercicio-5-modelo-para-el-crecimiento-del-moho}}

El aumento en el área de la colonia de moho durante cualquier intervalo
de tiempo es proporcional a la longitud de la circunferencia de la
colonia en el punto medio del intervalo de tiempo.

\hypertarget{a-explique-la-derivaciuxf3n-de-la-ecuaciuxf3n-dinuxe1mica}{%
\subsection*{a) Explique la derivación de la ecuación
dinámica}\label{a-explique-la-derivaciuxf3n-de-la-ecuaciuxf3n-dinuxe1mica}}

\[A_{t+1}-A_{t-1} = k C_t.\]

    Se habla del aumento en el área de la colonia de moho durante cualquier
intervalo de tiempo, entonces tomamos el intervalo \([t-1,t+1]\), el
punto medio de este intervalo es \(t\). El aumento en este intervalo es
\(A_{t+1}-A_{t-1}\) y se dice proporcional a \(C_t\) en el modelo, la
longitud de circunferencia de la colonia en el punto medio \(t\) del
intervalo. Digamos que es proporcional con constante \(k\) y tenemos
\[A_{t+1}-A_{t-1} = k C_t.\]

    Si se sustituye \(C_t\) por \(2\sqrt{\pi}\sqrt{A_t}\), y
\(K=2\sqrt{\pi}\) se tiene \[A_{t+1}-A_{t-1} = K\sqrt{A_t}.\]

\hypertarget{b-probar-que-a_t-frack216-t2-es-soluciuxf3n-en-la-uxfaltima-ecuaciuxf3n.}{%
\subsection*{\texorpdfstring{b) Probar que \[A_t = \frac{K^2}{16} t^2\]
es solución en la última
ecuación.}{b) Probar que A\_t = \textbackslash frac\{K\^{}2\}\{16\} t\^{}2 es solución en la última ecuación.}}\label{b-probar-que-a_t-frack216-t2-es-soluciuxf3n-en-la-uxfaltima-ecuaciuxf3n.}}

    Es solución si al sustituir lo anterior en la ecuación, la ecuación es
verdadera.
\[A_{t+1} = \frac{K^2}{16} (t+1)^2 = \frac{K^2}{16} (t^2+2t+1),\]
\[A_{t-1} = \frac{K^2}{16} (t-1)^2 = \frac{K^2}{16} (t^2-2t+1).\]
Entonces restando estas cantidades se tiene
\[A_{t+1}-A_{t-1} = \frac{K^2}{16} 4t = \frac{K^2}{4} t.\] En
sustituyendo \(A_t = \frac{K^2}{16} t^2\) en \(K\sqrt{A_t}\) se tiene
\[K\sqrt{A_t} = K\left(\frac{K}{4} t\right) = \frac{K^2}{4} t.\] Con lo
que se prueba que \(A_t = \frac{K^2}{16} t^2\) es solución de
\[A_{t+1}-A_{t-1} = K\sqrt{A_t}.\]

    \hypertarget{c-graficar-y-encontrar-una-soluciuxf3n-que-modele-los-datos}{%
\subsection*{c) Graficar y encontrar una solución que modele los
datos}\label{c-graficar-y-encontrar-una-soluciuxf3n-que-modele-los-datos}}

    \begin{tcolorbox}[breakable, size=fbox, boxrule=1pt, pad at break*=1mm,colback=cellbackground, colframe=cellborder]
\prompt{In}{incolor}{53}{\boxspacing}
\begin{Verbatim}[commandchars=\\\{\}]
\PY{c+c1}{\PYZsh{}Datos5c = pd.read\PYZus{}clipboard(header=[0])}
\PY{c+c1}{\PYZsh{}print(Datos5c.to\PYZus{}dict())}
\PY{n}{Datos5c} \PY{o}{=} \PY{n}{pd}\PY{o}{.}\PY{n}{DataFrame}\PY{p}{(}
    \PY{p}{\PYZob{}}\PY{l+s+s1}{\PYZsq{}}\PY{l+s+s1}{\PYZdl{}t\PYZdl{}}\PY{l+s+s1}{\PYZsq{}}\PY{p}{:} \PY{p}{\PYZob{}}\PY{l+m+mi}{0}\PY{p}{:} \PY{l+m+mi}{0}\PY{p}{,} \PY{l+m+mi}{1}\PY{p}{:} \PY{l+m+mi}{1}\PY{p}{,} \PY{l+m+mi}{2}\PY{p}{:} \PY{l+m+mi}{2}\PY{p}{,} \PY{l+m+mi}{3}\PY{p}{:} \PY{l+m+mi}{3}\PY{p}{,} \PY{l+m+mi}{4}\PY{p}{:} \PY{l+m+mi}{4}\PY{p}{,} \PY{l+m+mi}{5}\PY{p}{:} \PY{l+m+mi}{5}\PY{p}{,} \PY{l+m+mi}{6}\PY{p}{:} \PY{l+m+mi}{6}\PY{p}{,} \PY{l+m+mi}{7}\PY{p}{:} \PY{l+m+mi}{7}\PY{p}{,} \PY{l+m+mi}{8}\PY{p}{:} \PY{l+m+mi}{8}\PY{p}{,} \PY{l+m+mi}{9}\PY{p}{:} \PY{l+m+mi}{9}\PY{p}{\PYZcb{}}\PY{p}{,}
     \PY{l+s+s1}{\PYZsq{}}\PY{l+s+s1}{\PYZdl{}A\PYZus{}t\PYZdl{}}\PY{l+s+s1}{\PYZsq{}}\PY{p}{:} \PY{p}{\PYZob{}}\PY{l+m+mi}{0}\PY{p}{:} \PY{l+m+mi}{4}\PY{p}{,} \PY{l+m+mi}{1}\PY{p}{:} \PY{l+m+mi}{8}\PY{p}{,} \PY{l+m+mi}{2}\PY{p}{:} \PY{l+m+mi}{24}\PY{p}{,} \PY{l+m+mi}{3}\PY{p}{:} \PY{l+m+mi}{46}\PY{p}{,} \PY{l+m+mi}{4}\PY{p}{:} \PY{l+m+mi}{84}\PY{p}{,} \PY{l+m+mi}{5}\PY{p}{:} \PY{l+m+mi}{126}\PY{p}{,} \PY{l+m+mi}{6}\PY{p}{:} \PY{l+m+mi}{176}\PY{p}{,} \PY{l+m+mi}{7}\PY{p}{:} \PY{l+m+mi}{248}\PY{p}{,} \PY{l+m+mi}{8}\PY{p}{:} \PY{l+m+mi}{326}\PY{p}{,} \PY{l+m+mi}{9}\PY{p}{:} \PY{l+m+mi}{420}\PY{p}{\PYZcb{}}\PY{p}{\PYZcb{}}\PY{p}{)}
\end{Verbatim}
\end{tcolorbox}

    \begin{tcolorbox}[breakable, size=fbox, boxrule=1pt, pad at break*=1mm,colback=cellbackground, colframe=cellborder]
\prompt{In}{incolor}{54}{\boxspacing}
\begin{Verbatim}[commandchars=\\\{\}]
\PY{k+kn}{from} \PY{n+nn}{math} \PY{k+kn}{import} \PY{n}{sqrt}
\PY{n}{Datos5c}\PY{p}{[}\PY{l+s+s2}{\PYZdq{}}\PY{l+s+s2}{\PYZdl{}}\PY{l+s+s2}{\PYZbs{}}\PY{l+s+s2}{sqrt}\PY{l+s+si}{\PYZob{}A\PYZus{}t\PYZcb{}}\PY{l+s+s2}{\PYZdl{}}\PY{l+s+s2}{\PYZdq{}}\PY{p}{]} \PY{o}{=} \PY{n}{Datos5c}\PY{p}{[}\PY{l+s+s2}{\PYZdq{}}\PY{l+s+s2}{\PYZdl{}A\PYZus{}t\PYZdl{}}\PY{l+s+s2}{\PYZdq{}}\PY{p}{]}\PY{o}{.}\PY{n}{apply}\PY{p}{(}\PY{n}{sqrt}\PY{p}{)}

\PY{k}{def} \PY{n+nf}{diferencia\PYZus{}A}\PY{p}{(}\PY{n}{t}\PY{p}{,} \PY{n}{columna}\PY{p}{)}\PY{p}{:}
    \PY{k}{if} \PY{n}{t} \PY{o}{==} \PY{l+m+mi}{0}\PY{p}{:}
        \PY{k}{return} \PY{k+kc}{None}
    \PY{k}{if} \PY{n}{t} \PY{o}{==} \PY{p}{(}\PY{n+nb}{len}\PY{p}{(}\PY{n}{Datos5c}\PY{o}{.}\PY{n}{index}\PY{p}{)}\PY{o}{\PYZhy{}}\PY{l+m+mi}{1}\PY{p}{)}\PY{p}{:}
        \PY{k}{return} \PY{k+kc}{None}
    \PY{k}{return} \PY{n}{Datos5c}\PY{p}{[}\PY{n}{columna}\PY{p}{]}\PY{p}{[}\PY{n}{t}\PY{o}{+}\PY{l+m+mi}{1}\PY{p}{]}\PY{o}{\PYZhy{}}\PY{n}{Datos5c}\PY{p}{[}\PY{l+s+s2}{\PYZdq{}}\PY{l+s+s2}{\PYZdl{}A\PYZus{}t\PYZdl{}}\PY{l+s+s2}{\PYZdq{}}\PY{p}{]}\PY{p}{[}\PY{n}{t}\PY{o}{\PYZhy{}}\PY{l+m+mi}{1}\PY{p}{]}
    
\PY{n}{Datos5c}\PY{p}{[}\PY{l+s+s2}{\PYZdq{}}\PY{l+s+s2}{\PYZdl{}A\PYZus{}}\PY{l+s+s2}{\PYZob{}}\PY{l+s+s2}{t+1\PYZcb{}\PYZhy{}A\PYZus{}}\PY{l+s+s2}{\PYZob{}}\PY{l+s+s2}{t\PYZhy{}1\PYZcb{}\PYZdl{}}\PY{l+s+s2}{\PYZdq{}}\PY{p}{]} \PY{o}{=} \PY{n}{pd}\PY{o}{.}\PY{n}{DataFrame}\PY{p}{(}\PY{n+nb}{map}\PY{p}{(}\PY{k}{lambda} \PY{n}{t}\PY{p}{:} \PY{n}{diferencia\PYZus{}A}\PY{p}{(}\PY{n}{t}\PY{p}{,} \PY{l+s+s2}{\PYZdq{}}\PY{l+s+s2}{\PYZdl{}A\PYZus{}t\PYZdl{}}\PY{l+s+s2}{\PYZdq{}}\PY{p}{)}\PY{p}{,} \PY{n+nb}{list}\PY{p}{(}\PY{n}{Datos5c}\PY{o}{.}\PY{n}{index}\PY{p}{)}\PY{p}{)}\PY{p}{)}
\PY{c+c1}{\PYZsh{}Datos5c[\PYZsq{}\PYZdl{}A\PYZus{}\PYZob{}t+1\PYZcb{}\PYZhy{}A\PYZus{}\PYZob{}t\PYZhy{}1\PYZcb{}\PYZdl{}\PYZsq{}]}
\PY{n}{Datos5c}\PY{o}{.}\PY{n}{plot}\PY{p}{(}\PY{n}{x}\PY{o}{=}\PY{l+s+s2}{\PYZdq{}}\PY{l+s+s2}{\PYZdl{}}\PY{l+s+s2}{\PYZbs{}}\PY{l+s+s2}{sqrt}\PY{l+s+si}{\PYZob{}A\PYZus{}t\PYZcb{}}\PY{l+s+s2}{\PYZdl{}}\PY{l+s+s2}{\PYZdq{}}\PY{p}{,} \PY{n}{y}\PY{o}{=}\PY{l+s+s2}{\PYZdq{}}\PY{l+s+s2}{\PYZdl{}A\PYZus{}}\PY{l+s+s2}{\PYZob{}}\PY{l+s+s2}{t+1\PYZcb{}\PYZhy{}A\PYZus{}}\PY{l+s+s2}{\PYZob{}}\PY{l+s+s2}{t\PYZhy{}1\PYZcb{}\PYZdl{}}\PY{l+s+s2}{\PYZdq{}}\PY{p}{,} \PY{n}{marker}\PY{o}{=}\PY{l+s+s2}{\PYZdq{}}\PY{l+s+s2}{o}\PY{l+s+s2}{\PYZdq{}}\PY{p}{)}
\PY{c+c1}{\PYZsh{}print(Datos5c[[\PYZdq{}\PYZdl{}\PYZbs{}sqrt\PYZob{}A\PYZus{}t\PYZcb{}\PYZdl{}\PYZdq{},\PYZdq{}\PYZdl{}A\PYZus{}\PYZob{}t+1\PYZcb{}\PYZhy{}A\PYZus{}\PYZob{}t\PYZhy{}1\PYZcb{}\PYZdl{}\PYZdq{}]].to\PYZus{}latex(index=False, escape=False))}
\end{Verbatim}
\end{tcolorbox}

            \begin{tcolorbox}[breakable, size=fbox, boxrule=.5pt, pad at break*=1mm, opacityfill=0]
\prompt{Out}{outcolor}{54}{\boxspacing}
\begin{Verbatim}[commandchars=\\\{\}]
<AxesSubplot:xlabel='\$\textbackslash{}\textbackslash{}sqrt\{A\_t\}\$'>
\end{Verbatim}
\end{tcolorbox}
        
    \begin{center}
    \adjustimage{max size={0.9\linewidth}{0.9\paperheight}}{output_54_1.png}
    \end{center}
    { \hspace*{\fill} \\}
    
    \begin{tabular}{rr}
\toprule
 $\sqrt{A_t}$ &  $A_{t+1}-A_{t-1}$ \\
\midrule
        2.000 &                NaN \\
        2.828 &               20.0 \\
        4.899 &               38.0 \\
        6.782 &               60.0 \\
        9.165 &               80.0 \\
       11.225 &               92.0 \\
       13.266 &              122.0 \\
       15.748 &              150.0 \\
       18.055 &              172.0 \\
       20.494 &                NaN \\
\bottomrule
\end{tabular}

    Para encontrar \(K\) simplemente se toma la media del conjunto
\(\left\{\frac{A_{t+1}-A_{t-1}}{\sqrt{A_t}}:t=1,2,\ldots,8\right\}\).

    \begin{tcolorbox}[breakable, size=fbox, boxrule=1pt, pad at break*=1mm,colback=cellbackground, colframe=cellborder]
\prompt{In}{incolor}{55}{\boxspacing}
\begin{Verbatim}[commandchars=\\\{\}]
\PY{k+kn}{from} \PY{n+nn}{numpy} \PY{k+kn}{import} \PY{n}{median}\PY{p}{,}\PY{n}{mean}
\PY{n}{K} \PY{o}{=} \PY{n}{median}\PY{p}{(}\PY{n+nb}{list}\PY{p}{(}\PY{n+nb}{map}\PY{p}{(}\PY{k}{lambda} \PY{n}{x}\PY{p}{,}\PY{n}{y}\PY{p}{:} \PY{n}{y}\PY{o}{/}\PY{n}{x}\PY{p}{,}
           \PY{n}{Datos5c}\PY{p}{[}\PY{l+s+s2}{\PYZdq{}}\PY{l+s+s2}{\PYZdl{}}\PY{l+s+s2}{\PYZbs{}}\PY{l+s+s2}{sqrt}\PY{l+s+si}{\PYZob{}A\PYZus{}t\PYZcb{}}\PY{l+s+s2}{\PYZdl{}}\PY{l+s+s2}{\PYZdq{}}\PY{p}{]}\PY{p}{[}\PY{l+m+mi}{1}\PY{p}{:}\PY{o}{\PYZhy{}}\PY{l+m+mi}{1}\PY{p}{]}\PY{o}{.}\PY{n}{to\PYZus{}numpy}\PY{p}{(}\PY{p}{)}\PY{p}{,}\PY{n}{Datos5c}\PY{p}{[}\PY{l+s+s2}{\PYZdq{}}\PY{l+s+s2}{\PYZdl{}A\PYZus{}}\PY{l+s+s2}{\PYZob{}}\PY{l+s+s2}{t+1\PYZcb{}\PYZhy{}A\PYZus{}}\PY{l+s+s2}{\PYZob{}}\PY{l+s+s2}{t\PYZhy{}1\PYZcb{}\PYZdl{}}\PY{l+s+s2}{\PYZdq{}}\PY{p}{]}\PY{p}{[}\PY{l+m+mi}{1}\PY{p}{:}\PY{o}{\PYZhy{}}\PY{l+m+mi}{1}\PY{p}{]}\PY{o}{.}\PY{n}{to\PYZus{}numpy}\PY{p}{(}\PY{p}{)}\PY{p}{)} \PY{p}{)}\PY{p}{)}
\PY{n}{K}
\end{Verbatim}
\end{tcolorbox}

            \begin{tcolorbox}[breakable, size=fbox, boxrule=.5pt, pad at break*=1mm, opacityfill=0]
\prompt{Out}{outcolor}{55}{\boxspacing}
\begin{Verbatim}[commandchars=\\\{\}]
8.787616489366762
\end{Verbatim}
\end{tcolorbox}
        
    \begin{tcolorbox}[breakable, size=fbox, boxrule=1pt, pad at break*=1mm,colback=cellbackground, colframe=cellborder]
\prompt{In}{incolor}{56}{\boxspacing}
\begin{Verbatim}[commandchars=\\\{\}]
\PY{n}{Datos5c}\PY{p}{[}\PY{l+s+s2}{\PYZdq{}}\PY{l+s+s2}{\PYZdl{}K}\PY{l+s+s2}{\PYZbs{}}\PY{l+s+s2}{sqrt}\PY{l+s+si}{\PYZob{}A\PYZus{}t\PYZcb{}}\PY{l+s+s2}{\PYZdl{}}\PY{l+s+s2}{\PYZdq{}}\PY{p}{]} \PY{o}{=} \PY{n}{Datos5c}\PY{p}{[}\PY{l+s+s2}{\PYZdq{}}\PY{l+s+s2}{\PYZdl{}}\PY{l+s+s2}{\PYZbs{}}\PY{l+s+s2}{sqrt}\PY{l+s+si}{\PYZob{}A\PYZus{}t\PYZcb{}}\PY{l+s+s2}{\PYZdl{}}\PY{l+s+s2}{\PYZdq{}}\PY{p}{]}\PY{o}{.}\PY{n}{apply}\PY{p}{(}\PY{k}{lambda} \PY{n}{x}\PY{p}{:} \PY{n}{K}\PY{o}{*}\PY{n}{x}\PY{p}{)}
\PY{n}{Datos5c}\PY{p}{[}\PY{l+s+s2}{\PYZdq{}}\PY{l+s+s2}{\PYZdl{}}\PY{l+s+se}{\PYZbs{}\PYZbs{}}\PY{l+s+s2}{frac}\PY{l+s+s2}{\PYZob{}}\PY{l+s+s2}{K\PYZca{}2\PYZcb{}}\PY{l+s+si}{\PYZob{}16\PYZcb{}}\PY{l+s+s2}{t\PYZca{}2\PYZdl{}}\PY{l+s+s2}{\PYZdq{}}\PY{p}{]} \PY{o}{=} \PY{n}{Datos5c}\PY{p}{[}\PY{l+s+s2}{\PYZdq{}}\PY{l+s+s2}{\PYZdl{}t\PYZdl{}}\PY{l+s+s2}{\PYZdq{}}\PY{p}{]}\PY{o}{.}\PY{n}{apply}\PY{p}{(}\PY{k}{lambda} \PY{n}{t}\PY{p}{:} \PY{p}{(}\PY{n}{K}\PY{o}{*}\PY{o}{*}\PY{l+m+mi}{2}\PY{p}{)}\PY{o}{/}\PY{l+m+mi}{16}\PY{o}{*}\PY{n}{t}\PY{o}{*}\PY{o}{*}\PY{l+m+mi}{2}\PY{p}{)}
\PY{n}{Datos5c}
\PY{n}{display}\PY{p}{(}\PY{n}{Datos5c}\PY{o}{.}\PY{n}{plot}\PY{p}{(}\PY{n}{x}\PY{o}{=}\PY{l+s+s2}{\PYZdq{}}\PY{l+s+s2}{\PYZdl{}}\PY{l+s+s2}{\PYZbs{}}\PY{l+s+s2}{sqrt}\PY{l+s+si}{\PYZob{}A\PYZus{}t\PYZcb{}}\PY{l+s+s2}{\PYZdl{}}\PY{l+s+s2}{\PYZdq{}}\PY{p}{,}
                     \PY{n}{y}\PY{o}{=}\PY{p}{[}\PY{l+s+s2}{\PYZdq{}}\PY{l+s+s2}{\PYZdl{}A\PYZus{}}\PY{l+s+s2}{\PYZob{}}\PY{l+s+s2}{t+1\PYZcb{}\PYZhy{}A\PYZus{}}\PY{l+s+s2}{\PYZob{}}\PY{l+s+s2}{t\PYZhy{}1\PYZcb{}\PYZdl{}}\PY{l+s+s2}{\PYZdq{}}\PY{p}{,} \PY{l+s+s2}{\PYZdq{}}\PY{l+s+s2}{\PYZdl{}K}\PY{l+s+s2}{\PYZbs{}}\PY{l+s+s2}{sqrt}\PY{l+s+si}{\PYZob{}A\PYZus{}t\PYZcb{}}\PY{l+s+s2}{\PYZdl{}}\PY{l+s+s2}{\PYZdq{}}\PY{p}{]}\PY{p}{,} \PY{n}{marker}\PY{o}{=}\PY{l+s+s2}{\PYZdq{}}\PY{l+s+s2}{o}\PY{l+s+s2}{\PYZdq{}}\PY{p}{,} \PY{n}{color}\PY{o}{=}\PY{p}{[}\PY{l+s+s2}{\PYZdq{}}\PY{l+s+s2}{r}\PY{l+s+s2}{\PYZdq{}}\PY{p}{,}\PY{l+s+s2}{\PYZdq{}}\PY{l+s+s2}{b}\PY{l+s+s2}{\PYZdq{}}\PY{p}{]}\PY{p}{)}\PY{p}{)}
\PY{n}{display}\PY{p}{(}\PY{n}{Datos5c}\PY{o}{.}\PY{n}{plot}\PY{p}{(}\PY{n}{x}\PY{o}{=}\PY{l+s+s2}{\PYZdq{}}\PY{l+s+s2}{\PYZdl{}t\PYZdl{}}\PY{l+s+s2}{\PYZdq{}}\PY{p}{,}
                     \PY{n}{y}\PY{o}{=}\PY{p}{[}\PY{l+s+s2}{\PYZdq{}}\PY{l+s+s2}{\PYZdl{}A\PYZus{}t\PYZdl{}}\PY{l+s+s2}{\PYZdq{}}\PY{p}{,} \PY{l+s+s2}{\PYZdq{}}\PY{l+s+s2}{\PYZdl{}}\PY{l+s+se}{\PYZbs{}\PYZbs{}}\PY{l+s+s2}{frac}\PY{l+s+s2}{\PYZob{}}\PY{l+s+s2}{K\PYZca{}2\PYZcb{}}\PY{l+s+si}{\PYZob{}16\PYZcb{}}\PY{l+s+s2}{t\PYZca{}2\PYZdl{}}\PY{l+s+s2}{\PYZdq{}}\PY{p}{]}\PY{p}{,} \PY{n}{marker}\PY{o}{=}\PY{l+s+s2}{\PYZdq{}}\PY{l+s+s2}{o}\PY{l+s+s2}{\PYZdq{}}\PY{p}{,} \PY{n}{color}\PY{o}{=}\PY{p}{[}\PY{l+s+s2}{\PYZdq{}}\PY{l+s+s2}{r}\PY{l+s+s2}{\PYZdq{}}\PY{p}{,}\PY{l+s+s2}{\PYZdq{}}\PY{l+s+s2}{b}\PY{l+s+s2}{\PYZdq{}}\PY{p}{]}\PY{p}{)}\PY{p}{)}
\PY{c+c1}{\PYZsh{}print(Datos5c[[\PYZdq{}\PYZdl{}t\PYZdl{}\PYZdq{}, \PYZdq{}\PYZdl{}A\PYZus{}t\PYZdl{}\PYZdq{}, \PYZdq{}\PYZdl{}\PYZbs{}\PYZbs{}frac\PYZob{}K\PYZca{}2\PYZcb{}\PYZob{}16\PYZcb{}t\PYZca{}2\PYZdl{}\PYZdq{}]].to\PYZus{}latex(index=False, escape=False))}
\end{Verbatim}
\end{tcolorbox}

    
    \begin{Verbatim}[commandchars=\\\{\}]
<AxesSubplot:xlabel='\$\textbackslash{}\textbackslash{}sqrt\{A\_t\}\$'>
    \end{Verbatim}

    
    
    \begin{Verbatim}[commandchars=\\\{\}]
<AxesSubplot:xlabel='\$t\$'>
    \end{Verbatim}

    
    \begin{center}
    \adjustimage{max size={0.9\linewidth}{0.9\paperheight}}{output_58_2.png}
    \end{center}
    { \hspace*{\fill} \\}
    
    \begin{center}
    \adjustimage{max size={0.9\linewidth}{0.9\paperheight}}{output_58_3.png}
    \end{center}
    { \hspace*{\fill} \\}
    
    \begin{tabular}{rrr}
\toprule
 $t$ &  $A_t$ &  $\frac{K^2}{16}t^2$ \\
\midrule
   0 &      4 &                0.000 \\
   1 &      8 &                4.826 \\
   2 &     24 &               19.306 \\
   3 &     46 &               43.437 \\
   4 &     84 &               77.222 \\
   5 &    126 &              120.660 \\
   6 &    176 &              173.750 \\
   7 &    248 &              236.493 \\
   8 &    326 &              308.889 \\
   9 &    420 &              390.937 \\
\bottomrule
\end{tabular}

    En rojo los valores observados, y en azul la solución del modelo
propuesto. Se puede ver que meramente tomar \(\frac{K^2}{16}t^2\) como
solución subestima todos los datos del experimento. Esto es de
esperarse: La ecuación de diferencias se resuelve tomando \(K\) como la
media que mencionamos antes, quiere decir que no se estima ni se
subestima tanto en cuánto el aumento. Pero la solución empieza en \(0\)
mientras que sabemos que el primer dato es con un valor de \(4\) por lo
que casi todos los datos se subestiman por \(4\) al haber tomado una
tasa de cambio media.

Si a la solución anterior le añadimos \(4\) y le llamamos \(\hat A_t\),
se mejora la cercanía a los datos incluso en las diferencias relativas.
Para esta solución usamos verde en la gráfica.

    \begin{tcolorbox}[breakable, size=fbox, boxrule=1pt, pad at break*=1mm,colback=cellbackground, colframe=cellborder]
\prompt{In}{incolor}{57}{\boxspacing}
\begin{Verbatim}[commandchars=\\\{\}]
\PY{n}{Datos5c}\PY{p}{[}\PY{l+s+s2}{\PYZdq{}}\PY{l+s+s2}{\PYZdl{}}\PY{l+s+se}{\PYZbs{}\PYZbs{}}\PY{l+s+s2}{frac}\PY{l+s+s2}{\PYZob{}}\PY{l+s+s2}{K\PYZca{}2\PYZcb{}}\PY{l+s+si}{\PYZob{}16\PYZcb{}}\PY{l+s+s2}{t\PYZca{}2+4\PYZdl{}}\PY{l+s+s2}{\PYZdq{}}\PY{p}{]} \PY{o}{=} \PY{n}{Datos5c}\PY{p}{[}\PY{l+s+s2}{\PYZdq{}}\PY{l+s+s2}{\PYZdl{}}\PY{l+s+se}{\PYZbs{}\PYZbs{}}\PY{l+s+s2}{frac}\PY{l+s+s2}{\PYZob{}}\PY{l+s+s2}{K\PYZca{}2\PYZcb{}}\PY{l+s+si}{\PYZob{}16\PYZcb{}}\PY{l+s+s2}{t\PYZca{}2\PYZdl{}}\PY{l+s+s2}{\PYZdq{}}\PY{p}{]}\PY{o}{.}\PY{n}{apply}\PY{p}{(}\PY{k}{lambda} \PY{n}{x}\PY{p}{:} \PY{n}{x}\PY{o}{+}\PY{l+m+mi}{4}\PY{p}{)}
\PY{n}{Datos5c}\PY{p}{[}\PY{l+s+s2}{\PYZdq{}}\PY{l+s+s2}{\PYZdl{}}\PY{l+s+se}{\PYZbs{}\PYZbs{}}\PY{l+s+s2}{hat A\PYZus{}}\PY{l+s+s2}{\PYZob{}}\PY{l+s+s2}{t+1\PYZcb{}\PYZhy{}}\PY{l+s+se}{\PYZbs{}\PYZbs{}}\PY{l+s+s2}{hat A\PYZus{}}\PY{l+s+s2}{\PYZob{}}\PY{l+s+s2}{t\PYZhy{}1\PYZcb{}\PYZdl{}}\PY{l+s+s2}{\PYZdq{}}\PY{p}{]} \PY{o}{=} \PY{n}{pd}\PY{o}{.}\PY{n}{DataFrame}\PY{p}{(}\PY{n+nb}{map}\PY{p}{(}
    \PY{k}{lambda} \PY{n}{t}\PY{p}{:} \PY{n}{diferencia\PYZus{}A}\PY{p}{(}\PY{n}{t}\PY{p}{,} \PY{l+s+s2}{\PYZdq{}}\PY{l+s+s2}{\PYZdl{}}\PY{l+s+se}{\PYZbs{}\PYZbs{}}\PY{l+s+s2}{frac}\PY{l+s+s2}{\PYZob{}}\PY{l+s+s2}{K\PYZca{}2\PYZcb{}}\PY{l+s+si}{\PYZob{}16\PYZcb{}}\PY{l+s+s2}{t\PYZca{}2+4\PYZdl{}}\PY{l+s+s2}{\PYZdq{}}\PY{p}{)}\PY{p}{,}
    \PY{n+nb}{list}\PY{p}{(}\PY{n}{Datos5c}\PY{o}{.}\PY{n}{index}\PY{p}{)}\PY{p}{)}\PY{p}{)}

\PY{n}{display}\PY{p}{(}\PY{n}{Datos5c}\PY{o}{.}\PY{n}{plot}\PY{p}{(}\PY{n}{x}\PY{o}{=}\PY{l+s+s2}{\PYZdq{}}\PY{l+s+s2}{\PYZdl{}}\PY{l+s+s2}{\PYZbs{}}\PY{l+s+s2}{sqrt}\PY{l+s+si}{\PYZob{}A\PYZus{}t\PYZcb{}}\PY{l+s+s2}{\PYZdl{}}\PY{l+s+s2}{\PYZdq{}}\PY{p}{,}
                     \PY{n}{y}\PY{o}{=}\PY{p}{[}\PY{l+s+s2}{\PYZdq{}}\PY{l+s+s2}{\PYZdl{}A\PYZus{}}\PY{l+s+s2}{\PYZob{}}\PY{l+s+s2}{t+1\PYZcb{}\PYZhy{}A\PYZus{}}\PY{l+s+s2}{\PYZob{}}\PY{l+s+s2}{t\PYZhy{}1\PYZcb{}\PYZdl{}}\PY{l+s+s2}{\PYZdq{}}\PY{p}{,} \PY{l+s+s2}{\PYZdq{}}\PY{l+s+s2}{\PYZdl{}}\PY{l+s+se}{\PYZbs{}\PYZbs{}}\PY{l+s+s2}{hat A\PYZus{}}\PY{l+s+s2}{\PYZob{}}\PY{l+s+s2}{t+1\PYZcb{}\PYZhy{}}\PY{l+s+se}{\PYZbs{}\PYZbs{}}\PY{l+s+s2}{hat A\PYZus{}}\PY{l+s+s2}{\PYZob{}}\PY{l+s+s2}{t\PYZhy{}1\PYZcb{}\PYZdl{}}\PY{l+s+s2}{\PYZdq{}}\PY{p}{]}\PY{p}{,} \PY{n}{marker}\PY{o}{=}\PY{l+s+s2}{\PYZdq{}}\PY{l+s+s2}{o}\PY{l+s+s2}{\PYZdq{}}\PY{p}{,} \PY{n}{color}\PY{o}{=}\PY{p}{[}\PY{l+s+s2}{\PYZdq{}}\PY{l+s+s2}{r}\PY{l+s+s2}{\PYZdq{}}\PY{p}{,}\PY{l+s+s2}{\PYZdq{}}\PY{l+s+s2}{g}\PY{l+s+s2}{\PYZdq{}}\PY{p}{]}\PY{p}{)}\PY{p}{)}
\PY{n}{display}\PY{p}{(}\PY{n}{Datos5c}\PY{o}{.}\PY{n}{plot}\PY{p}{(}\PY{n}{x}\PY{o}{=}\PY{l+s+s2}{\PYZdq{}}\PY{l+s+s2}{\PYZdl{}t\PYZdl{}}\PY{l+s+s2}{\PYZdq{}}\PY{p}{,}
                     \PY{n}{y}\PY{o}{=}\PY{p}{[}\PY{l+s+s2}{\PYZdq{}}\PY{l+s+s2}{\PYZdl{}A\PYZus{}t\PYZdl{}}\PY{l+s+s2}{\PYZdq{}}\PY{p}{,} \PY{l+s+s2}{\PYZdq{}}\PY{l+s+s2}{\PYZdl{}}\PY{l+s+se}{\PYZbs{}\PYZbs{}}\PY{l+s+s2}{frac}\PY{l+s+s2}{\PYZob{}}\PY{l+s+s2}{K\PYZca{}2\PYZcb{}}\PY{l+s+si}{\PYZob{}16\PYZcb{}}\PY{l+s+s2}{t\PYZca{}2+4\PYZdl{}}\PY{l+s+s2}{\PYZdq{}}\PY{p}{]}\PY{p}{,} \PY{n}{marker}\PY{o}{=}\PY{l+s+s2}{\PYZdq{}}\PY{l+s+s2}{o}\PY{l+s+s2}{\PYZdq{}}\PY{p}{,} \PY{n}{color}\PY{o}{=}\PY{p}{[}\PY{l+s+s2}{\PYZdq{}}\PY{l+s+s2}{r}\PY{l+s+s2}{\PYZdq{}}\PY{p}{,}\PY{l+s+s2}{\PYZdq{}}\PY{l+s+s2}{g}\PY{l+s+s2}{\PYZdq{}}\PY{p}{]}\PY{p}{)}\PY{p}{)}
\PY{k}{if} \PY{k+kc}{False}\PY{p}{:}
    \PY{n+nb}{print}\PY{p}{(}\PY{n}{Datos5c}\PY{p}{[}\PY{p}{[}\PY{l+s+s2}{\PYZdq{}}\PY{l+s+s2}{\PYZdl{}t\PYZdl{}}\PY{l+s+s2}{\PYZdq{}}\PY{p}{,} \PY{l+s+s2}{\PYZdq{}}\PY{l+s+s2}{\PYZdl{}A\PYZus{}t\PYZdl{}}\PY{l+s+s2}{\PYZdq{}}\PY{p}{,} \PY{l+s+s2}{\PYZdq{}}\PY{l+s+s2}{\PYZdl{}}\PY{l+s+se}{\PYZbs{}\PYZbs{}}\PY{l+s+s2}{frac}\PY{l+s+s2}{\PYZob{}}\PY{l+s+s2}{K\PYZca{}2\PYZcb{}}\PY{l+s+si}{\PYZob{}16\PYZcb{}}\PY{l+s+s2}{t\PYZca{}2+4\PYZdl{}}\PY{l+s+s2}{\PYZdq{}}\PY{p}{,}
               \PY{l+s+s2}{\PYZdq{}}\PY{l+s+s2}{\PYZdl{}A\PYZus{}}\PY{l+s+s2}{\PYZob{}}\PY{l+s+s2}{t+1\PYZcb{}\PYZhy{}A\PYZus{}}\PY{l+s+s2}{\PYZob{}}\PY{l+s+s2}{t\PYZhy{}1\PYZcb{}\PYZdl{}}\PY{l+s+s2}{\PYZdq{}}\PY{p}{,} \PY{l+s+s2}{\PYZdq{}}\PY{l+s+s2}{\PYZdl{}}\PY{l+s+se}{\PYZbs{}\PYZbs{}}\PY{l+s+s2}{hat A\PYZus{}}\PY{l+s+s2}{\PYZob{}}\PY{l+s+s2}{t+1\PYZcb{}\PYZhy{}}\PY{l+s+se}{\PYZbs{}\PYZbs{}}\PY{l+s+s2}{hat A\PYZus{}}\PY{l+s+s2}{\PYZob{}}\PY{l+s+s2}{t\PYZhy{}1\PYZcb{}\PYZdl{}}\PY{l+s+s2}{\PYZdq{}}\PY{p}{]}\PY{p}{]}\PY{o}{.}\PY{n}{to\PYZus{}latex}\PY{p}{(}\PY{n}{index}\PY{o}{=}\PY{k+kc}{False}\PY{p}{,} \PY{n}{escape}\PY{o}{=}\PY{k+kc}{False}\PY{p}{)}\PY{p}{)}
\end{Verbatim}
\end{tcolorbox}

    
    \begin{Verbatim}[commandchars=\\\{\}]
<AxesSubplot:xlabel='\$\textbackslash{}\textbackslash{}sqrt\{A\_t\}\$'>
    \end{Verbatim}

    
    
    \begin{Verbatim}[commandchars=\\\{\}]
<AxesSubplot:xlabel='\$t\$'>
    \end{Verbatim}

    
    \begin{center}
    \adjustimage{max size={0.9\linewidth}{0.9\paperheight}}{output_61_2.png}
    \end{center}
    { \hspace*{\fill} \\}
    
    \begin{center}
    \adjustimage{max size={0.9\linewidth}{0.9\paperheight}}{output_61_3.png}
    \end{center}
    { \hspace*{\fill} \\}
    
    \begin{tabular}{rrrrr}
\toprule
 $t$ &  $A_t$ &  $\frac{K^2}{16}t^2+4$ &  $A_{t+1}-A_{t-1}$ &  $\hat A_{t+1}-\hat A_{t-1}$ \\
\midrule
   0 &      4 &                  4.000 &                NaN &                          NaN \\
   1 &      8 &                  8.826 &               20.0 &                       19.306 \\
   2 &     24 &                 23.306 &               38.0 &                       39.437 \\
   3 &     46 &                 47.437 &               60.0 &                       57.222 \\
   4 &     84 &                 81.222 &               80.0 &                       78.660 \\
   5 &    126 &                124.660 &               92.0 &                       93.750 \\
   6 &    176 &                177.750 &              122.0 &                      114.493 \\
   7 &    248 &                240.493 &              150.0 &                      136.889 \\
   8 &    326 &                312.889 &              172.0 &                      146.937 \\
   9 &    420 &                394.937 &                NaN &                          NaN \\
\bottomrule
\end{tabular}

    \hypertarget{ejercicio-6-eliminaciuxf3n-de-penicilina}{%
\section*{Ejercicio 6: Eliminación de
Penicilina}\label{ejercicio-6-eliminaciuxf3n-de-penicilina}}

Se ingiere una pastilla de 500 miligramos de penicilina e ingresa
inmediatamente al intestino. Cada cinco minutos después de la ingestión
de la píldora. - El 10 \% de la penicilina en el intestino al comienzo
del período se absorbe en el plasma. - El riñón elimina el 15 \% de la
penicilina en el plasma al comienzo del período.

Sea \(I_t\) la cantidad de penicilina en el intestino y \(S_t\) la
cantidad de penicilina en el plasma al final del período de cinco
minutos después de la ingestión de la píldora, ambos en miligramos.

    \hypertarget{condiciones-iniciales}{%
\subsection*{Condiciones iniciales}\label{condiciones-iniciales}}

\[I_0 = 500,\] \[S_0 = 0.\]

    \hypertarget{cambio-de-penicilina-por-periodo-de-tiempo}{%
\subsection*{Cambio de penicilina por periodo de
tiempo}\label{cambio-de-penicilina-por-periodo-de-tiempo}}

Con \(0\) penicilina ingerida entre periodos, la penicilina agregada al
intestino es \(0\). Pero se elimina \(10\%\) por el plasma.
\[I_{t+1}-I_t = -0.10 I_t.\]

En cambio en el plasma, se agregan \(10\%\) del intestino, y el riñon
elimina \(15\%\) de la plasma. \[S_{t+1}-S_t = 0.10 I_t -0.15 S_t.\]

    \hypertarget{ejercicio-7.}{%
\section*{Ejercicio 7.}\label{ejercicio-7.}}

Se remplaza la notación del sistema de a) y de b) por la de los
siguientes respectivos sistemas equivalentes. \[A_0 = 0,\]
\[A_{t+1} = 100+0.2 A_t.\] \[B_0 = 0,\] \[B_{t+1} = 10 -0.1 B_t.\]

    \hypertarget{dar-los-tuxe9rminos-01234-para-cada-sistema}{%
\subsection*{\texorpdfstring{1. Dar los términos \(0,1,2,3,4\) para cada
sistema}{1. Dar los términos 0,1,2,3,4 para cada sistema}}\label{dar-los-tuxe9rminos-01234-para-cada-sistema}}

    \begin{tcolorbox}[breakable, size=fbox, boxrule=1pt, pad at break*=1mm,colback=cellbackground, colframe=cellborder]
\prompt{In}{incolor}{58}{\boxspacing}
\begin{Verbatim}[commandchars=\\\{\}]
\PY{n}{A} \PY{o}{=} \PY{k}{lambda} \PY{n}{a}\PY{p}{:} \PY{l+m+mi}{100}\PY{o}{+}\PY{l+m+mf}{0.2}\PY{o}{*}\PY{n}{a}
\PY{n}{B} \PY{o}{=} \PY{k}{lambda} \PY{n}{b}\PY{p}{:} \PY{l+m+mi}{10}\PY{o}{\PYZhy{}}\PY{l+m+mf}{0.1}\PY{o}{*}\PY{n}{b}

\PY{k}{def} \PY{n+nf}{recur}\PY{p}{(}\PY{n}{f}\PY{p}{,} \PY{n}{f\PYZus{}0}\PY{p}{,} \PY{n}{t}\PY{p}{)}\PY{p}{:}
    \PY{n}{seq} \PY{o}{=} \PY{n+nb}{list}\PY{p}{(}\PY{p}{)}
    \PY{k}{for} \PY{n}{t} \PY{o+ow}{in} \PY{n+nb}{range}\PY{p}{(}\PY{n}{t}\PY{p}{)}\PY{p}{:}
        \PY{n}{f\PYZus{}0} \PY{o}{=} \PY{n}{f}\PY{p}{(}\PY{n}{f\PYZus{}0}\PY{p}{)}
        \PY{n}{seq}\PY{o}{.}\PY{n}{append}\PY{p}{(}\PY{n}{f\PYZus{}0}\PY{p}{)}
    \PY{k}{return} \PY{n}{seq}

\PY{n}{Datos7} \PY{o}{=} \PY{n}{pd}\PY{o}{.}\PY{n}{DataFrame}\PY{p}{(}\PY{n+nb}{map}\PY{p}{(}\PY{k}{lambda} \PY{n}{t}\PY{p}{,}\PY{n}{a}\PY{p}{,}\PY{n}{b}\PY{p}{:} \PY{p}{(}\PY{n}{t}\PY{p}{,}\PY{n}{a}\PY{p}{,}\PY{n}{b}\PY{p}{)}\PY{p}{,}
                 \PY{n+nb}{list}\PY{p}{(}\PY{n+nb}{range}\PY{p}{(}\PY{l+m+mi}{5}\PY{p}{)}\PY{p}{)}\PY{p}{,}\PY{n}{recur}\PY{p}{(}\PY{n}{A}\PY{p}{,} \PY{l+m+mi}{0}\PY{p}{,} \PY{l+m+mi}{5}\PY{p}{)}\PY{p}{,}\PY{n}{recur}\PY{p}{(}\PY{n}{B}\PY{p}{,} \PY{l+m+mi}{0}\PY{p}{,} \PY{l+m+mi}{5}\PY{p}{)}\PY{p}{)}\PY{p}{,}
             \PY{n}{columns} \PY{o}{=} \PY{p}{[}\PY{l+s+s2}{\PYZdq{}}\PY{l+s+s2}{\PYZdl{}t\PYZdl{}}\PY{l+s+s2}{\PYZdq{}}\PY{p}{,} \PY{l+s+s2}{\PYZdq{}}\PY{l+s+s2}{\PYZdl{}A\PYZus{}t\PYZdl{}}\PY{l+s+s2}{\PYZdq{}}\PY{p}{,} \PY{l+s+s2}{\PYZdq{}}\PY{l+s+s2}{\PYZdl{}B\PYZus{}t\PYZdl{}}\PY{l+s+s2}{\PYZdq{}}\PY{p}{]}\PY{p}{)}
\PY{c+c1}{\PYZsh{}print(Datos7.to\PYZus{}latex(index=False, escape=False))}
\end{Verbatim}
\end{tcolorbox}

    \begin{tabular}{rrr}
\toprule
 $t$ &  $A_t$ &  $B_t$ \\
\midrule
   0 & 100.00 & 10.000 \\
   1 & 120.00 &  9.000 \\
   2 & 124.00 &  9.100 \\
   3 & 124.80 &  9.090 \\
   4 & 124.96 &  9.091 \\
\bottomrule
\end{tabular}

    \hypertarget{encontrar-el-valor-de-equilibrio-para-cada-sistema}{%
\subsection*{2. Encontrar el valor de equilibrio para cada
sistema}\label{encontrar-el-valor-de-equilibrio-para-cada-sistema}}

En un sistema

\[W_{t+1} = w + k W_t,\]

el punto de equilibrio \(W\) es tal que \(W = W_{t+1} = W_t\) es decir

\[W = w + kW.\]

Cuando \(k\neq 1\),

\[W = \frac{w}{1-k}.\]

(Usualmente se usa \(K\) en lugar de \(k\).)

    Sustituyendo los valores del sistema de a) y b) sus puntos de equilibrio
son respectivamente

\[A = \frac{100}{0.8} = 125,\]

\[B = \frac{10}{1.1} = 9.\overline{09}.\]

    \hypertarget{escribir-ecuaciuxf3n-soluciuxf3n-para-cada-sistema}{%
\subsection*{3. Escribir ecuación solución para cada
sistema}\label{escribir-ecuaciuxf3n-soluciuxf3n-para-cada-sistema}}

Al sistema

\[W_{t+1} = w + k W_t,\]

se le resta la ecuación

\(W = w + kW\) para obtener \[W_{t+1}-W = k W_t-kW.\]

Se denota a \(W_t-W\) como \(D_t\) y la ecuación anterior se reduce a
una cuya solución ya conocemos.

\[D_{t+1} = k D_t,\]

\[D_{t+1}-D_t = (1-k) D_t.\]

Con solución

\[D_t = k^t D_0.\]

O escrito respecto de \(W_t\)

\[W_t = W+ k^t (W_0-W).\]

    Sustituyendo los valores para el sistema a) y b) tenemos respectivamente
las ecuaciones de solución

\[A_t = 125- 0.2^t \cdot 125,\]

\[B_t = 9.\overline{09}- (-0.1)^t \cdot9.\overline{09}.\]

    \hypertarget{calcular-a_100-y-b_100}{%
\subsection*{\texorpdfstring{4. Calcular \(A_{100}\) y
\(B_{100}\)}{4. Calcular A\_\{100\} y B\_\{100\}}}\label{calcular-a_100-y-b_100}}

    \[A_{100} =125.0,\] \[B_{100} =9.090909090909092.\]

    \hypertarget{tiempo-de-reducciuxf3n-a-la-mitad-de-los-sistemas}{%
\subsection*{5. Tiempo de reducción a la mitad de los
sistemas}\label{tiempo-de-reducciuxf3n-a-la-mitad-de-los-sistemas}}

    Si \(t\) es tal que \(W_{t+t_0}-0.5 W_{t_0}=0\), \(t\) es el tiempo de
reducción a la mitad para el sistema de \(W_t\) desde el tiempo \(t_0\).
La ecuación

\[W_{t+t_0}-0.5 W_{t_0}=0\]

se puede reescribir con la solución del sistema como

\[W+ k^{t+t_0} (W_0-W) -0.5\left[W+ k^{t_0} (W_0-W)\right] = 0,\]
\[0.5W+(k^{t+t_0}-0.5k^{t_0}) (W_0-W)= 0,\]

\[0.5W+k^{t_0}(k^{t}-0.5) (W_0-W)= 0.\]

De donde si \(W_0\neq W\) y \(k> 0\)

\[2k^{t} = -\frac{W}{k^{t_0}(W_0-W)}+1,\]

    \[\log(2)+t \log(k) = \log\left[-\frac{W}{k^{t_0}(W_0-W)}+1\right],\]

    \[t= \frac{\log\left[-\frac{W}{k^{t_0}(W_0-W)}+1\right]-\log{2}}{\log{k}}.\]

    Si alguno de los logaritmos no está definido entonces no hay tiempo de
reducción a la mitad para el sistema desde el tiempo \(t_0\). Solo está
definido \(t\) cuando

\[1> \frac{W}{k^{t_0}(W_0-W)},\] \[k^{t_0}(W_0-W)> W,\]
\[W_0> \left(k^{-t_0}+1\right)W.\]

    Para el sistema a) no hay tiempo de reducción a la mitad porque
\(A_0=0\) pero el lado derecho de la condición anterior es positivo pues
\(k=0.2\):

\[0> \left(0.2^{-t_0}+1\right)125\]

siempre es falso.

Para el sistema b) falla la condición desde antes porque \(k<0\).

    \hypertarget{ejercicio-8.}{%
\section*{Ejercicio 8.}\label{ejercicio-8.}}

El modelo original tenía una ecuación dinámica para la cantidad de
químicos \(W_t\) en el lago en kg, en el día \(t\):
\[W_{t+1}-W_t = 100-5\times 10^{-4} W_t.\] Donde \(5\times 10^{-4}\) era
por la proporcionalidad de la salida de químicos con el flujo de agua a
través del lago y la concentración de químicos de ese día. Mantenemos
esta constante para la salida de químicos y se siguen virtiendo 100 kg
de químicos al lago. Pero ahora imitando al modelo de la eliminación de
penicilina, separamos al lago en una proximidad de la fábrica, y ``el
resto del lago'' donde la concentración de químicos se vuelve uniforme.

Suponemos que solo de entre los químicos en el resto del lago es de
dónde los químicos salen del lago. Sean \(P_t\) los químicos diluidos en
la proximidad de la fábrica, y \(R_t\) los químicos en el resto del lago
al día; ambos en kilogramos. Entonces salen \(5\times 10^{-4} R_t\) kg
de químicos del lago al día.

Es supuesto del ejercicio que cada 10 días 100kg de químico se mezclan
uniformemente en la totalidad del lago. En nuestro modelo esto
corresponde a que salen 10kg de la proximidad de la fabrica hacía el
resto del lago al día.

Entonces en un día, en la proximidad del lago entran 100kg de la
fabrica, y salen 10kg a diluirse al resto del lago.
\[P_{t+1}-P_t = -10+100 = 90.\]

En el resto del lago en un día, se añaden 10kg de la proximidad de la
fabrica que se diluyen uniformemente; y así 100kg en 10 días son los que
se diluyen en el lago uniformemente. Pero también sale una proporción de
los químicos presentes ese día por el flujo del agua, como hemos dicho
\(5\times 10^{-4} R_t\). \[R_{t+1}-R_t = 10-5\times 10^{-4} R_t.\]

    Estas ecuaciones son independientes una de la otra: Cada día 90kg más se
acumulan en una proximidad de la fábrica mientras que el resto del lago
está siendo vertido con solo 10kg de químicos al día, expulsando la
misma cantidad que en el modelo original.

    Si se ignora el requísito de que solo 100kg cada 10 días se diluyan en
el lago, podemos hacer proporcional la cantidad de químicos que se
diluye al resto del lago con la cantidad de químicos en la proximidad
del lago. Digamos que \(10\%\) del contenido químico en la proximidad
del lago es diluido en el resto del lago cada día. Entonces el sistema
sería

\[P_{t+1}-P_t = 100-0.1P_t,\]
\[R_{t+1}-R_t = 0.1P_t -5\times 10^{-4} R_t.\]

    \hypertarget{ejercicio-9.}{%
\section*{Ejercicio 9.}\label{ejercicio-9.}}

Sea \(P_t\) es la población de codornices en el año \(t\). La
descripción del fenómeno nos dice:

\[P_{t+1} = -1000+1.2P_t.\]

El punto de equilibrio \(E\) si existe cumple que \(E=P_{t+1}=P_{t}\) o

\[E = -1000+1.2E,\] \[E = \frac{1000}{0.2} = 5000.\]

Como resolvimos en el ejercicio 7.3, la solución al sistema es entonces

\[P_t = 5000+1.2^t\cdot(P_0-5000).\]

    \begin{enumerate}
\def\labelenumi{\alph{enumi})}
\item
  Si \(P_0=5000\) la población se mantiene constante, en 5 años se tiene
  la misma cantidad de población.
\item
  Si \(P_0=6000\), en 5 años se tiene
  \(P_5= 5000+1.2^5\cdot 1000=7488.32\) de población o \(7488\)
  redondeando.
\item
  Si \(P_0=4000\), disminuye la población; en 5 años hay
  \(P_5=5000-1.2^t\cdot 1000=2511\) aves (redondeando).
\end{enumerate}


    % Add a bibliography block to the postdoc
    
    
    
\end{document}
