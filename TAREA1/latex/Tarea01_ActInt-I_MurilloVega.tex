\documentclass[11pt]{article}

    \usepackage[breakable]{tcolorbox}
    \usepackage{parskip} % Stop auto-indenting (to mimic markdown behaviour)
    \usepackage[utf8]{inputenc}
    
    \usepackage{iftex}
    \ifPDFTeX
    	\usepackage[T1]{fontenc}
    	\usepackage{mathpazo}
    \else
    	\usepackage{fontspec}
    \fi

    % Basic figure setup, for now with no caption control since it's done
    % automatically by Pandoc (which extracts ![](path) syntax from Markdown).
    \usepackage{graphicx}
    % Maintain compatibility with old templates. Remove in nbconvert 6.0
    \let\Oldincludegraphics\includegraphics
    % Ensure that by default, figures have no caption (until we provide a
    % proper Figure object with a Caption API and a way to capture that
    % in the conversion process - todo).
    \usepackage{caption}
    \DeclareCaptionFormat{nocaption}{}
    \captionsetup{format=nocaption,aboveskip=0pt,belowskip=0pt}

    \usepackage{float}
    \floatplacement{figure}{H} % forces figures to be placed at the correct location
    \usepackage{xcolor} % Allow colors to be defined
    \usepackage{enumerate} % Needed for markdown enumerations to work
    \usepackage{geometry} % Used to adjust the document margins
    \usepackage{amsmath} % Equations
    \usepackage{amssymb} % Equations
    \usepackage{textcomp} % defines textquotesingle
    % Hack from http://tex.stackexchange.com/a/47451/13684:
    \AtBeginDocument{%
        \def\PYZsq{\textquotesingle}% Upright quotes in Pygmentized code
    }
    \usepackage{upquote} % Upright quotes for verbatim code
    \usepackage{eurosym} % defines \euro
    \usepackage[mathletters]{ucs} % Extended unicode (utf-8) support
    \usepackage{fancyvrb} % verbatim replacement that allows latex
    \usepackage{grffile} % extends the file name processing of package graphics 
                         % to support a larger range
    \makeatletter % fix for old versions of grffile with XeLaTeX
    \@ifpackagelater{grffile}{2019/11/01}
    {
      % Do nothing on new versions
    }
    {
      \def\Gread@@xetex#1{%
        \IfFileExists{"\Gin@base".bb}%
        {\Gread@eps{\Gin@base.bb}}%
        {\Gread@@xetex@aux#1}%
      }
    }
    \makeatother
    \usepackage[Export]{adjustbox} % Used to constrain images to a maximum size
    \adjustboxset{max size={0.9\linewidth}{0.9\paperheight}}

    % The hyperref package gives us a pdf with properly built
    % internal navigation ('pdf bookmarks' for the table of contents,
    % internal cross-reference links, web links for URLs, etc.)
    \usepackage{hyperref}
    % The default LaTeX title has an obnoxious amount of whitespace. By default,
    % titling removes some of it. It also provides customization options.
    \usepackage{titling}
    \usepackage{longtable} % longtable support required by pandoc >1.10
    \usepackage{booktabs}  % table support for pandoc > 1.12.2
    \usepackage[inline]{enumitem} % IRkernel/repr support (it uses the enumerate* environment)
    \usepackage[normalem]{ulem} % ulem is needed to support strikethroughs (\sout)
                                % normalem makes italics be italics, not underlines
    \usepackage{mathrsfs}
    

    
    % Colors for the hyperref package
    \definecolor{urlcolor}{rgb}{0,.145,.698}
    \definecolor{linkcolor}{rgb}{.71,0.21,0.01}
    \definecolor{citecolor}{rgb}{.12,.54,.11}

    % ANSI colors
    \definecolor{ansi-black}{HTML}{3E424D}
    \definecolor{ansi-black-intense}{HTML}{282C36}
    \definecolor{ansi-red}{HTML}{E75C58}
    \definecolor{ansi-red-intense}{HTML}{B22B31}
    \definecolor{ansi-green}{HTML}{00A250}
    \definecolor{ansi-green-intense}{HTML}{007427}
    \definecolor{ansi-yellow}{HTML}{DDB62B}
    \definecolor{ansi-yellow-intense}{HTML}{B27D12}
    \definecolor{ansi-blue}{HTML}{208FFB}
    \definecolor{ansi-blue-intense}{HTML}{0065CA}
    \definecolor{ansi-magenta}{HTML}{D160C4}
    \definecolor{ansi-magenta-intense}{HTML}{A03196}
    \definecolor{ansi-cyan}{HTML}{60C6C8}
    \definecolor{ansi-cyan-intense}{HTML}{258F8F}
    \definecolor{ansi-white}{HTML}{C5C1B4}
    \definecolor{ansi-white-intense}{HTML}{A1A6B2}
    \definecolor{ansi-default-inverse-fg}{HTML}{FFFFFF}
    \definecolor{ansi-default-inverse-bg}{HTML}{000000}

    % common color for the border for error outputs.
    \definecolor{outerrorbackground}{HTML}{FFDFDF}

    % commands and environments needed by pandoc snippets
    % extracted from the output of `pandoc -s`
    \providecommand{\tightlist}{%
      \setlength{\itemsep}{0pt}\setlength{\parskip}{0pt}}
    \DefineVerbatimEnvironment{Highlighting}{Verbatim}{commandchars=\\\{\}}
    % Add ',fontsize=\small' for more characters per line
    \newenvironment{Shaded}{}{}
    \newcommand{\KeywordTok}[1]{\textcolor[rgb]{0.00,0.44,0.13}{\textbf{{#1}}}}
    \newcommand{\DataTypeTok}[1]{\textcolor[rgb]{0.56,0.13,0.00}{{#1}}}
    \newcommand{\DecValTok}[1]{\textcolor[rgb]{0.25,0.63,0.44}{{#1}}}
    \newcommand{\BaseNTok}[1]{\textcolor[rgb]{0.25,0.63,0.44}{{#1}}}
    \newcommand{\FloatTok}[1]{\textcolor[rgb]{0.25,0.63,0.44}{{#1}}}
    \newcommand{\CharTok}[1]{\textcolor[rgb]{0.25,0.44,0.63}{{#1}}}
    \newcommand{\StringTok}[1]{\textcolor[rgb]{0.25,0.44,0.63}{{#1}}}
    \newcommand{\CommentTok}[1]{\textcolor[rgb]{0.38,0.63,0.69}{\textit{{#1}}}}
    \newcommand{\OtherTok}[1]{\textcolor[rgb]{0.00,0.44,0.13}{{#1}}}
    \newcommand{\AlertTok}[1]{\textcolor[rgb]{1.00,0.00,0.00}{\textbf{{#1}}}}
    \newcommand{\FunctionTok}[1]{\textcolor[rgb]{0.02,0.16,0.49}{{#1}}}
    \newcommand{\RegionMarkerTok}[1]{{#1}}
    \newcommand{\ErrorTok}[1]{\textcolor[rgb]{1.00,0.00,0.00}{\textbf{{#1}}}}
    \newcommand{\NormalTok}[1]{{#1}}
    
    % Additional commands for more recent versions of Pandoc
    \newcommand{\ConstantTok}[1]{\textcolor[rgb]{0.53,0.00,0.00}{{#1}}}
    \newcommand{\SpecialCharTok}[1]{\textcolor[rgb]{0.25,0.44,0.63}{{#1}}}
    \newcommand{\VerbatimStringTok}[1]{\textcolor[rgb]{0.25,0.44,0.63}{{#1}}}
    \newcommand{\SpecialStringTok}[1]{\textcolor[rgb]{0.73,0.40,0.53}{{#1}}}
    \newcommand{\ImportTok}[1]{{#1}}
    \newcommand{\DocumentationTok}[1]{\textcolor[rgb]{0.73,0.13,0.13}{\textit{{#1}}}}
    \newcommand{\AnnotationTok}[1]{\textcolor[rgb]{0.38,0.63,0.69}{\textbf{\textit{{#1}}}}}
    \newcommand{\CommentVarTok}[1]{\textcolor[rgb]{0.38,0.63,0.69}{\textbf{\textit{{#1}}}}}
    \newcommand{\VariableTok}[1]{\textcolor[rgb]{0.10,0.09,0.49}{{#1}}}
    \newcommand{\ControlFlowTok}[1]{\textcolor[rgb]{0.00,0.44,0.13}{\textbf{{#1}}}}
    \newcommand{\OperatorTok}[1]{\textcolor[rgb]{0.40,0.40,0.40}{{#1}}}
    \newcommand{\BuiltInTok}[1]{{#1}}
    \newcommand{\ExtensionTok}[1]{{#1}}
    \newcommand{\PreprocessorTok}[1]{\textcolor[rgb]{0.74,0.48,0.00}{{#1}}}
    \newcommand{\AttributeTok}[1]{\textcolor[rgb]{0.49,0.56,0.16}{{#1}}}
    \newcommand{\InformationTok}[1]{\textcolor[rgb]{0.38,0.63,0.69}{\textbf{\textit{{#1}}}}}
    \newcommand{\WarningTok}[1]{\textcolor[rgb]{0.38,0.63,0.69}{\textbf{\textit{{#1}}}}}
    
    
    % Define a nice break command that doesn't care if a line doesn't already
    % exist.
    \def\br{\hspace*{\fill} \\* }
    % Math Jax compatibility definitions
    \def\gt{>}
    \def\lt{<}
    \let\Oldtex\TeX
    \let\Oldlatex\LaTeX
    \renewcommand{\TeX}{\textrm{\Oldtex}}
    \renewcommand{\LaTeX}{\textrm{\Oldlatex}}
    % Document parameters
    % Document title
    \title{Actividades Interdisciplinarias I, Tarea I}
    
    
    
    \author{Murillo Vega, Gustavo\\ Peñuelas Alvarez, Adnan Eliacit}
    \date{31 de Agosto 2022}
    
    
    
% Pygments definitions
\makeatletter
\def\PY@reset{\let\PY@it=\relax \let\PY@bf=\relax%
    \let\PY@ul=\relax \let\PY@tc=\relax%
    \let\PY@bc=\relax \let\PY@ff=\relax}
\def\PY@tok#1{\csname PY@tok@#1\endcsname}
\def\PY@toks#1+{\ifx\relax#1\empty\else%
    \PY@tok{#1}\expandafter\PY@toks\fi}
\def\PY@do#1{\PY@bc{\PY@tc{\PY@ul{%
    \PY@it{\PY@bf{\PY@ff{#1}}}}}}}
\def\PY#1#2{\PY@reset\PY@toks#1+\relax+\PY@do{#2}}

\@namedef{PY@tok@w}{\def\PY@tc##1{\textcolor[rgb]{0.73,0.73,0.73}{##1}}}
\@namedef{PY@tok@c}{\let\PY@it=\textit\def\PY@tc##1{\textcolor[rgb]{0.25,0.50,0.50}{##1}}}
\@namedef{PY@tok@cp}{\def\PY@tc##1{\textcolor[rgb]{0.74,0.48,0.00}{##1}}}
\@namedef{PY@tok@k}{\let\PY@bf=\textbf\def\PY@tc##1{\textcolor[rgb]{0.00,0.50,0.00}{##1}}}
\@namedef{PY@tok@kp}{\def\PY@tc##1{\textcolor[rgb]{0.00,0.50,0.00}{##1}}}
\@namedef{PY@tok@kt}{\def\PY@tc##1{\textcolor[rgb]{0.69,0.00,0.25}{##1}}}
\@namedef{PY@tok@o}{\def\PY@tc##1{\textcolor[rgb]{0.40,0.40,0.40}{##1}}}
\@namedef{PY@tok@ow}{\let\PY@bf=\textbf\def\PY@tc##1{\textcolor[rgb]{0.67,0.13,1.00}{##1}}}
\@namedef{PY@tok@nb}{\def\PY@tc##1{\textcolor[rgb]{0.00,0.50,0.00}{##1}}}
\@namedef{PY@tok@nf}{\def\PY@tc##1{\textcolor[rgb]{0.00,0.00,1.00}{##1}}}
\@namedef{PY@tok@nc}{\let\PY@bf=\textbf\def\PY@tc##1{\textcolor[rgb]{0.00,0.00,1.00}{##1}}}
\@namedef{PY@tok@nn}{\let\PY@bf=\textbf\def\PY@tc##1{\textcolor[rgb]{0.00,0.00,1.00}{##1}}}
\@namedef{PY@tok@ne}{\let\PY@bf=\textbf\def\PY@tc##1{\textcolor[rgb]{0.82,0.25,0.23}{##1}}}
\@namedef{PY@tok@nv}{\def\PY@tc##1{\textcolor[rgb]{0.10,0.09,0.49}{##1}}}
\@namedef{PY@tok@no}{\def\PY@tc##1{\textcolor[rgb]{0.53,0.00,0.00}{##1}}}
\@namedef{PY@tok@nl}{\def\PY@tc##1{\textcolor[rgb]{0.63,0.63,0.00}{##1}}}
\@namedef{PY@tok@ni}{\let\PY@bf=\textbf\def\PY@tc##1{\textcolor[rgb]{0.60,0.60,0.60}{##1}}}
\@namedef{PY@tok@na}{\def\PY@tc##1{\textcolor[rgb]{0.49,0.56,0.16}{##1}}}
\@namedef{PY@tok@nt}{\let\PY@bf=\textbf\def\PY@tc##1{\textcolor[rgb]{0.00,0.50,0.00}{##1}}}
\@namedef{PY@tok@nd}{\def\PY@tc##1{\textcolor[rgb]{0.67,0.13,1.00}{##1}}}
\@namedef{PY@tok@s}{\def\PY@tc##1{\textcolor[rgb]{0.73,0.13,0.13}{##1}}}
\@namedef{PY@tok@sd}{\let\PY@it=\textit\def\PY@tc##1{\textcolor[rgb]{0.73,0.13,0.13}{##1}}}
\@namedef{PY@tok@si}{\let\PY@bf=\textbf\def\PY@tc##1{\textcolor[rgb]{0.73,0.40,0.53}{##1}}}
\@namedef{PY@tok@se}{\let\PY@bf=\textbf\def\PY@tc##1{\textcolor[rgb]{0.73,0.40,0.13}{##1}}}
\@namedef{PY@tok@sr}{\def\PY@tc##1{\textcolor[rgb]{0.73,0.40,0.53}{##1}}}
\@namedef{PY@tok@ss}{\def\PY@tc##1{\textcolor[rgb]{0.10,0.09,0.49}{##1}}}
\@namedef{PY@tok@sx}{\def\PY@tc##1{\textcolor[rgb]{0.00,0.50,0.00}{##1}}}
\@namedef{PY@tok@m}{\def\PY@tc##1{\textcolor[rgb]{0.40,0.40,0.40}{##1}}}
\@namedef{PY@tok@gh}{\let\PY@bf=\textbf\def\PY@tc##1{\textcolor[rgb]{0.00,0.00,0.50}{##1}}}
\@namedef{PY@tok@gu}{\let\PY@bf=\textbf\def\PY@tc##1{\textcolor[rgb]{0.50,0.00,0.50}{##1}}}
\@namedef{PY@tok@gd}{\def\PY@tc##1{\textcolor[rgb]{0.63,0.00,0.00}{##1}}}
\@namedef{PY@tok@gi}{\def\PY@tc##1{\textcolor[rgb]{0.00,0.63,0.00}{##1}}}
\@namedef{PY@tok@gr}{\def\PY@tc##1{\textcolor[rgb]{1.00,0.00,0.00}{##1}}}
\@namedef{PY@tok@ge}{\let\PY@it=\textit}
\@namedef{PY@tok@gs}{\let\PY@bf=\textbf}
\@namedef{PY@tok@gp}{\let\PY@bf=\textbf\def\PY@tc##1{\textcolor[rgb]{0.00,0.00,0.50}{##1}}}
\@namedef{PY@tok@go}{\def\PY@tc##1{\textcolor[rgb]{0.53,0.53,0.53}{##1}}}
\@namedef{PY@tok@gt}{\def\PY@tc##1{\textcolor[rgb]{0.00,0.27,0.87}{##1}}}
\@namedef{PY@tok@err}{\def\PY@bc##1{{\setlength{\fboxsep}{\string -\fboxrule}\fcolorbox[rgb]{1.00,0.00,0.00}{1,1,1}{\strut ##1}}}}
\@namedef{PY@tok@kc}{\let\PY@bf=\textbf\def\PY@tc##1{\textcolor[rgb]{0.00,0.50,0.00}{##1}}}
\@namedef{PY@tok@kd}{\let\PY@bf=\textbf\def\PY@tc##1{\textcolor[rgb]{0.00,0.50,0.00}{##1}}}
\@namedef{PY@tok@kn}{\let\PY@bf=\textbf\def\PY@tc##1{\textcolor[rgb]{0.00,0.50,0.00}{##1}}}
\@namedef{PY@tok@kr}{\let\PY@bf=\textbf\def\PY@tc##1{\textcolor[rgb]{0.00,0.50,0.00}{##1}}}
\@namedef{PY@tok@bp}{\def\PY@tc##1{\textcolor[rgb]{0.00,0.50,0.00}{##1}}}
\@namedef{PY@tok@fm}{\def\PY@tc##1{\textcolor[rgb]{0.00,0.00,1.00}{##1}}}
\@namedef{PY@tok@vc}{\def\PY@tc##1{\textcolor[rgb]{0.10,0.09,0.49}{##1}}}
\@namedef{PY@tok@vg}{\def\PY@tc##1{\textcolor[rgb]{0.10,0.09,0.49}{##1}}}
\@namedef{PY@tok@vi}{\def\PY@tc##1{\textcolor[rgb]{0.10,0.09,0.49}{##1}}}
\@namedef{PY@tok@vm}{\def\PY@tc##1{\textcolor[rgb]{0.10,0.09,0.49}{##1}}}
\@namedef{PY@tok@sa}{\def\PY@tc##1{\textcolor[rgb]{0.73,0.13,0.13}{##1}}}
\@namedef{PY@tok@sb}{\def\PY@tc##1{\textcolor[rgb]{0.73,0.13,0.13}{##1}}}
\@namedef{PY@tok@sc}{\def\PY@tc##1{\textcolor[rgb]{0.73,0.13,0.13}{##1}}}
\@namedef{PY@tok@dl}{\def\PY@tc##1{\textcolor[rgb]{0.73,0.13,0.13}{##1}}}
\@namedef{PY@tok@s2}{\def\PY@tc##1{\textcolor[rgb]{0.73,0.13,0.13}{##1}}}
\@namedef{PY@tok@sh}{\def\PY@tc##1{\textcolor[rgb]{0.73,0.13,0.13}{##1}}}
\@namedef{PY@tok@s1}{\def\PY@tc##1{\textcolor[rgb]{0.73,0.13,0.13}{##1}}}
\@namedef{PY@tok@mb}{\def\PY@tc##1{\textcolor[rgb]{0.40,0.40,0.40}{##1}}}
\@namedef{PY@tok@mf}{\def\PY@tc##1{\textcolor[rgb]{0.40,0.40,0.40}{##1}}}
\@namedef{PY@tok@mh}{\def\PY@tc##1{\textcolor[rgb]{0.40,0.40,0.40}{##1}}}
\@namedef{PY@tok@mi}{\def\PY@tc##1{\textcolor[rgb]{0.40,0.40,0.40}{##1}}}
\@namedef{PY@tok@il}{\def\PY@tc##1{\textcolor[rgb]{0.40,0.40,0.40}{##1}}}
\@namedef{PY@tok@mo}{\def\PY@tc##1{\textcolor[rgb]{0.40,0.40,0.40}{##1}}}
\@namedef{PY@tok@ch}{\let\PY@it=\textit\def\PY@tc##1{\textcolor[rgb]{0.25,0.50,0.50}{##1}}}
\@namedef{PY@tok@cm}{\let\PY@it=\textit\def\PY@tc##1{\textcolor[rgb]{0.25,0.50,0.50}{##1}}}
\@namedef{PY@tok@cpf}{\let\PY@it=\textit\def\PY@tc##1{\textcolor[rgb]{0.25,0.50,0.50}{##1}}}
\@namedef{PY@tok@c1}{\let\PY@it=\textit\def\PY@tc##1{\textcolor[rgb]{0.25,0.50,0.50}{##1}}}
\@namedef{PY@tok@cs}{\let\PY@it=\textit\def\PY@tc##1{\textcolor[rgb]{0.25,0.50,0.50}{##1}}}

\def\PYZbs{\char`\\}
\def\PYZus{\char`\_}
\def\PYZob{\char`\{}
\def\PYZcb{\char`\}}
\def\PYZca{\char`\^}
\def\PYZam{\char`\&}
\def\PYZlt{\char`\<}
\def\PYZgt{\char`\>}
\def\PYZsh{\char`\#}
\def\PYZpc{\char`\%}
\def\PYZdl{\char`\$}
\def\PYZhy{\char`\-}
\def\PYZsq{\char`\'}
\def\PYZdq{\char`\"}
\def\PYZti{\char`\~}
% for compatibility with earlier versions
\def\PYZat{@}
\def\PYZlb{[}
\def\PYZrb{]}
\makeatother


    % For linebreaks inside Verbatim environment from package fancyvrb. 
    \makeatletter
        \newbox\Wrappedcontinuationbox 
        \newbox\Wrappedvisiblespacebox 
        \newcommand*\Wrappedvisiblespace {\textcolor{red}{\textvisiblespace}} 
        \newcommand*\Wrappedcontinuationsymbol {\textcolor{red}{\llap{\tiny$\m@th\hookrightarrow$}}} 
        \newcommand*\Wrappedcontinuationindent {3ex } 
        \newcommand*\Wrappedafterbreak {\kern\Wrappedcontinuationindent\copy\Wrappedcontinuationbox} 
        % Take advantage of the already applied Pygments mark-up to insert 
        % potential linebreaks for TeX processing. 
        %        {, <, #, %, $, ' and ": go to next line. 
        %        _, }, ^, &, >, - and ~: stay at end of broken line. 
        % Use of \textquotesingle for straight quote. 
        \newcommand*\Wrappedbreaksatspecials {% 
            \def\PYGZus{\discretionary{\char`\_}{\Wrappedafterbreak}{\char`\_}}% 
            \def\PYGZob{\discretionary{}{\Wrappedafterbreak\char`\{}{\char`\{}}% 
            \def\PYGZcb{\discretionary{\char`\}}{\Wrappedafterbreak}{\char`\}}}% 
            \def\PYGZca{\discretionary{\char`\^}{\Wrappedafterbreak}{\char`\^}}% 
            \def\PYGZam{\discretionary{\char`\&}{\Wrappedafterbreak}{\char`\&}}% 
            \def\PYGZlt{\discretionary{}{\Wrappedafterbreak\char`\<}{\char`\<}}% 
            \def\PYGZgt{\discretionary{\char`\>}{\Wrappedafterbreak}{\char`\>}}% 
            \def\PYGZsh{\discretionary{}{\Wrappedafterbreak\char`\#}{\char`\#}}% 
            \def\PYGZpc{\discretionary{}{\Wrappedafterbreak\char`\%}{\char`\%}}% 
            \def\PYGZdl{\discretionary{}{\Wrappedafterbreak\char`\$}{\char`\$}}% 
            \def\PYGZhy{\discretionary{\char`\-}{\Wrappedafterbreak}{\char`\-}}% 
            \def\PYGZsq{\discretionary{}{\Wrappedafterbreak\textquotesingle}{\textquotesingle}}% 
            \def\PYGZdq{\discretionary{}{\Wrappedafterbreak\char`\"}{\char`\"}}% 
            \def\PYGZti{\discretionary{\char`\~}{\Wrappedafterbreak}{\char`\~}}% 
        } 
        % Some characters . , ; ? ! / are not pygmentized. 
        % This macro makes them "active" and they will insert potential linebreaks 
        \newcommand*\Wrappedbreaksatpunct {% 
            \lccode`\~`\.\lowercase{\def~}{\discretionary{\hbox{\char`\.}}{\Wrappedafterbreak}{\hbox{\char`\.}}}% 
            \lccode`\~`\,\lowercase{\def~}{\discretionary{\hbox{\char`\,}}{\Wrappedafterbreak}{\hbox{\char`\,}}}% 
            \lccode`\~`\;\lowercase{\def~}{\discretionary{\hbox{\char`\;}}{\Wrappedafterbreak}{\hbox{\char`\;}}}% 
            \lccode`\~`\:\lowercase{\def~}{\discretionary{\hbox{\char`\:}}{\Wrappedafterbreak}{\hbox{\char`\:}}}% 
            \lccode`\~`\?\lowercase{\def~}{\discretionary{\hbox{\char`\?}}{\Wrappedafterbreak}{\hbox{\char`\?}}}% 
            \lccode`\~`\!\lowercase{\def~}{\discretionary{\hbox{\char`\!}}{\Wrappedafterbreak}{\hbox{\char`\!}}}% 
            \lccode`\~`\/\lowercase{\def~}{\discretionary{\hbox{\char`\/}}{\Wrappedafterbreak}{\hbox{\char`\/}}}% 
            \catcode`\.\active
            \catcode`\,\active 
            \catcode`\;\active
            \catcode`\:\active
            \catcode`\?\active
            \catcode`\!\active
            \catcode`\/\active 
            \lccode`\~`\~ 	
        }
    \makeatother

    \let\OriginalVerbatim=\Verbatim
    \makeatletter
    \renewcommand{\Verbatim}[1][1]{%
        %\parskip\z@skip
        \sbox\Wrappedcontinuationbox {\Wrappedcontinuationsymbol}%
        \sbox\Wrappedvisiblespacebox {\FV@SetupFont\Wrappedvisiblespace}%
        \def\FancyVerbFormatLine ##1{\hsize\linewidth
            \vtop{\raggedright\hyphenpenalty\z@\exhyphenpenalty\z@
                \doublehyphendemerits\z@\finalhyphendemerits\z@
                \strut ##1\strut}%
        }%
        % If the linebreak is at a space, the latter will be displayed as visible
        % space at end of first line, and a continuation symbol starts next line.
        % Stretch/shrink are however usually zero for typewriter font.
        \def\FV@Space {%
            \nobreak\hskip\z@ plus\fontdimen3\font minus\fontdimen4\font
            \discretionary{\copy\Wrappedvisiblespacebox}{\Wrappedafterbreak}
            {\kern\fontdimen2\font}%
        }%
        
        % Allow breaks at special characters using \PYG... macros.
        \Wrappedbreaksatspecials
        % Breaks at punctuation characters . , ; ? ! and / need catcode=\active 	
        \OriginalVerbatim[#1,codes*=\Wrappedbreaksatpunct]%
    }
    \makeatother

    % Exact colors from NB
    \definecolor{incolor}{HTML}{303F9F}
    \definecolor{outcolor}{HTML}{D84315}
    \definecolor{cellborder}{HTML}{CFCFCF}
    \definecolor{cellbackground}{HTML}{F7F7F7}
    
    % prompt
    \makeatletter
    \newcommand{\boxspacing}{\kern\kvtcb@left@rule\kern\kvtcb@boxsep}
    \makeatother
    \newcommand{\prompt}[4]{
        {\ttfamily\llap{{\color{#2}[#3]:\hspace{3pt}#4}}\vspace{-\baselineskip}}
    }
    

    
    % Prevent overflowing lines due to hard-to-break entities
    \sloppy 
    % Setup hyperref package
    \hypersetup{
      breaklinks=true,  % so long urls are correctly broken across lines
      colorlinks=true,
      urlcolor=urlcolor,
      linkcolor=linkcolor,
      citecolor=citecolor,
      }
    % Slightly bigger margins than the latex defaults
    
    \geometry{verbose,tmargin=1in,bmargin=1in,lmargin=1in,rmargin=1in}
    
    

\begin{document}
    
    \maketitle
    
    

    
    \hypertarget{crecimiento-de-la-bacteria-v.-natriegens-en-medio-de-cultivo-con-ph-de-7.85}{%
\section{\texorpdfstring{Crecimiento de la bacteria \emph{V.
natriegens} en medio de cultivo con pH de
7.85}{1. Crecimiento de la bacteria V. natriegens en medio de cultivo con pH de 7.85}}\label{crecimiento-de-la-bacteria-v.-natriegens-en-medio-de-cultivo-con-ph-de-7.85}}

    En el experimento se recopilarón los siguientes datos del crecimiento
de la población de bacterias. Donde ``Índice de Tiempo'' se refiere a
intervalos de 16 minutos.

    \begin{longtable}[]{@{}lll@{}}
\toprule
Tiempo (minutos) & Índice de Tiempo & Densidad de Población \\
\midrule
\endhead
0 & 0 & 0.028 \\
16 & 1 & 0.047 \\
32 & 2 & 0.082 \\
48 & 3 & 0.141 \\
64 & 4 & 0.240 \\
80 & 5 & 0.381 \\
\bottomrule
\end{longtable}

    \hypertarget{mathrmi-repitase-el-anuxe1lisis-de-datos-visto-en-clases-para-esta-tabla.}{%
\subsection{\texorpdfstring{\(\mathrm{i}\)) Repitase el análisis de
datos visto en clases para esta
tabla.}{\textbackslash mathrm\{i\}) Repitase el análisis de datos visto en clases para esta tabla.}}\label{mathrmi-repitase-el-anuxe1lisis-de-datos-visto-en-clases-para-esta-tabla.}}

    \hypertarget{notaciuxf3n}{%
\subsubsection{Notación}\label{notaciuxf3n}}

Usamos \(t\) para denotar el intervalo de tiempo de 16 minutos en que se
mide la densidad de población \(B_t\); dónde \(B_t\) se mide al inicio
del intervalo y \(t=0\) es el primer intervalo. Así \(B_t\) ha sido
medido a los \(t\times 16\) minutos. Usamos \(\Delta\) para asociar el
símbolo \(\Delta X_y\) con \(X_{y+1}-X_y\) por igualdad.

    \hypertarget{gruxe1ficas-de-datos}{%
\subsubsection{Gráficas de Datos}\label{gruxe1ficas-de-datos}}

Aplicando transformaciones a los datos esperamos encontrar una relación
lineal. Gran parte del código no es interesante y es más para obtener las gráficas o tablas. Por experimentar
con el software al final conocemos formas de hacer el trabajo más compacto, pero por no darse a la tarea de romper cosas y repararlas se deja para cuando haya más tiempo.

    \begin{tcolorbox}[breakable, size=fbox, boxrule=1pt, pad at break*=1mm,colback=cellbackground, colframe=cellborder]
\prompt{In}{incolor}{1}{\boxspacing}
\begin{Verbatim}[commandchars=\\\{\}]
\PY{c+c1}{\PYZsh{} !! Variables Globales: Tabla1, Tabla2}
\PY{k}{def} \PY{n+nf}{MostrarGraficaTabla}\PY{p}{(}\PY{n}{tabla}\PY{p}{,} \PY{n}{nombre}\PY{o}{=}\PY{k+kc}{None}\PY{p}{,} \PY{n}{ticks}\PY{o}{=}\PY{k+kc}{None}\PY{p}{,} \PY{n}{figsize}\PY{o}{=}\PY{l+m+mi}{4}\PY{p}{,} \PY{n}{fontsize}\PY{o}{=}\PY{l+m+mi}{8}\PY{p}{,} \PY{n}{label}\PY{o}{=}\PY{k+kc}{None}\PY{p}{,} \PY{n}{dotted}\PY{o}{=}\PY{k+kc}{False}\PY{p}{,}
                       \PY{n}{pointsize}\PY{o}{=}\PY{l+m+mi}{10}\PY{p}{)}\PY{p}{:}
    \PY{n}{G} \PY{o}{=} \PY{n}{points}\PY{p}{(}\PY{n}{tabla}\PY{p}{,} \PY{n}{rgbcolor}\PY{o}{=}\PY{p}{(}\PY{l+m+mi}{0}\PY{p}{,}\PY{l+m+mi}{0}\PY{p}{,}\PY{l+m+mi}{0}\PY{p}{)}\PY{p}{,} \PY{n}{pointsize}\PY{o}{=}\PY{n}{pointsize}\PY{p}{,} \PY{n}{legend\PYZus{}label}\PY{o}{=}\PY{n}{label}\PY{p}{)}
    \PY{k}{if} \PY{n}{dotted}\PY{p}{:}
        \PY{n}{G} \PY{o}{+}\PY{o}{=} \PY{n}{list\PYZus{}plot}\PY{p}{(}\PY{n}{tabla}\PY{p}{,} \PY{n}{plotjoined}\PY{o}{=}\PY{k+kc}{True}\PY{p}{,} \PY{n}{linestyle}\PY{o}{=}\PY{l+s+s2}{\PYZdq{}}\PY{l+s+s2}{:}\PY{l+s+s2}{\PYZdq{}}\PY{p}{,} \PY{n}{rgbcolor}\PY{o}{=}\PY{p}{(}\PY{l+m+mf}{0.6}\PY{p}{,}\PY{l+m+mi}{0}\PY{p}{,}\PY{l+m+mi}{0}\PY{p}{)}\PY{p}{)}
    \PY{n}{show}\PY{p}{(}\PY{n}{G}\PY{p}{,} \PY{n}{title}\PY{o}{=}\PY{n}{nombre}\PY{p}{,} \PY{n}{ticks}\PY{o}{=}\PY{n}{ticks}\PY{p}{,}
                \PY{n}{figsize}\PY{o}{=}\PY{n}{figsize}\PY{p}{,} \PY{n}{fontsize}\PY{o}{=}\PY{n}{fontsize}\PY{p}{,} \PY{n}{dpi}\PY{o}{=}\PY{l+m+mi}{120}\PY{p}{)}
    
\PY{k}{def} \PY{n+nf}{GraficaTabla}\PY{p}{(}\PY{n}{tabla}\PY{p}{,} \PY{n}{nombre}\PY{o}{=}\PY{k+kc}{None}\PY{p}{,} \PY{n}{ticks}\PY{o}{=}\PY{k+kc}{None}\PY{p}{,} \PY{n}{figsize}\PY{o}{=}\PY{l+m+mi}{4}\PY{p}{,} \PY{n}{fontsize}\PY{o}{=}\PY{l+m+mi}{8}\PY{p}{,} \PY{n}{label}\PY{o}{=}\PY{k+kc}{None}\PY{p}{,} \PY{n}{dotted}\PY{o}{=}\PY{k+kc}{False}\PY{p}{,}
                \PY{n}{pointsize}\PY{o}{=}\PY{l+m+mi}{10}\PY{p}{)}\PY{p}{:}
    \PY{n}{G} \PY{o}{=} \PY{n}{points}\PY{p}{(}\PY{n}{tabla}\PY{p}{,} \PY{n}{rgbcolor}\PY{o}{=}\PY{p}{(}\PY{l+m+mi}{0}\PY{p}{,}\PY{l+m+mi}{0}\PY{p}{,}\PY{l+m+mi}{0}\PY{p}{)}\PY{p}{,} \PY{n}{pointsize}\PY{o}{=}\PY{n}{pointsize}\PY{p}{,} \PY{n}{legend\PYZus{}label}\PY{o}{=}\PY{n}{label}\PY{p}{)}
    \PY{k}{if} \PY{n}{dotted}\PY{p}{:}
        \PY{n}{G} \PY{o}{+}\PY{o}{=} \PY{n}{list\PYZus{}plot}\PY{p}{(}\PY{n}{tabla}\PY{p}{,} \PY{n}{plotjoined}\PY{o}{=}\PY{k+kc}{True}\PY{p}{,} \PY{n}{linestyle}\PY{o}{=}\PY{l+s+s2}{\PYZdq{}}\PY{l+s+s2}{:}\PY{l+s+s2}{\PYZdq{}}\PY{p}{,} \PY{n}{rgbcolor}\PY{o}{=}\PY{p}{(}\PY{l+m+mf}{0.6}\PY{p}{,}\PY{l+m+mi}{0}\PY{p}{,}\PY{l+m+mi}{0}\PY{p}{)}\PY{p}{)}
    \PY{k}{return} \PY{n}{plot}\PY{p}{(}\PY{n}{G}\PY{p}{)}
    
\PY{n}{Tabla1} \PY{o}{=} \PY{n+nb}{list}\PY{p}{(}\PY{n+nb}{map}\PY{p}{(}\PY{k}{lambda} \PY{n}{P}\PY{p}{:} \PY{p}{(}\PY{n}{P}\PY{p}{[}\PY{l+m+mi}{0}\PY{p}{]}\PY{p}{,} \PY{n+nb}{float}\PY{p}{(}\PY{n}{P}\PY{p}{[}\PY{l+m+mi}{1}\PY{p}{]}\PY{p}{)}\PY{p}{)}\PY{p}{,}
        \PY{p}{[}\PY{p}{(}\PY{l+m+mi}{0}\PY{p}{,} \PY{l+m+mf}{0.028}\PY{p}{)}\PY{p}{,}
        \PY{p}{(}\PY{l+m+mi}{1}\PY{p}{,} \PY{l+m+mf}{0.047}\PY{p}{)}\PY{p}{,}
        \PY{p}{(}\PY{l+m+mi}{2}\PY{p}{,} \PY{l+m+mf}{0.082}\PY{p}{)}\PY{p}{,}
        \PY{p}{(}\PY{l+m+mi}{3}\PY{p}{,} \PY{l+m+mf}{0.141}\PY{p}{)}\PY{p}{,}
        \PY{p}{(}\PY{l+m+mi}{4}\PY{p}{,} \PY{l+m+mf}{0.240}\PY{p}{)}\PY{p}{,}
        \PY{p}{(}\PY{l+m+mi}{5}\PY{p}{,} \PY{l+m+mf}{0.381}\PY{p}{)}\PY{p}{]}\PY{p}{)}\PY{p}{)}


\PY{n}{Tabla2} \PY{o}{=} \PY{n+nb}{list}\PY{p}{(}\PY{p}{)}
\PY{k}{for} \PY{n}{t} \PY{o+ow}{in} \PY{n+nb}{range}\PY{p}{(}\PY{l+m+mi}{5}\PY{p}{)}\PY{p}{:}
    \PY{n}{Tabla2}\PY{o}{.}\PY{n}{append}\PY{p}{(} \PY{p}{(}\PY{n}{t}\PY{p}{,} \PY{n}{Tabla1}\PY{p}{[}\PY{n}{t}\PY{o}{+}\PY{l+m+mi}{1}\PY{p}{]}\PY{p}{[}\PY{l+m+mi}{1}\PY{p}{]}\PY{o}{\PYZhy{}}\PY{n}{Tabla1}\PY{p}{[}\PY{n}{t}\PY{p}{]}\PY{p}{[}\PY{l+m+mi}{1}\PY{p}{]}\PY{p}{)} \PY{p}{)} \PY{c+c1}{\PYZsh{} \PYZlt{}\PYZhy{} aumento B\PYZus{}\PYZob{}t+1\PYZcb{}\PYZhy{}B\PYZus{}t}
    
    
\PY{n}{MostrarGraficaTabla}\PY{p}{(}\PY{n}{Tabla1}\PY{p}{,} \PY{n}{nombre}\PY{o}{=}\PY{l+s+s2}{\PYZdq{}}\PY{l+s+s2}{Fig. 1 \PYZdl{}B\PYZus{}t\PYZdl{} contra \PYZdl{}t\PYZdl{}}\PY{l+s+s2}{\PYZdq{}}\PY{p}{,} \PY{n}{ticks} \PY{o}{=} \PY{p}{[}\PY{l+m+mi}{1}\PY{p}{,}\PY{k+kc}{None}\PY{p}{]}\PY{p}{)}
\PY{n}{MostrarGraficaTabla}\PY{p}{(}\PY{n}{Tabla2}\PY{p}{,} \PY{n}{nombre}\PY{o}{=}\PY{l+s+s2}{\PYZdq{}}\PY{l+s+s2}{Fig. 2 \PYZdl{}B\PYZus{}}\PY{l+s+s2}{\PYZob{}}\PY{l+s+s2}{t+1\PYZcb{}\PYZhy{}B\PYZus{}t\PYZdl{} contra \PYZdl{}t\PYZdl{}}\PY{l+s+s2}{\PYZdq{}}\PY{p}{,} \PY{n}{ticks} \PY{o}{=} \PY{p}{[}\PY{l+m+mi}{1}\PY{p}{,}\PY{k+kc}{None}\PY{p}{]}\PY{p}{,} \PY{n}{dotted} \PY{o}{=} \PY{k+kc}{True}\PY{p}{)}
\end{Verbatim}
\end{tcolorbox}

    \begin{center}
    \adjustimage{max size={0.9\linewidth}{0.9\paperheight}}{output_6_0.png}
    \end{center}
    { \hspace*{\fill} \\}
    
    \begin{center}
    \adjustimage{max size={0.9\linewidth}{0.9\paperheight}}{output_6_1.png}
    \end{center}
    { \hspace*{\fill} \\}
    
    Se observa de la figura 1 que cada vez está más separado un valor de
otro, con lo que el crecimiento se hace cada vez mayor. Pero de la
figura 2 se observa que el crecimiento \(B_{t+1}-B_t\) tampoco es lineal
sobre el tiempo.

Como \(B_t\) también aumenta con el tiempo, se piensa que podría
relacionarse mejor (linealmente) con \(B_{t+1}-B_{t}\) que \(t\). Dado
que ambas son cantidades que aumentan con el tiempo.

    \begin{tcolorbox}[breakable, size=fbox, boxrule=1pt, pad at break*=1mm,colback=cellbackground, colframe=cellborder]
\prompt{In}{incolor}{2}{\boxspacing}
\begin{Verbatim}[commandchars=\\\{\}]
\PY{c+c1}{\PYZsh{} !! Variable Global: Tabla3 !!}
\PY{n}{Tabla3} \PY{o}{=} \PY{n+nb}{list}\PY{p}{(}\PY{p}{)}
\PY{k}{for} \PY{n}{t} \PY{o+ow}{in} \PY{n+nb}{range}\PY{p}{(}\PY{l+m+mi}{5}\PY{p}{)}\PY{p}{:}
    \PY{n}{Tabla3}\PY{o}{.}\PY{n}{append}\PY{p}{(} \PY{p}{(}\PY{n}{Tabla1}\PY{p}{[}\PY{n}{t}\PY{p}{]}\PY{p}{[}\PY{l+m+mi}{1}\PY{p}{]}\PY{p}{,} \PY{n}{Tabla2}\PY{p}{[}\PY{n}{t}\PY{p}{]}\PY{p}{[}\PY{l+m+mi}{1}\PY{p}{]}\PY{p}{)} \PY{p}{)}
    
\PY{n}{MostrarGraficaTabla}\PY{p}{(}\PY{n}{Tabla3}\PY{p}{,} \PY{n}{nombre}\PY{o}{=}\PY{l+s+s2}{\PYZdq{}}\PY{l+s+s2}{Fig. 3 \PYZdl{}B\PYZus{}}\PY{l+s+s2}{\PYZob{}}\PY{l+s+s2}{t+1\PYZcb{}\PYZhy{}B\PYZus{}t\PYZdl{} contra \PYZdl{}B\PYZus{}t\PYZdl{}}\PY{l+s+s2}{\PYZdq{}}\PY{p}{)}
\end{Verbatim}
\end{tcolorbox}

    \begin{center}
    \adjustimage{max size={0.9\linewidth}{0.9\paperheight}}{output_8_0.png}
    \end{center}
    { \hspace*{\fill} \\}
    
    Se puede observar que estos datos muestran una mejor distribución lineal
que los de la figura 1 o 2.

    \hypertarget{ecuaciuxf3n-dinuxe1mica}{%
\subsubsection{Ecuación Dinámica}\label{ecuaciuxf3n-dinuxe1mica}}

Por ahora usaremos \(\hat B_t\) para referirnos a la aproximación del
valor \(B_t\) dado por un modelo lineal. \(k\) se refiere a cualquier
pendiente de la recta de este modelo (recta que aproxima los datos de la
figura 3). De manera que la ecuación de la recta será descrita por una
ecuación en terminos de \(y = \Delta \hat B_t\) y \(x = \hat B_t\), de
la forma \(y = kx+b\) si \(k\) es la pendiente y \(b\) es alguna
constante real. Es decir

\[\Delta \hat B_t = k \hat B_t +b,\] o en términos de meramente las
densidades \[\hat B_{t+1} = (1+k)\hat B_t +b.\]

Sea \(K = 1+k\), en la ecuación anterior:

    \[\hat B_{t+1} = K\hat B_t +b.\] Luego para \(t = n = 1,2,\ldots\) \[
    \hat B_n = K \hat B_{n-1}+b = K \left(K \hat B_{n-2} +b\right)+b = K^2 \hat B_{n-2}+Kb+b = \ldots \\
    = K^n \hat B_0+ b\sum^{n-1}_{j=0}K^j.
\] Sin embargo para \(n=0\) lo anterior nos da la identidad ignorando
las igualdades intermedias. Por lo que también es válido en ese caso
usar la última expresión.

Si \(K\neq 1\), (que pasa solo cuando \(k=0\), y no estamos considerando
valores tan bajos) \[
    \hat B_n = K^n \hat B_0+ b\frac{1-K^n}{1-K}.
\]

Así la solución de la ecuación dinámica de \(t\) propuesta al inicio de
esta sección es \[
    \hat B_t = (1+k)^t B_0 +b \frac{(1+k)^t-1}{k} \\
    = (1+k)^t \left(0.028 +\frac{b}{k}\right)-\frac{b}{k}.
\]

Si escogemos \(b\) tal que la recta pasa por un punto de los datos
\((\hat B_t, \Delta \hat B_t) = (x, y)\) con \(x= B_t, y=\Delta B_t\);
tenemos de la primer ecuación en esta sección que

\[b = y -k x.\]

Por lo que podemos ajustar este modelo solo por el parámetro \(k\neq 0\)
y escoger algún punto \((x,y)\) (o valor de \(b\)). Pero como afecta
solo como constante y el rango de \(b\) es pequeño con los datos
considerados se toma como 0.

    \hypertarget{pendiente}{%
\subsubsection{Pendiente}\label{pendiente}}

Para \(t=0,1,2,3\): La pendiente \(m_t\) de la recta que conecta
\((B_t,\Delta B_t)\) con \((B_{t+1}, \Delta B_{t+1})\) es
\[\frac{\Delta B_{t+1} - \Delta B_t}{B_{t+1} - B_t} = \frac{\Delta B_{t+1}-\Delta B_t}{\Delta B_t}.\]
O lo que es lo mismo \[m_t = \frac{\Delta B_{t+1}}{\Delta B_t}-1.\]

    \textbf{Figura 4.}

\begin{longtable}[]{@{}llll@{}}
\toprule
\(t\) & \(B_t\) & \(\Delta B_t\) & \(m_t\) \\
\midrule
\endhead
0 & 0.028 & 0.019 & 0.842 \\
1 & 0.047 & 0.035 & 0.686 \\
2 & 0.082 & 0.059 & 0.678 \\
3 & 0.141 & 0.099 & 0.424 \\
4 & 0.240 & 0.141 & \\
5 & 0.381 & & \\
\bottomrule
\end{longtable}

    Para tomar una buena pendiente \(k\) podríamos usar la mediana ya que da
la coincidencia que tres puntos son casi colineales en la figura 3. Para
ajustar qué tanto se parece la pendiente al promedio o a la mediana
tomamos distintos valores de \(p\) en la función de promedio ponderado
definida en el código. Si \(p\) es 0 el promedio ponderado es el normal,
si \(p\) es grande entonces los valores lejanos de la media tienen más
peso.

    \begin{tcolorbox}[breakable, size=fbox, boxrule=1pt, pad at break*=1mm,colback=cellbackground, colframe=cellborder]
\prompt{In}{incolor}{3}{\boxspacing}
\begin{Verbatim}[commandchars=\\\{\}]
\PY{c+c1}{\PYZsh{} !! Variable Global: lista\PYZus{}m !!}
\PY{n}{lista\PYZus{}m} \PY{o}{=} \PY{n+nb}{list}\PY{p}{(}\PY{p}{)}
\PY{k}{for} \PY{n}{t} \PY{o+ow}{in} \PY{n+nb}{range}\PY{p}{(}\PY{n+nb}{len}\PY{p}{(}\PY{n}{Tabla3}\PY{p}{)}\PY{o}{\PYZhy{}}\PY{l+m+mi}{1}\PY{p}{)}\PY{p}{:}
        \PY{n}{lista\PYZus{}m}\PY{o}{.}\PY{n}{append}\PY{p}{(}\PY{n}{Tabla3}\PY{p}{[}\PY{n}{t}\PY{o}{+}\PY{l+m+mi}{1}\PY{p}{]}\PY{p}{[}\PY{l+m+mi}{1}\PY{p}{]}\PY{o}{/}\PY{n}{Tabla3}\PY{p}{[}\PY{n}{t}\PY{p}{]}\PY{p}{[}\PY{l+m+mi}{1}\PY{p}{]}\PY{o}{\PYZhy{}}\PY{l+m+mi}{1}\PY{p}{)}

\PY{c+c1}{\PYZsh{}promedio ponderado por inverso de distancia promedio (elevado a potencia p)}
\PY{c+c1}{\PYZsh{}cuando mayor sea p, mayor la influencia de la distancia}
\PY{c+c1}{\PYZsh{}cuando p = 0, es simplemente el promedio de pesos iguales}
\PY{c+c1}{\PYZsh{}https://en.wikipedia.org/wiki/Inverse\PYZus{}distance\PYZus{}weighting}

\PY{c+c1}{\PYZsh{} Cuando se usa se imprime una tabla con los pesos que se le pone a cada valor por default}
\PY{c+c1}{\PYZsh{} Los digitos son solo para los datos que se imprimen, no el valor regresado por la función}
\PY{k}{def} \PY{n+nf}{promedio\PYZus{}ponderado}\PY{p}{(}\PY{n}{conjunto}\PY{p}{,} \PY{n}{p}\PY{p}{,} \PY{n}{verbose}\PY{o}{=}\PY{k+kc}{False}\PY{p}{,} \PY{n}{digitos}\PY{o}{=}\PY{l+m+mi}{3}\PY{p}{,} \PY{n}{nombre}\PY{o}{=}\PY{k+kc}{None}\PY{p}{)}\PY{p}{:}
    \PY{n}{promedio} \PY{o}{=} \PY{n}{mean}\PY{p}{(}\PY{n}{conjunto}\PY{p}{)}
    \PY{k}{if} \PY{n}{verbose}\PY{p}{:}
        \PY{n+nb}{print}\PY{p}{(}\PY{n}{nombre}\PY{p}{)}
        \PY{n+nb}{print}\PY{p}{(}\PY{l+s+s2}{\PYZdq{}}\PY{l+s+s2}{Valor}\PY{l+s+se}{\PYZbs{}t}\PY{l+s+s2}{Peso}\PY{l+s+s2}{\PYZdq{}}\PY{p}{)}

    \PY{n}{n} \PY{o}{=} \PY{n+nb}{len}\PY{p}{(}\PY{n}{conjunto}\PY{p}{)}
    \PY{n}{suma\PYZus{}pesos} \PY{o}{=} \PY{l+m+mi}{0}
    \PY{n}{suma\PYZus{}valor\PYZus{}peso} \PY{o}{=} \PY{l+m+mi}{0}

    \PY{k}{for} \PY{n}{valor} \PY{o+ow}{in} \PY{n}{conjunto}\PY{p}{:}
        \PY{k}{if} \PY{n+nb}{abs}\PY{p}{(}\PY{n}{valor} \PY{o}{\PYZhy{}} \PY{n}{promedio}\PY{p}{)} \PY{o}{==} \PY{l+m+mi}{0}\PY{p}{:} \PY{k}{return} \PY{n}{valor} 
        \PY{n}{suma\PYZus{}pesos} \PY{o}{+}\PY{o}{=} \PY{l+m+mi}{1}\PY{o}{/}\PY{n+nb}{abs}\PY{p}{(}\PY{n}{valor} \PY{o}{\PYZhy{}} \PY{n}{promedio}\PY{p}{)}\PY{o}{*}\PY{o}{*}\PY{n}{p}
        \PY{n}{suma\PYZus{}valor\PYZus{}peso} \PY{o}{+}\PY{o}{=} \PY{n}{valor}\PY{o}{/}\PY{n+nb}{abs}\PY{p}{(}\PY{n}{valor} \PY{o}{\PYZhy{}} \PY{n}{promedio}\PY{p}{)}\PY{o}{*}\PY{o}{*}\PY{n}{p}
        \PY{k}{if} \PY{n}{verbose}\PY{p}{:}
            \PY{n+nb}{print}\PY{p}{(}\PY{n+nb}{round}\PY{p}{(}\PY{n}{valor}\PY{p}{,} \PY{n}{digitos}\PY{p}{)}\PY{p}{,} \PY{l+s+s2}{\PYZdq{}}\PY{l+s+se}{\PYZbs{}t}\PY{l+s+se}{\PYZbs{}t}\PY{l+s+s2}{\PYZdq{}}\PY{p}{,} \PY{n+nb}{round}\PY{p}{(}\PY{l+m+mi}{1}\PY{o}{/}\PY{n+nb}{abs}\PY{p}{(}\PY{n}{valor} \PY{o}{\PYZhy{}} \PY{n}{promedio}\PY{p}{)}\PY{o}{*}\PY{o}{*}\PY{n}{p}\PY{p}{,} \PY{n}{digitos}\PY{p}{)}\PY{p}{)}
    
    \PY{k}{return} \PY{n}{suma\PYZus{}valor\PYZus{}peso}\PY{o}{/}\PY{n}{suma\PYZus{}pesos}
\end{Verbatim}
\end{tcolorbox}

    Como ejemplo, el promedio con pesos iguales para cada pendiente es el
siguiente.

    \begin{tcolorbox}[breakable, size=fbox, boxrule=1pt, pad at break*=1mm,colback=cellbackground, colframe=cellborder]
\prompt{In}{incolor}{4}{\boxspacing}
\begin{Verbatim}[commandchars=\\\{\}]
\PY{n}{promedio\PYZus{}ponderado}\PY{p}{(}\PY{n}{lista\PYZus{}m}\PY{p}{,} \PY{n}{p}\PY{o}{=}\PY{l+m+mi}{0}\PY{p}{)}
\end{Verbatim}
\end{tcolorbox}

            \begin{tcolorbox}[breakable, size=fbox, boxrule=.5pt, pad at break*=1mm, opacityfill=0]
\prompt{Out}{outcolor}{4}{\boxspacing}
\begin{Verbatim}[commandchars=\\\{\}]
0.65750701870238
\end{Verbatim}
\end{tcolorbox}
        
    \hypertarget{graficas-de-soluciones}{%
\subsubsection{Graficas de Soluciones}\label{graficas-de-soluciones}}

    Tomamos un promedio ponderado de las pendientes y ajustamos a ojo por
una mejor solución.

    \begin{tcolorbox}[breakable, size=fbox, boxrule=1pt, pad at break*=1mm,colback=cellbackground, colframe=cellborder]
\prompt{In}{incolor}{5}{\boxspacing}
\begin{Verbatim}[commandchars=\\\{\}]
\PY{c+c1}{\PYZsh{} !! Var. Global SolucionDensidad, B\PYZus{}0, digitos\PYZus{}entrada, formato !!}
\PY{k+kn}{import} \PY{n+nn}{pandas} \PY{k}{as} \PY{n+nn}{pd}
\PY{n}{pd}\PY{o}{.}\PY{n}{set\PYZus{}option}\PY{p}{(}\PY{l+s+s1}{\PYZsq{}}\PY{l+s+s1}{display.precision}\PY{l+s+s1}{\PYZsq{}}\PY{p}{,} \PY{l+m+mi}{3}\PY{n}{r}\PY{p}{)}
\PY{n}{pd}\PY{o}{.}\PY{n}{set\PYZus{}option}\PY{p}{(}\PY{l+s+s2}{\PYZdq{}}\PY{l+s+s2}{display.latex.repr}\PY{l+s+s2}{\PYZdq{}}\PY{p}{,} \PY{k+kc}{True}\PY{p}{)}

\PY{n}{B\PYZus{}0} \PY{o}{=} \PY{n}{Tabla1}\PY{p}{[}\PY{l+m+mi}{0}\PY{p}{]}\PY{p}{[}\PY{l+m+mi}{1}\PY{p}{]}

\PY{n}{digitos\PYZus{}entrada} \PY{o}{=} \PY{l+m+mi}{4}
\PY{n}{digitos\PYZus{}salida} \PY{o}{=} \PY{l+m+mi}{3}

\PY{k}{def} \PY{n+nf}{TablaRecta}\PY{p}{(}\PY{n}{k}\PY{p}{,} \PY{n}{b}\PY{o}{=}\PY{l+m+mi}{0}\PY{p}{)}\PY{p}{:}
    \PY{n}{y} \PY{o}{=} \PY{l+s+s2}{\PYZdq{}}\PY{l+s+s2}{\PYZdl{}}\PY{l+s+se}{\PYZbs{}\PYZbs{}}\PY{l+s+s2}{Delta }\PY{l+s+se}{\PYZbs{}\PYZbs{}}\PY{l+s+s2}{hat B\PYZus{}t = }\PY{l+s+s2}{\PYZdq{}}\PY{o}{+}\PY{n+nb}{str}\PY{p}{(}\PY{n+nb}{round}\PY{p}{(}\PY{n}{k}\PY{p}{,}\PY{n}{digitos\PYZus{}entrada}\PY{p}{)}\PY{p}{)}\PY{o}{+}\PY{l+s+s2}{\PYZdq{}}\PY{l+s+se}{\PYZbs{}\PYZbs{}}\PY{l+s+s2}{hat B\PYZus{}t}\PY{l+s+s2}{\PYZdq{}}
    \PY{k}{if} \PY{n}{b}\PY{o}{!=}\PY{l+m+mi}{0}\PY{p}{:}
        \PY{n}{y}\PY{o}{+}\PY{o}{=}\PY{l+s+s2}{\PYZdq{}}\PY{l+s+s2}{+(}\PY{l+s+s2}{\PYZdq{}}\PY{o}{+}\PY{n+nb}{str}\PY{p}{(}\PY{n+nb}{round}\PY{p}{(}\PY{n}{b}\PY{p}{,}\PY{l+m+mi}{5}\PY{p}{)}\PY{p}{)}\PY{o}{+}\PY{l+s+s2}{\PYZdq{}}\PY{l+s+s2}{)}\PY{l+s+s2}{\PYZdq{}}
    \PY{n}{y}\PY{o}{+}\PY{o}{=} \PY{l+s+s2}{\PYZdq{}}\PY{l+s+s2}{\PYZdl{}}\PY{l+s+s2}{\PYZdq{}}
    \PY{n}{df} \PY{o}{=} \PY{n}{pd}\PY{o}{.}\PY{n}{DataFrame}\PY{p}{(}\PY{n+nb}{map}\PY{p}{(}\PY{k}{lambda} \PY{n}{P}\PY{p}{:} \PY{p}{(}\PY{n}{P}\PY{p}{[}\PY{l+m+mi}{0}\PY{p}{]}\PY{p}{,} \PY{n}{k}\PY{o}{*}\PY{n}{P}\PY{p}{[}\PY{l+m+mi}{0}\PY{p}{]}\PY{o}{+}\PY{n}{b}\PY{p}{,} \PY{n}{P}\PY{p}{[}\PY{l+m+mi}{1}\PY{p}{]}\PY{p}{)}\PY{p}{,} \PY{n}{Tabla3}\PY{p}{)}\PY{p}{,}
              \PY{n}{columns}\PY{o}{=}\PY{p}{[}\PY{l+s+s2}{\PYZdq{}}\PY{l+s+s2}{\PYZdl{}}\PY{l+s+se}{\PYZbs{}\PYZbs{}}\PY{l+s+s2}{hat B\PYZus{}t\PYZdl{}}\PY{l+s+s2}{\PYZdq{}}\PY{p}{,} \PY{n}{y}\PY{p}{,} \PY{l+s+s2}{\PYZdq{}}\PY{l+s+s2}{\PYZdl{}}\PY{l+s+se}{\PYZbs{}\PYZbs{}}\PY{l+s+s2}{Delta B\PYZus{}t\PYZdl{}}\PY{l+s+s2}{\PYZdq{}}\PY{p}{]}\PY{p}{,} \PY{n}{index}\PY{o}{=}\PY{n+nb}{range}\PY{p}{(}\PY{n+nb}{len}\PY{p}{(}\PY{n}{Tabla3}\PY{p}{)}\PY{p}{)}\PY{p}{)}
    \PY{k}{return} \PY{n}{df}

\PY{n}{SolucionDensidad} \PY{o}{=} \PY{k}{lambda} \PY{n}{k}\PY{p}{,}\PY{n}{b}\PY{o}{=}\PY{l+m+mi}{0}\PY{p}{:} \PY{p}{(}\PY{l+m+mi}{1}\PY{o}{+}\PY{n}{k}\PY{p}{)}\PY{o}{\PYZca{}}\PY{n}{t} \PY{o}{*}\PY{p}{(}\PY{n}{B\PYZus{}0} \PY{o}{+}\PY{n}{b}\PY{o}{/}\PY{n}{k}\PY{p}{)}\PY{o}{\PYZhy{}}\PY{n}{b}\PY{o}{/}\PY{n}{k}
    
\PY{k}{def} \PY{n+nf}{TablaDensidad}\PY{p}{(}\PY{n}{k}\PY{p}{,} \PY{n}{b}\PY{o}{=}\PY{l+m+mi}{0}\PY{p}{)}\PY{p}{:}
    \PY{n}{y} \PY{o}{=} \PY{l+s+s2}{\PYZdq{}}\PY{l+s+s2}{\PYZdl{}}\PY{l+s+se}{\PYZbs{}\PYZbs{}}\PY{l+s+s2}{hat B\PYZus{}t = (}\PY{l+s+s2}{\PYZdq{}}
    \PY{n}{y} \PY{o}{+}\PY{o}{=} \PY{n+nb}{str}\PY{p}{(}\PY{n+nb}{round}\PY{p}{(}\PY{l+m+mi}{1}\PY{o}{+}\PY{n}{k}\PY{p}{,}\PY{n}{digitos\PYZus{}entrada}\PY{p}{)}\PY{p}{)}\PY{o}{+}\PY{l+s+s2}{\PYZdq{}}\PY{l+s+s2}{)\PYZca{}t}\PY{l+s+s2}{\PYZdq{}}
    \PY{n}{y} \PY{o}{+}\PY{o}{=} \PY{l+s+s2}{\PYZdq{}}\PY{l+s+se}{\PYZbs{}\PYZbs{}}\PY{l+s+s2}{cdot}\PY{l+s+s2}{\PYZdq{}}\PY{o}{+}\PY{n+nb}{str}\PY{p}{(}\PY{n+nb}{round}\PY{p}{(}\PY{n}{B\PYZus{}0}\PY{o}{+}\PY{n}{b}\PY{o}{/}\PY{n}{k}\PY{p}{,} \PY{n}{digitos\PYZus{}entrada}\PY{p}{)}\PY{p}{)}
    \PY{k}{if} \PY{n}{b}\PY{o}{!=}\PY{l+m+mi}{0}\PY{p}{:}
        \PY{n}{y}\PY{o}{+}\PY{o}{=} \PY{l+s+s2}{\PYZdq{}}\PY{l+s+s2}{\PYZhy{}(}\PY{l+s+s2}{\PYZdq{}}\PY{o}{+}\PY{n+nb}{str}\PY{p}{(}\PY{n+nb}{round}\PY{p}{(}\PY{n}{b}\PY{p}{,}\PY{l+m+mi}{5}\PY{p}{)}\PY{p}{)}\PY{o}{+}\PY{l+s+s2}{\PYZdq{}}\PY{l+s+s2}{)}\PY{l+s+s2}{\PYZdq{}}
    \PY{n}{y} \PY{o}{+}\PY{o}{=} \PY{l+s+s2}{\PYZdq{}}\PY{l+s+s2}{\PYZdl{}}\PY{l+s+s2}{\PYZdq{}}
    \PY{n}{df} \PY{o}{=} \PY{n}{pd}\PY{o}{.}\PY{n}{DataFrame}\PY{p}{(}\PY{n+nb}{map}\PY{p}{(}\PY{k}{lambda} \PY{n}{P}\PY{p}{:} \PY{p}{(}\PY{n+nb}{float}\PY{p}{(}\PY{n}{SolucionDensidad}\PY{p}{(}\PY{n}{k}\PY{p}{,}\PY{n}{b}\PY{p}{)}\PY{p}{(}\PY{n}{t}\PY{o}{=}\PY{n}{P}\PY{p}{[}\PY{l+m+mi}{0}\PY{p}{]}\PY{p}{)}\PY{p}{)}\PY{p}{,} \PY{n}{P}\PY{p}{[}\PY{l+m+mi}{1}\PY{p}{]}\PY{p}{)}\PY{p}{,} \PY{n}{Tabla1}\PY{p}{)}\PY{p}{,}
                      \PY{n}{columns}\PY{o}{=}\PY{p}{[}\PY{n}{y}\PY{p}{,} \PY{l+s+s2}{\PYZdq{}}\PY{l+s+s2}{\PYZdl{}B\PYZus{}t\PYZdl{}}\PY{l+s+s2}{\PYZdq{}}\PY{p}{]}\PY{p}{,}
                     \PY{n}{index}\PY{o}{=}\PY{n+nb}{range}\PY{p}{(}\PY{n+nb}{len}\PY{p}{(}\PY{n}{Tabla1}\PY{p}{)}\PY{p}{)}\PY{p}{)}
    \PY{k}{return} \PY{n}{df}

\PY{k}{def} \PY{n+nf}{mostrarGraficaRecta}\PY{p}{(}\PY{n}{k}\PY{p}{,} \PY{n}{titulo}\PY{p}{,} \PY{n}{b}\PY{o}{=}\PY{l+m+mi}{0}\PY{p}{)}\PY{p}{:}
    \PY{n}{var}\PY{p}{(}\PY{l+s+s1}{\PYZsq{}}\PY{l+s+s1}{t}\PY{l+s+s1}{\PYZsq{}}\PY{p}{)}
    \PY{n}{legend}\PY{o}{=}\PY{l+s+s2}{\PYZdq{}}\PY{l+s+s2}{\PYZdl{}}\PY{l+s+se}{\PYZbs{}\PYZbs{}}\PY{l+s+s2}{Delta }\PY{l+s+se}{\PYZbs{}\PYZbs{}}\PY{l+s+s2}{hat B\PYZus{}t = }\PY{l+s+s2}{\PYZdq{}}\PY{o}{+}\PY{n+nb}{str}\PY{p}{(}\PY{n+nb}{round}\PY{p}{(}\PY{n}{k}\PY{p}{,} \PY{n}{digitos\PYZus{}entrada}\PY{p}{)}\PY{p}{)}\PY{o}{+}\PY{l+s+s2}{\PYZdq{}}\PY{l+s+se}{\PYZbs{}\PYZbs{}}\PY{l+s+s2}{hat B\PYZus{}t}\PY{l+s+s2}{\PYZdq{}}
    \PY{k}{if} \PY{n}{b}\PY{o}{!=}\PY{l+m+mi}{0}\PY{p}{:} \PY{n}{legend}\PY{o}{+}\PY{o}{=}\PY{l+s+s2}{\PYZdq{}}\PY{l+s+s2}{+(}\PY{l+s+s2}{\PYZdq{}}\PY{o}{+}\PY{n+nb}{str}\PY{p}{(}\PY{n+nb}{round}\PY{p}{(}\PY{n}{b}\PY{p}{,}\PY{l+m+mi}{5}\PY{p}{)}\PY{p}{)}\PY{o}{+}\PY{l+s+s2}{\PYZdq{}}\PY{l+s+s2}{)}\PY{l+s+s2}{\PYZdq{}}
    \PY{n}{legend}\PY{o}{+}\PY{o}{=}\PY{l+s+s2}{\PYZdq{}}\PY{l+s+s2}{\PYZdl{}}\PY{l+s+s2}{\PYZdq{}}
    \PY{n}{GraficaCambio} \PY{o}{=} \PY{n}{plot}\PY{p}{(}\PY{n}{k}\PY{o}{*}\PY{n}{t}\PY{o}{+}\PY{n}{b}\PY{p}{,} \PY{c+c1}{\PYZsh{} \PYZlt{}\PYZhy{} EC RECTA}
                          \PY{c+c1}{\PYZsh{} t de 0 a max\PYZob{}x: (x,y) en Tabla3\PYZcb{}*1.1}
                         \PY{p}{(}\PY{n}{t}\PY{p}{,}\PY{l+m+mi}{0}\PY{p}{,}\PY{n+nb}{max}\PY{p}{(}\PY{n+nb}{map}\PY{p}{(}\PY{k}{lambda} \PY{n}{P}\PY{p}{:} \PY{n}{P}\PY{p}{[}\PY{l+m+mi}{0}\PY{p}{]}\PY{p}{,} \PY{n}{Tabla3}\PY{p}{)}\PY{p}{)}\PY{o}{*}\PY{l+m+mf}{1.1}\PY{p}{)}\PY{p}{,}
                         \PY{n}{legend\PYZus{}label}\PY{o}{=}\PY{n}{legend}\PY{p}{)}
    \PY{n}{GraficaCambio}\PY{o}{.}\PY{n}{set\PYZus{}legend\PYZus{}options}\PY{p}{(}\PY{n}{font\PYZus{}size}\PY{o}{=}\PY{l+m+mi}{8}\PY{p}{)}
    
    \PY{n}{show}\PY{p}{(}\PY{n}{GraficaCambio}\PY{o}{+}\PY{n}{GraficaTabla}\PY{p}{(}\PY{n}{tabla}\PY{o}{=}\PY{n}{Tabla3}\PY{p}{,} \PY{n}{label}\PY{o}{=}\PY{l+s+s2}{\PYZdq{}}\PY{l+s+s2}{\PYZdl{}}\PY{l+s+se}{\PYZbs{}\PYZbs{}}\PY{l+s+s2}{Delta B\PYZus{}t\PYZdl{}}\PY{l+s+s2}{\PYZdq{}}\PY{p}{,} \PY{n}{pointsize}\PY{o}{=}\PY{l+m+mi}{5}\PY{p}{)}\PY{p}{,}
         \PY{n}{title}\PY{o}{=}\PY{n}{titulo}\PY{p}{,}
         \PY{n}{figsize}\PY{o}{=}\PY{l+m+mi}{4}\PY{p}{,} \PY{n}{fontsize}\PY{o}{=}\PY{l+m+mi}{8}\PY{p}{,} \PY{n}{ticks}\PY{o}{=}\PY{p}{[}\PY{l+m+mf}{0.05}\PY{p}{,}\PY{k+kc}{None}\PY{p}{]}\PY{p}{,} \PY{n}{axes\PYZus{}labels}\PY{o}{=}\PY{p}{[}\PY{l+s+s1}{\PYZsq{}}\PY{l+s+s1}{\PYZdl{}B\PYZus{}t\PYZdl{}}\PY{l+s+s1}{\PYZsq{}}\PY{p}{,}\PY{k+kc}{None}\PY{p}{]}\PY{p}{,} \PY{n}{dpi}\PY{o}{=}\PY{l+m+mi}{120}\PY{p}{)}
    \PY{k}{return} \PY{n}{TablaRecta}\PY{p}{(}\PY{n}{k}\PY{p}{,} \PY{n}{b}\PY{p}{)}

\PY{k}{def} \PY{n+nf}{mostrarGraficaDensidad}\PY{p}{(}\PY{n}{k}\PY{p}{,} \PY{n}{titulo}\PY{p}{,} \PY{n}{b}\PY{o}{=}\PY{l+m+mi}{0}\PY{p}{)}\PY{p}{:}
    \PY{n}{var}\PY{p}{(}\PY{l+s+s1}{\PYZsq{}}\PY{l+s+s1}{t}\PY{l+s+s1}{\PYZsq{}}\PY{p}{)}
    \PY{n}{legend} \PY{o}{=} \PY{l+s+s2}{\PYZdq{}}\PY{l+s+s2}{\PYZdl{}}\PY{l+s+se}{\PYZbs{}\PYZbs{}}\PY{l+s+s2}{hat B\PYZus{}t =}\PY{l+s+s2}{\PYZdq{}}\PY{o}{+}\PY{n+nb}{str}\PY{p}{(}\PY{n+nb}{round}\PY{p}{(}\PY{l+m+mi}{1}\PY{o}{+}\PY{n}{k}\PY{p}{,} \PY{n}{digitos\PYZus{}entrada}\PY{p}{)}\PY{p}{)}\PY{o}{+}\PY{l+s+s2}{\PYZdq{}}\PY{l+s+s2}{\PYZca{}t}\PY{l+s+se}{\PYZbs{}\PYZbs{}}\PY{l+s+s2}{cdot(}\PY{l+s+s2}{\PYZdq{}}\PY{o}{+}\PY{n+nb}{str}\PY{p}{(}\PY{n+nb}{round}\PY{p}{(}\PY{n}{B\PYZus{}0}\PY{o}{+}\PY{n}{b}\PY{o}{/}\PY{n}{k}\PY{p}{,} \PY{l+m+mi}{5}\PY{p}{)}\PY{p}{)}\PY{o}{+}\PY{l+s+s2}{\PYZdq{}}\PY{l+s+s2}{)}\PY{l+s+s2}{\PYZdq{}}
    \PY{k}{if} \PY{n}{b}\PY{o}{!=} \PY{l+m+mi}{0}\PY{p}{:} \PY{n}{legend} \PY{o}{+}\PY{o}{=} \PY{l+s+s2}{\PYZdq{}}\PY{l+s+s2}{\PYZhy{}(}\PY{l+s+s2}{\PYZdq{}}\PY{o}{+}\PY{n+nb}{str}\PY{p}{(}\PY{n+nb}{round}\PY{p}{(}\PY{n}{b}\PY{o}{/}\PY{n}{k}\PY{p}{,}\PY{l+m+mi}{5}\PY{p}{)}\PY{p}{)}\PY{o}{+}\PY{l+s+s2}{\PYZdq{}}\PY{l+s+s2}{)}\PY{l+s+s2}{\PYZdq{}}
    \PY{n}{legend} \PY{o}{+}\PY{o}{=} \PY{l+s+s2}{\PYZdq{}}\PY{l+s+s2}{\PYZdl{}}\PY{l+s+s2}{\PYZdq{}}
    \PY{n}{GraficaDensidad} \PY{o}{=} \PY{n}{plot}\PY{p}{(}\PY{n}{SolucionDensidad}\PY{p}{(}\PY{n}{k}\PY{p}{,} \PY{n}{b}\PY{o}{=}\PY{l+m+mi}{0}\PY{p}{)}\PY{p}{,} \PY{c+c1}{\PYZsh{} \PYZlt{}\PYZhy{} MODELO}
                           \PY{p}{(}\PY{n}{t}\PY{p}{,} \PY{l+m+mi}{0}\PY{p}{,} \PY{n+nb}{max}\PY{p}{(}\PY{n+nb}{map}\PY{p}{(}\PY{k}{lambda} \PY{n}{P}\PY{p}{:} \PY{n}{P}\PY{p}{[}\PY{l+m+mi}{0}\PY{p}{]}\PY{p}{,} \PY{n}{Tabla1}\PY{p}{)}\PY{p}{)}\PY{o}{*}\PY{l+m+mf}{1.1}\PY{p}{)}\PY{p}{,}
                           \PY{n}{legend\PYZus{}label}\PY{o}{=}\PY{n}{legend}\PY{p}{)}
    \PY{n}{GraficaDensidad}\PY{o}{.}\PY{n}{set\PYZus{}legend\PYZus{}options}\PY{p}{(}\PY{n}{font\PYZus{}size}\PY{o}{=}\PY{l+m+mi}{8}\PY{p}{)}
    
    \PY{n}{show}\PY{p}{(}\PY{n}{GraficaDensidad}\PY{o}{+}\PY{n}{GraficaTabla}\PY{p}{(}\PY{n}{Tabla1}\PY{p}{,} \PY{n}{label}\PY{o}{=}\PY{l+s+s2}{\PYZdq{}}\PY{l+s+s2}{\PYZdl{}B\PYZus{}t\PYZdl{}}\PY{l+s+s2}{\PYZdq{}}\PY{p}{,} \PY{n}{pointsize}\PY{o}{=}\PY{l+m+mi}{10}\PY{p}{)}\PY{p}{,}
         \PY{n}{title}\PY{o}{=}\PY{n}{titulo}\PY{p}{,}
         \PY{n}{figsize}\PY{o}{=}\PY{l+m+mi}{4}\PY{p}{,} \PY{n}{fontsize}\PY{o}{=}\PY{l+m+mi}{8}\PY{p}{,} \PY{n}{axes\PYZus{}labels}\PY{o}{=}\PY{p}{[}\PY{l+s+s1}{\PYZsq{}}\PY{l+s+s1}{\PYZdl{}t\PYZdl{}}\PY{l+s+s1}{\PYZsq{}}\PY{p}{,}\PY{k+kc}{None}\PY{p}{]}\PY{p}{,} \PY{n}{dpi}\PY{o}{=}\PY{l+m+mi}{120}\PY{p}{)}
    \PY{k}{return} \PY{n}{TablaDensidad}\PY{p}{(}\PY{n}{k}\PY{p}{,} \PY{n}{b}\PY{p}{)}
\end{Verbatim}
\end{tcolorbox}

    \begin{tcolorbox}[breakable, size=fbox, boxrule=1pt, pad at break*=1mm,colback=cellbackground, colframe=cellborder]
\prompt{In}{incolor}{6}{\boxspacing}
\begin{Verbatim}[commandchars=\\\{\}]
\PY{n}{b} \PY{o}{=} \PY{k}{lambda} \PY{n}{P}\PY{p}{,}\PY{n}{k}\PY{p}{:} \PY{n+nb}{float}\PY{p}{(}\PY{n}{P}\PY{p}{[}\PY{l+m+mi}{1}\PY{p}{]} \PY{o}{\PYZhy{}} \PY{n}{k}\PY{o}{*}\PY{n}{P}\PY{p}{[}\PY{l+m+mi}{0}\PY{p}{]}\PY{p}{)} \PY{c+c1}{\PYZsh{} y \PYZhy{}kx}

\PY{k}{def} \PY{n+nf}{Graficas}\PY{p}{(}\PY{n}{k}\PY{p}{,} \PY{n}{n\PYZus{}fig}\PY{p}{,} \PY{n}{P}\PY{o}{=}\PY{p}{[}\PY{l+m+mi}{0}\PY{n}{r}\PY{p}{,}\PY{l+m+mi}{0}\PY{n}{r}\PY{p}{]}\PY{p}{,} \PY{n}{ajuste\PYZus{}b}\PY{o}{=}\PY{l+m+mi}{0}\PY{p}{)}\PY{p}{:}
    \PY{n}{b\PYZus{}redondeado} \PY{o}{=} \PY{n+nb}{round}\PY{p}{(}\PY{n}{b}\PY{p}{(}\PY{n}{P}\PY{p}{,}\PY{n}{k}\PY{p}{)}\PY{o}{*}\PY{n}{ajuste\PYZus{}b}\PY{p}{,}\PY{l+m+mi}{5}\PY{p}{)}
    \PY{n}{nombre} \PY{o}{=} \PY{l+s+s2}{\PYZdq{}}\PY{l+s+s2}{Figura }\PY{l+s+s2}{\PYZdq{}}\PY{o}{+}\PY{n+nb}{str}\PY{p}{(}\PY{n}{n\PYZus{}fig}\PY{p}{)}\PY{o}{+}\PY{l+s+s2}{\PYZdq{}}\PY{l+s+s2}{.1  Tasa de Crecimiento Relativo \PYZdl{}k=}\PY{l+s+s2}{\PYZdq{}}\PY{o}{+}\PY{n+nb}{str}\PY{p}{(}\PY{n+nb}{round}\PY{p}{(}\PY{n}{k}\PY{p}{,} \PY{n}{digitos\PYZus{}entrada}\PY{p}{)}\PY{p}{)}
    \PY{k}{if} \PY{n}{ajuste\PYZus{}b} \PY{o}{!=} \PY{l+m+mi}{0}\PY{p}{:}
        \PY{n}{nombre}\PY{o}{+}\PY{o}{=}\PY{l+s+s2}{\PYZdq{}}\PY{l+s+s2}{,b=}\PY{l+s+s2}{\PYZdq{}}\PY{o}{+}\PY{n+nb}{str}\PY{p}{(}\PY{n}{b\PYZus{}redondeado}\PY{o}{*}\PY{n}{ajuste\PYZus{}b}\PY{p}{)}
    \PY{n}{nombre}\PY{o}{+}\PY{o}{=} \PY{l+s+s2}{\PYZdq{}}\PY{l+s+s2}{\PYZdl{}}\PY{l+s+s2}{\PYZdq{}}
    \PY{n}{Tabla1} \PY{o}{=} \PY{n}{mostrarGraficaRecta}\PY{p}{(}\PY{n}{k}\PY{p}{,} \PY{n}{nombre}\PY{p}{,} \PY{n}{b}\PY{p}{(}\PY{n}{P}\PY{p}{,}\PY{n}{k}\PY{p}{)}\PY{o}{*}\PY{n}{ajuste\PYZus{}b}\PY{p}{)}
    
    \PY{n}{nombre} \PY{o}{=} \PY{l+s+s2}{\PYZdq{}}\PY{l+s+s2}{Figura }\PY{l+s+s2}{\PYZdq{}}\PY{o}{+}\PY{n+nb}{str}\PY{p}{(}\PY{n}{n\PYZus{}fig}\PY{p}{)}\PY{o}{+}\PY{l+s+s2}{\PYZdq{}}\PY{l+s+s2}{.2  Solución del Sistema Dinámico \PYZdl{}k=}\PY{l+s+s2}{\PYZdq{}}\PY{o}{+}\PY{n+nb}{str}\PY{p}{(}\PY{n+nb}{round}\PY{p}{(}\PY{n}{k}\PY{p}{,} \PY{n}{digitos\PYZus{}entrada}\PY{p}{)}\PY{p}{)}
    \PY{k}{if} \PY{n}{ajuste\PYZus{}b} \PY{o}{!=}\PY{l+m+mi}{0}\PY{p}{:}
        \PY{n}{nombre} \PY{o}{+}\PY{o}{=} \PY{l+s+s2}{\PYZdq{}}\PY{l+s+s2}{,b=}\PY{l+s+s2}{\PYZdq{}}\PY{o}{+}\PY{n+nb}{str}\PY{p}{(}\PY{n+nb}{round}\PY{p}{(}\PY{n}{b}\PY{p}{(}\PY{n}{P}\PY{p}{,}\PY{n}{k}\PY{p}{)}\PY{o}{*}\PY{n}{ajuste\PYZus{}b}\PY{p}{,}\PY{l+m+mi}{5}\PY{p}{)}\PY{p}{)}
    \PY{n}{nombre}\PY{o}{+}\PY{o}{=} \PY{l+s+s2}{\PYZdq{}}\PY{l+s+s2}{\PYZdl{}}\PY{l+s+s2}{\PYZdq{}}
    \PY{n}{Tabla2} \PY{o}{=} \PY{n}{mostrarGraficaDensidad}\PY{p}{(}\PY{n}{k}\PY{p}{,} \PY{n}{nombre}\PY{p}{,} \PY{n}{b}\PY{p}{(}\PY{n}{P}\PY{p}{,} \PY{n}{k}\PY{p}{)}\PY{o}{*}\PY{n}{ajuste\PYZus{}b}\PY{p}{)}
    \PY{k}{return} \PY{n}{Tabla1}\PY{p}{,}\PY{n}{Tabla2}
\end{Verbatim}
\end{tcolorbox}

    Damos dos ajustes para el modelo. El primero dado por la figura 5 se
ajusta para acercarse más a solo los puntos que parecen colineales. El
de la figura 6 intenta minimizar la distancia de la recta a todos los
puntos.

    \begin{tcolorbox}[breakable, size=fbox, boxrule=1pt, pad at break*=1mm,colback=cellbackground, colframe=cellborder]
\prompt{In}{incolor}{7}{\boxspacing}
\begin{Verbatim}[commandchars=\\\{\}]
\PY{n}{k1} \PY{o}{=} \PY{n+nb}{float}\PY{p}{(}\PY{n}{promedio\PYZus{}ponderado}\PY{p}{(}\PY{n}{lista\PYZus{}m}\PY{p}{[}\PY{p}{:}\PY{o}{\PYZhy{}}\PY{l+m+mi}{1}\PY{p}{]}\PY{p}{,} \PY{n}{p}\PY{o}{=}\PY{p}{(}\PY{l+m+mf}{1.11}\PY{o}{+}\PY{l+m+mf}{80E\PYZhy{}4}\PY{p}{)}\PY{p}{)}\PY{p}{)}
\PY{n}{counter} \PY{o}{=} \PY{l+m+mi}{3}

\PY{k}{for} \PY{n}{tabla} \PY{o+ow}{in} \PY{n}{Graficas}\PY{p}{(}\PY{n}{k1}\PY{p}{,} \PY{l+m+mi}{5}\PY{p}{)}\PY{p}{:}
    \PY{k}{if} \PY{k+kc}{False}\PY{p}{:} \PY{n+nb}{print}\PY{p}{(}\PY{n}{tabla}\PY{o}{.}\PY{n}{to\PYZus{}markdown}\PY{p}{(}\PY{p}{)}\PY{p}{)}
\end{Verbatim}
\end{tcolorbox}

    \begin{center}
    \adjustimage{max size={0.9\linewidth}{0.9\paperheight}}{output_23_0.png}
    \end{center}
    { \hspace*{\fill} \\}
    
    \begin{center}
    \adjustimage{max size={0.9\linewidth}{0.9\paperheight}}{output_23_1.png}
    \end{center}
    { \hspace*{\fill} \\}
    
    \begin{longtable}[]{@{}rrrr@{}}
\toprule
& \(\hat B_t\) & \(\Delta \hat B_t = 0.712\hat B_t\) & \(\Delta B_t\) \\
\midrule
\endhead
0 & 0.028 & 0.0199346 & 0.019 \\
1 & 0.047 & 0.0334617 & 0.035 \\
2 & 0.082 & 0.05838 & 0.059 \\
3 & 0.141 & 0.100385 & 0.099 \\
4 & 0.24 & 0.170868 & 0.141 \\
\bottomrule
\end{longtable}

\begin{longtable}[]{@{}rrr@{}}
\toprule
& \(\hat B_t = (1.712)^t\cdot0.028\) & \(B_t\) \\
\midrule
\endhead
0 & 0.028 & 0.028 \\
1 & 0.0479346 & 0.047 \\
2 & 0.0820617 & 0.082 \\
3 & 0.140486 & 0.141 \\
4 & 0.240504 & 0.24 \\
5 & 0.411732 & 0.381 \\
\bottomrule
\end{longtable}

    \begin{tcolorbox}[breakable, size=fbox, boxrule=1pt, pad at break*=1mm,colback=cellbackground, colframe=cellborder]
\prompt{In}{incolor}{8}{\boxspacing}
\begin{Verbatim}[commandchars=\\\{\}]
\PY{n}{k2} \PY{o}{=} \PY{n}{promedio\PYZus{}ponderado}\PY{p}{(}\PY{n}{lista\PYZus{}m}\PY{p}{[}\PY{p}{:}\PY{p}{]}\PY{p}{,} \PY{n}{p}\PY{o}{=}\PY{l+m+mi}{8}\PY{p}{)}\PY{o}{+}\PY{n}{promedio\PYZus{}ponderado}\PY{p}{(}\PY{n}{lista\PYZus{}m}\PY{p}{[}\PY{p}{:}\PY{o}{\PYZhy{}}\PY{l+m+mi}{1}\PY{p}{]}\PY{p}{,} \PY{n}{p}\PY{o}{=}\PY{l+m+mf}{1.11}\PY{p}{)}
\PY{n}{k2} \PY{o}{=} \PY{n+nb}{float}\PY{p}{(}\PY{n}{k2}\PY{o}{/}\PY{l+m+mi}{2}\PY{p}{)}

\PY{n}{counter} \PY{o}{=} \PY{l+m+mi}{3}
\PY{k}{for} \PY{n}{tabla} \PY{o+ow}{in} \PY{n}{Graficas}\PY{p}{(}\PY{n}{k2}\PY{p}{,} \PY{l+m+mi}{6}\PY{p}{)}\PY{p}{:}
    \PY{k}{if} \PY{k+kc}{False}\PY{p}{:} \PY{n+nb}{print}\PY{p}{(}\PY{n}{tabla}\PY{o}{.}\PY{n}{to\PYZus{}markdown}\PY{p}{(}\PY{p}{)}\PY{p}{)}
\end{Verbatim}
\end{tcolorbox}

    \begin{center}
    \adjustimage{max size={0.9\linewidth}{0.9\paperheight}}{output_25_0.png}
    \end{center}
    { \hspace*{\fill} \\}
    
    \begin{center}
    \adjustimage{max size={0.9\linewidth}{0.9\paperheight}}{output_25_1.png}
    \end{center}
    { \hspace*{\fill} \\}
    
    \begin{longtable}[]{@{}rrrr@{}}
\toprule
& \(\hat B_t\) & \(\Delta \hat B_t = 0.6953\hat B_t\) &
\(\Delta B_t\) \\
\midrule
\endhead
0 & 0.028 & 0.0194684 & 0.019 \\
1 & 0.047 & 0.0326792 & 0.035 \\
2 & 0.082 & 0.0570147 & 0.059 \\
3 & 0.141 & 0.0980375 & 0.099 \\
4 & 0.24 & 0.166872 & 0.141 \\
\bottomrule
\end{longtable}

\begin{longtable}[]{@{}rrr@{}}
\toprule
& \(\hat B_t = (1.6953)^t\cdot0.028\) & \(B_t\) \\
\midrule
\endhead
0 & 0.028 & 0.028 \\
1 & 0.0474684 & 0.047 \\
2 & 0.0804733 & 0.082 \\
3 & 0.136426 & 0.141 \\
4 & 0.231284 & 0.24 \\
5 & 0.392096 & 0.381 \\
\bottomrule
\end{longtable}

    El modelo preferido depende del conocimiento del experto en el área del
experimento. Si el experto sabe que los datos de la tasa de crecimiento
relativo se deben ajustar a una recta con un varianza bastante pequeña,
se prefiere el modelo de la figura 5 y se ignora el último punto como
error de medición. Pues tenemos en densidad, una diferencia con los
datos de a lo mucho el último digito.

El segundo modelo dado por la figura 6 se prefiere, si es el
conocimiento del experto es que la varianza del experimento sí debería
ser alta en algunos intervalos de tiempo. De manera que los datos
anormales pueden ser bien medidos, y reducir la distancia a todos los
puntos nos acercaría al valor esperado de un experimento en estas
condiciones.

    \hypertarget{ii-comparaciuxf3n-de-tasa-de-crecimiento-relativo-de-v.-natriegens-a-ph-7.85-con-tasa-de-crecimiento-relativo-23-a-ph-6.25}{%
\subsection{ii) Comparación de Tasa de Crecimiento Relativo de V.
natriegens a pH 7.85 con Tasa de Crecimiento Relativo 2/3 a pH
6.25}\label{ii-comparaciuxf3n-de-tasa-de-crecimiento-relativo-de-v.-natriegens-a-ph-7.85-con-tasa-de-crecimiento-relativo-23-a-ph-6.25}}

Comparamos los resultados anteriores con el modelo cuando \(k=2/3\).

    \begin{tcolorbox}[breakable, size=fbox, boxrule=1pt, pad at break*=1mm,colback=cellbackground, colframe=cellborder]
\prompt{In}{incolor}{9}{\boxspacing}
\begin{Verbatim}[commandchars=\\\{\}]
\PY{k}{def} \PY{n+nf}{TablaComparacion}\PY{p}{(}\PY{n}{k\PYZus{}tuple}\PY{p}{,} \PY{n}{Datos}\PY{p}{)}\PY{p}{:}
    \PY{n}{columnas} \PY{o}{=} \PY{p}{[}\PY{l+s+s2}{\PYZdq{}}\PY{l+s+s2}{\PYZdl{}}\PY{l+s+se}{\PYZbs{}\PYZbs{}}\PY{l+s+s2}{hat B\PYZus{}t\PYZdl{}}\PY{l+s+s2}{\PYZdq{}}\PY{p}{]}
    \PY{c+c1}{\PYZsh{} Convertir a string redondeado si float}
    \PY{k}{def} \PY{n+nf}{k\PYZus{}display}\PY{p}{(}\PY{n}{k}\PY{p}{)}\PY{p}{:}
        \PY{k}{if} \PY{n+nb}{type}\PY{p}{(}\PY{n}{k}\PY{p}{)} \PY{o}{==} \PY{n+nb}{type}\PY{p}{(}\PY{l+m+mi}{2}\PY{o}{/}\PY{l+m+mi}{3}\PY{p}{)}\PY{p}{:}
            \PY{k}{return} \PY{n+nb}{str}\PY{p}{(}\PY{n}{k}\PY{p}{)}
        \PY{k}{else}\PY{p}{:}
            \PY{k}{return} \PY{n+nb}{str}\PY{p}{(}\PY{n+nb}{round}\PY{p}{(}\PY{n}{k}\PY{p}{,} \PY{l+m+mi}{3}\PY{p}{)}\PY{p}{)}
    \PY{n}{columnas}\PY{o}{.}\PY{n}{extend}\PY{p}{(}\PY{n+nb}{list}\PY{p}{(}
        \PY{n+nb}{map}\PY{p}{(}\PY{k}{lambda} \PY{n}{k}\PY{p}{:} \PY{n+nb}{str}\PY{p}{(}\PY{l+s+s2}{\PYZdq{}}\PY{l+s+s2}{\PYZdl{}}\PY{l+s+s2}{\PYZdq{}}\PY{o}{+}\PY{n}{k\PYZus{}display}\PY{p}{(}\PY{n}{k}\PY{p}{)}\PY{o}{+}\PY{l+s+s2}{\PYZdq{}}\PY{l+s+se}{\PYZbs{}\PYZbs{}}\PY{l+s+s2}{hat B\PYZus{}t\PYZdl{}}\PY{l+s+s2}{\PYZdq{}}\PY{p}{)}\PY{p}{,} \PY{n}{k\PYZus{}tuple}\PY{p}{)} \PY{p}{)}\PY{p}{)}

    \PY{n}{df} \PY{o}{=} \PY{n+nb}{list}\PY{p}{(}\PY{n+nb}{map}\PY{p}{(}\PY{k}{lambda} \PY{n}{P}\PY{p}{:} \PY{p}{[}\PY{n}{P}\PY{p}{[}\PY{l+m+mi}{0}\PY{p}{]}\PY{p}{]}\PY{p}{,} \PY{n}{Datos}\PY{p}{)}\PY{p}{)}
    \PY{k}{for} \PY{n}{k} \PY{o+ow}{in} \PY{n}{k\PYZus{}tuple}\PY{p}{:}
        \PY{k}{for} \PY{n}{fila} \PY{o+ow}{in} \PY{n}{df}\PY{p}{:}
            \PY{n}{fila}\PY{o}{.}\PY{n}{insert}\PY{p}{(}\PY{o}{\PYZhy{}}\PY{l+m+mi}{1}\PY{p}{,} \PY{n+nb}{float}\PY{p}{(}\PY{n}{k}\PY{o}{*}\PY{n}{fila}\PY{p}{[}\PY{l+m+mi}{0}\PY{p}{]}\PY{p}{)}\PY{p}{)}
    \PY{n}{df} \PY{o}{=} \PY{n}{pd}\PY{o}{.}\PY{n}{DataFrame}\PY{p}{(}\PY{n}{df}\PY{p}{,} \PY{n}{columns}\PY{o}{=}\PY{n}{columnas}\PY{p}{,} \PY{n}{index}\PY{o}{=}\PY{n+nb}{range}\PY{p}{(}\PY{n+nb}{len}\PY{p}{(}\PY{n}{Datos}\PY{p}{)}\PY{p}{)}\PY{p}{)}
    \PY{k}{return} \PY{n}{df}

\PY{n}{Tabla12} \PY{o}{=} \PY{n}{TablaComparacion}\PY{p}{(}\PY{p}{(}\PY{l+m+mi}{2}\PY{o}{/}\PY{l+m+mi}{3}\PY{p}{,} \PY{n+nb}{float}\PY{p}{(}\PY{n}{k1}\PY{p}{)}\PY{p}{)}\PY{p}{,} \PY{n}{Tabla3}\PY{p}{)}
\PY{c+c1}{\PYZsh{}Tabla12.to\PYZus{}clipboard()}
\PY{c+c1}{\PYZsh{}Tabla122 = pd.read\PYZus{}clipboard()}
\PY{c+c1}{\PYZsh{}print(Tabla122.to\PYZus{}markdown())}
\end{Verbatim}
\end{tcolorbox}

    \begin{longtable}[]{@{}rrrr@{}}
\toprule
& \(\hat B_t\) & \(2/3\hat B_t\) & \(0.712\hat B_t\) \\
\midrule
\endhead
0 & 0.019 & 0.013 & 0.028 \\
1 & 0.031 & 0.022 & 0.047 \\
2 & 0.055 & 0.039 & 0.082 \\
3 & 0.094 & 0.067 & 0.141 \\
4 & 0.16 & 0.114 & 0.24 \\
\bottomrule
\end{longtable}

    En la tabla anterior \(B_t\) es para los datos de pH 7.85. Podemos
observar que para una misma densidad de población \(B_t\), la tasa de
crecimiento es siempre inferior para el pH 6.25.

Sin embargo cuando \(t=4\) el modelo con tasa relativa de crecimiento
\(k=2/3\) se asemeja más al dato medido para el pH de 7.85. De hecho la
media normal para las posibles constantes \(k\) para el pH de 7.85 es
\(0.657\) que es menor a \(2/3\).

Por lo que bien podría ser que \(k=2/3\) sea una solución para el
crecimiento de la bacteria a pH 7.85. Pero esto no significa que vayan a
ser las mismas, pues \(k=2/3\) se ajusta a los datos de su experimento
de pH 6.25 de manera similar que \(k=0.712\) se ajusta a los datos del
experimento a pH 7.85.

\textbf{Modelo con \(k=2/3\) comparado con los datos del experimento con
pH 6.25}

\begin{longtable}[]{@{}lll@{}}
\toprule
\(t\) & \(\hat B_t\) & \(B_t\) \\
\midrule
\endhead
0 & 0.022 & 0.022 \\
1 & 0.037 & 0.036 \\
2 & 0.061 & 0.06 \\
3 & 0.102 & 0.101 \\
4 & 0.17 & 0.169 \\
5 & 0.238 & 0.266 \\
\bottomrule
\end{longtable}

La diferencia es a lo más un último digito hasta el último dato que se
ajusta peor. De manera que la diferencia entre las tasas de crecimientos
está justificada bajo los mismos supuestos. Es decir, supuestos como que
se debe ajustar mejor el modelo a los primeros puntos.

    \hypertarget{b_t-b_t1-r-b_t1-tiene-la-misma-informaciuxf3n-que-b_t1-b_t-r-b_t}{%
\section{\texorpdfstring{\(B_t − B_{t−1} = r B_{t−1}\) , tiene la
misma información que
\(B_{t+1} − B_t = r B_t\)}{2. B\_t − B\_\{t−1\} = r B\_\{t−1\} , tiene la misma información que B\_\{t+1\} − B\_t = r B\_t}}\label{b_t-b_t1-r-b_t1-tiene-la-misma-informaciuxf3n-que-b_t1-b_t-r-b_t}}

\hypertarget{a-dado-r-y-b_0-encontrar-b_4-a-partir-de-las-ecuaciones-b_t-b_t1-r-b_t1-con-t1234.}{%
\subsection{\texorpdfstring{a) Dado \(r\) y \(B_0\) encontrar \(B_4\) a
partir de las ecuaciones \(B_t − B_{t−1} = r B_{t−1}\) con
\(t=1,2,3,4\).}{a) Dado r y B\_0 encontrar B\_4 a partir de las ecuaciones B\_t − B\_\{t−1\} = r B\_\{t−1\} con t=1,2,3,4.}}\label{a-dado-r-y-b_0-encontrar-b_4-a-partir-de-las-ecuaciones-b_t-b_t1-r-b_t1-con-t1234.}}

    Dadas las siguientes ecuaciones

    \[B_1-B_{0}= r B_{0},\] \[B_2-B_{1}= r B_{1},\] \[B_3-B_{2}= r B_{2},\]
\[B_4-B_{3}= r B_{3}.\]

    Se tiene por sumar respectivamente y agrupar \(B_t,t=0,1,2,3\) que
\[B_1= (1+r) B_{0},\] \[B_2= (1+r) B_{1},\] \[B_3= (1+r) B_{2},\]
\[B_4= (1+r) B_{3}.\]

De aquí meramente se sustituye en la derecha de \(B_4\) con las
igualdades. \[B_4= (1+r) (1+r) B_{2},\]
\[B_4= (1+r) (1+r) (1+r) B_{1} ,\]
\[B_4= (1+r) (1+r) (1+r) (1+r) B_{0} ,\] \[B_4= (1+r)^4 B_0.\]

    \hypertarget{prueba-por-inducciuxf3n}{%
\subsubsection{Prueba por Inducción}\label{prueba-por-inducciuxf3n}}

En general si una secuencia de reales \(\{B_t\}_{t\in\mathbb N}\) cumple
que para un \(r\) fijo, \[B_t-B_{t-1}= r B_{t-1}, t\in\mathbb N\]

Entonces también cumple que \[B_{t+1} -B_t= rB_t, t\in\mathbb Z^+.\]

Si además \(r\neq 0\). Para \(t=0\) es cierto que \[B_t = (1+r)^t B_0.\]
Si se cumple lo anterior para \(t\) no necesariamente \(0\), tenemos
\[B_{t+1} -B_t= r B_t,\] \[B_{t+1} = (1+r)B_t.\] Y como para \(t\) hemos
supuesto que \(B_t = (1+r)^t B_0,\)
\[B_{t+1} = (1+r)\left[(1+r)^t B_0\right],\]
\[B_{t+1} = (1+r)^{t+1} B_0.\] Por lo que también se cumple
\(B_n = (1+r)^n B_0\) para \(n=t+1\) dado que se cumple para \(n=t\).
Como es cierto para \(t=0\), por inducción es cierto para
\(t\in\mathbb Z^+\), con el supuesto que
\[B_t-B_{t-1}= r B_{t-1}, t\in\mathbb N.\]

    \hypertarget{b-escribir-ecuaciones-para-b_40-en-los-siguientes-casos.}{%
\subsection{\texorpdfstring{b) Escribir ecuaciones para \(B_{40}\) en
los siguientes
casos.}{b) Escribir ecuaciones para B\_\{40\} en los siguientes casos.}}\label{b-escribir-ecuaciones-para-b_40-en-los-siguientes-casos.}}

\hypertarget{b_050-dado-que-b_t-b_t-1-0.2-b_t-1.}{%
\subsubsection{\texorpdfstring{1) \(B_0=50\) dado que
\(B_t-B_{t-1}= 0.2 B_{t-1}.\)}{1) B\_0=50 dado que B\_t-B\_\{t-1\}= 0.2 B\_\{t-1\}.}}\label{b_050-dado-que-b_t-b_t-1-0.2-b_t-1.}}

Por lo que acabamos de probar, tomando \(r = 0.2\) tenemos que
\[B_{40} = (1.2)^{40} \times 50.\]

    \hypertarget{b_0-50-dado-que-b_t-b_t-1--0.1-b_t-1.}{%
\subsubsection{\texorpdfstring{2) \(B_0 = 50\) dado que
\(B_t-B_{t-1} = -0.1 B_{t-1}\).}{2) B\_0 = 50 dado que B\_t-B\_\{t-1\} = -0.1 B\_\{t-1\}.}}\label{b_0-50-dado-que-b_t-b_t-1--0.1-b_t-1.}}

En lo que probamos anteriormente tomamos \(r= -0.1\):
\[B_{40} = (0.9)^{40} \times 50.\]

    \hypertarget{reacciuxf3n-en-cadena-de-la-polimerasa-para-copiar-segmentos-de-adn}{%
\section{Reacción en Cadena de la Polimerasa para Copiar Segmentos de
ADN}\label{reacciuxf3n-en-cadena-de-la-polimerasa-para-copiar-segmentos-de-adn}}

    \hypertarget{descripciuxf3n-y-notaciuxf3n}{%
\subsection{Descripción y Notación}\label{descripciuxf3n-y-notaciuxf3n}}

Llamamos a un ciclo a la duplicación de una cantidad de ADN inicial. En
notación matemática esto es que al terminar el ciclo \(n=0,1,\ldots\)
hay una cantidad \(B_{n+1}=2B_n\) de ADN, dado que el ciclo inició con
la cantidad \(B_n\) de ADN.

    Con estos dos datos podemos deducir que habiendo pasado \(n\geq 0\)
ciclos el proceso obedece que
\[B_n = 2 B_{n-1} = 2^2 B_{n-2} =\ldots = 2^n B_0.\]

(Ignorando las igualdades intermedias si \(B_{n-1}\) o \(B_{n-2}\) no
está definido.)

    Con una cantidad inicial \(B_0 = 10^{-12}\text{g}\) de ADN en el inicio
del ciclo 0, tras 30 ciclos tenemos
\(B_{30} = 2^{30}\times 10^{-12}\text{g}\) de ADN (al inicio del ciclo
30).

    \begin{tcolorbox}[breakable, size=fbox, boxrule=1pt, pad at break*=1mm,colback=cellbackground, colframe=cellborder]
\prompt{In}{incolor}{10}{\boxspacing}
\begin{Verbatim}[commandchars=\\\{\}]
\PY{n+nb}{round}\PY{p}{(}\PY{l+m+mi}{2}\PY{o}{*}\PY{o}{*}\PY{l+m+mi}{30}\PY{o}{*}\PY{l+m+mi}{10}\PY{o}{*}\PY{o}{*}\PY{p}{(}\PY{o}{\PYZhy{}}\PY{l+m+mi}{12}\PY{p}{)}\PY{p}{,}\PY{l+m+mi}{12}\PY{p}{)}
\end{Verbatim}
\end{tcolorbox}

            \begin{tcolorbox}[breakable, size=fbox, boxrule=.5pt, pad at break*=1mm, opacityfill=0]
\prompt{Out}{outcolor}{10}{\boxspacing}
\begin{Verbatim}[commandchars=\\\{\}]
0.001073741824
\end{Verbatim}
\end{tcolorbox}
        
    Esto es \(0.001073741824\) gramos.

    \hypertarget{crecimiento-exponencial-de-la-poblaciuxf3n-humana}{%
\section{Crecimiento Exponencial de la Población
Humana}\label{crecimiento-exponencial-de-la-poblaciuxf3n-humana}}

    \hypertarget{descripciuxf3n-y-notaciuxf3n}{%
\subsection{Descripción y Notación}\label{descripciuxf3n-y-notaciuxf3n}}

\(t\) es el número de décadas después de 1940. \(P_t\) es la población
en millones en el año \(1940+10t\). Sea \(\Delta P_t = P_{t+1}-P_t\).

    \begin{tcolorbox}[breakable, size=fbox, boxrule=1pt, pad at break*=1mm,colback=cellbackground, colframe=cellborder]
\prompt{In}{incolor}{11}{\boxspacing}
\begin{Verbatim}[commandchars=\\\{\}]
\PY{k+kn}{from} \PY{n+nn}{numpy} \PY{k+kn}{import} \PY{n}{array}
\PY{c+c1}{\PYZsh{}Tabla41 = pd.read\PYZus{}clipboard()}
\PY{c+c1}{\PYZsh{}Tabla41.to\PYZus{}dict()}
\PY{n}{data} \PY{o}{=} \PY{p}{\PYZob{}}\PY{l+s+s1}{\PYZsq{}}\PY{l+s+s1}{\PYZdl{}t\PYZdl{}}\PY{l+s+s1}{\PYZsq{}}\PY{p}{:} \PY{p}{\PYZob{}}\PY{l+m+mi}{1940}\PY{p}{:} \PY{l+m+mi}{0}\PY{p}{,} \PY{l+m+mi}{1950}\PY{p}{:} \PY{l+m+mi}{1}\PY{p}{,} \PY{l+m+mi}{1960}\PY{p}{:} \PY{l+m+mi}{2}\PY{p}{,} \PY{l+m+mi}{1970}\PY{p}{:} \PY{l+m+mi}{3}\PY{p}{,} \PY{l+m+mi}{1980}\PY{p}{:} \PY{l+m+mi}{4}\PY{p}{,} \PY{l+m+mi}{1990}\PY{p}{:} \PY{l+m+mi}{5}\PY{p}{,} \PY{l+m+mi}{2000}\PY{p}{:} \PY{l+m+mi}{6}\PY{p}{,} \PY{l+m+mi}{2010}\PY{p}{:} \PY{l+m+mi}{7}\PY{p}{\PYZcb{}}\PY{p}{,}
 \PY{l+s+s1}{\PYZsq{}}\PY{l+s+s1}{Poblacion \PYZdl{}10\PYZca{}6\PYZdl{}}\PY{l+s+s1}{\PYZsq{}}\PY{p}{:} \PY{p}{\PYZob{}}\PY{l+m+mi}{1940}\PY{p}{:} \PY{l+m+mf}{2.3}\PY{n}{r}\PY{p}{,}
  \PY{l+m+mi}{1950}\PY{p}{:} \PY{l+m+mf}{2.52}\PY{n}{r}\PY{p}{,}
  \PY{l+m+mi}{1960}\PY{p}{:} \PY{l+m+mf}{3.02}\PY{n}{r}\PY{p}{,}
  \PY{l+m+mi}{1970}\PY{p}{:} \PY{l+m+mf}{3.7}\PY{n}{r}\PY{p}{,}
  \PY{l+m+mi}{1980}\PY{p}{:} \PY{l+m+mf}{4.45}\PY{n}{r}\PY{p}{,}
  \PY{l+m+mi}{1990}\PY{p}{:} \PY{l+m+mf}{5.3}\PY{n}{r}\PY{p}{,}
  \PY{l+m+mi}{2000}\PY{p}{:} \PY{l+m+mf}{6.06}\PY{n}{r}\PY{p}{,}
  \PY{l+m+mi}{2010}\PY{p}{:} \PY{l+m+mf}{6.8}\PY{n}{r}\PY{p}{\PYZcb{}}\PY{p}{\PYZcb{}}
\PY{n}{Tabla41} \PY{o}{=} \PY{n}{pd}\PY{o}{.}\PY{n}{DataFrame}\PY{p}{(}\PY{n}{data}\PY{p}{)}
\end{Verbatim}
\end{tcolorbox}

    \begin{longtable}[]{@{}rrr@{}}
\toprule
& \(t\) & Poblacion \(10^6\) \\
\midrule
\endhead
1940 & 0 & 2.3 \\
1950 & 1 & 2.52 \\
1960 & 2 & 3.02 \\
1970 & 3 & 3.7 \\
1980 & 4 & 4.45 \\
1990 & 5 & 5.3 \\
2000 & 6 & 6.06 \\
2010 & 7 & 6.8 \\
\bottomrule
\end{longtable}

    \hypertarget{a-probar-la-ecuaciuxf3n-contra-los-datos}{%
\subsection{a) Probar la ecuación contra los
datos}\label{a-probar-la-ecuaciuxf3n-contra-los-datos}}

    \begin{tcolorbox}[breakable, size=fbox, boxrule=1pt, pad at break*=1mm,colback=cellbackground, colframe=cellborder]
\prompt{In}{incolor}{12}{\boxspacing}
\begin{Verbatim}[commandchars=\\\{\}]
\PY{n}{var}\PY{p}{(}\PY{l+s+s1}{\PYZsq{}}\PY{l+s+s1}{t}\PY{l+s+s1}{\PYZsq{}}\PY{p}{)}
\PY{n}{show}\PY{p}{(}\PY{n}{points}\PY{p}{(}\PY{n}{Tabla41}\PY{o}{.}\PY{n}{to\PYZus{}numpy}\PY{p}{(}\PY{p}{)}\PY{p}{,} \PY{n}{rgbcolor}\PY{o}{=}\PY{p}{(}\PY{l+m+mi}{0}\PY{p}{,}\PY{l+m+mi}{0}\PY{p}{,}\PY{l+m+mi}{0}\PY{p}{)}\PY{p}{,} \PY{n}{legend\PYZus{}label}\PY{o}{=}\PY{l+s+s2}{\PYZdq{}}\PY{l+s+s2}{Población \PYZdl{}}\PY{l+s+se}{\PYZbs{}\PYZbs{}}\PY{l+s+s2}{times 10\PYZca{}6\PYZdl{}}\PY{l+s+s2}{\PYZdq{}}\PY{p}{)}
     \PY{o}{+}\PY{n}{plot}\PY{p}{(}\PY{l+m+mf}{2.2}\PY{o}{*}\PY{l+m+mf}{1.19}\PY{o}{*}\PY{o}{*}\PY{n}{t}\PY{p}{,} \PY{p}{(}\PY{n}{x}\PY{p}{,}\PY{l+m+mi}{0}\PY{p}{,}\PY{l+m+mi}{7}\PY{p}{)}\PY{p}{,} \PY{n}{legend\PYZus{}label}\PY{o}{=}\PY{l+s+s2}{\PYZdq{}}\PY{l+s+s2}{\PYZdl{}P\PYZus{}t = 2.2}\PY{l+s+se}{\PYZbs{}\PYZbs{}}\PY{l+s+s2}{cdot 1.19\PYZca{}t\PYZdl{}}\PY{l+s+s2}{\PYZdq{}}\PY{p}{)}\PY{p}{,}
     \PY{n}{dpi}\PY{o}{=}\PY{l+m+mi}{120}\PY{p}{,} \PY{n}{axes\PYZus{}labels}\PY{o}{=}\PY{p}{[}\PY{l+s+s2}{\PYZdq{}}\PY{l+s+s2}{\PYZdl{}t\PYZdl{}}\PY{l+s+s2}{\PYZdq{}}\PY{p}{,}\PY{k+kc}{None}\PY{p}{]}\PY{p}{)}
\end{Verbatim}
\end{tcolorbox}

    \begin{center}
    \adjustimage{max size={0.9\linewidth}{0.9\paperheight}}{output_50_0.png}
    \end{center}
    { \hspace*{\fill} \\}
    
    \begin{tcolorbox}[breakable, size=fbox, boxrule=1pt, pad at break*=1mm,colback=cellbackground, colframe=cellborder]
\prompt{In}{incolor}{13}{\boxspacing}
\begin{Verbatim}[commandchars=\\\{\}]
\PY{n}{Comparacion} \PY{o}{=} \PY{n+nb}{list}\PY{p}{(}\PY{n+nb}{map}\PY{p}{(}\PY{k}{lambda} \PY{n}{P}\PY{p}{:} \PY{p}{(}\PY{n}{P}\PY{p}{[}\PY{l+m+mi}{0}\PY{p}{]}\PY{p}{,} \PY{n}{P}\PY{p}{[}\PY{l+m+mi}{1}\PY{p}{]}\PY{p}{,} \PY{n+nb}{float}\PY{p}{(}\PY{l+m+mf}{2.2}\PY{o}{*}\PY{l+m+mf}{1.19}\PY{o}{*}\PY{o}{*}\PY{n}{P}\PY{p}{[}\PY{l+m+mi}{0}\PY{p}{]}\PY{p}{)}\PY{p}{)}\PY{p}{,} \PY{n}{Tabla41}\PY{o}{.}\PY{n}{to\PYZus{}numpy}\PY{p}{(}\PY{p}{)}\PY{p}{)}\PY{p}{)}
\PY{n}{Comparacion} \PY{o}{=} \PY{n}{pd}\PY{o}{.}\PY{n}{DataFrame}\PY{p}{(}\PY{n}{Comparacion}\PY{p}{,}
                           \PY{n}{columns} \PY{o}{=} \PY{p}{[}\PY{l+s+s2}{\PYZdq{}}\PY{l+s+s2}{t}\PY{l+s+s2}{\PYZdq{}}\PY{p}{,}\PY{l+s+s2}{\PYZdq{}}\PY{l+s+s2}{Poblacion \PYZdl{}}\PY{l+s+se}{\PYZbs{}\PYZbs{}}\PY{l+s+s2}{times 10\PYZca{}6\PYZdl{}}\PY{l+s+s2}{\PYZdq{}}\PY{p}{,} \PY{l+s+s2}{\PYZdq{}}\PY{l+s+s2}{\PYZdl{}2.2}\PY{l+s+se}{\PYZbs{}\PYZbs{}}\PY{l+s+s2}{cdot 1.19\PYZca{}t\PYZdl{}}\PY{l+s+s2}{\PYZdq{}}\PY{p}{]}\PY{p}{,}
                          \PY{n}{index}\PY{o}{=}\PY{p}{[}\PY{l+m+mi}{1940}\PY{o}{+}\PY{n}{j}\PY{o}{*}\PY{l+m+mi}{10} \PY{k}{for} \PY{n}{j} \PY{o+ow}{in} \PY{n+nb}{range}\PY{p}{(}\PY{n+nb}{len}\PY{p}{(}\PY{n}{Comparacion}\PY{p}{)}\PY{p}{)}\PY{p}{]}\PY{p}{)}
\end{Verbatim}
\end{tcolorbox}

    \begin{longtable}[]{@{}rrrr@{}}
\toprule
& t & Poblacion \(\times 10^6\) & \(2.2\cdot 1.19^t\) \\
\midrule
\endhead
1940 & 0 & 2.3 & 2.2 \\
1950 & 1 & 2.52 & 2.618 \\
1960 & 2 & 3.02 & 3.11542 \\
1970 & 3 & 3.7 & 3.70735 \\
1980 & 4 & 4.45 & 4.41175 \\
1990 & 5 & 5.3 & 5.24998 \\
2000 & 6 & 6.06 & 6.24747 \\
2010 & 7 & 6.8 & 7.43449 \\
\bottomrule
\end{longtable}

    Se puede observar que los datos están cerca de lo que predice la
ecuación en un margen de \(10^5\) sin sobrestimar o subestimar todos los
puntos. Es decir la distancia a los datos es relativamente buena,
también tomando en cuenta la complejidad de lo que se intenta modelar.

    \hypertarget{b-quuxe9-porcentaje-de-aumento-en-la-poblaciuxf3n-en-cada-duxe9cada-supone-el-modelo-para-esta-ecuaciuxf3n}{%
\subsection{b) ¿Qué porcentaje de aumento en la población en cada década
supone el modelo para esta
ecuación?}\label{b-quuxe9-porcentaje-de-aumento-en-la-poblaciuxf3n-en-cada-duxe9cada-supone-el-modelo-para-esta-ecuaciuxf3n}}

Como el modelo es \(P_t = 2.2\cdot 1.19^t\), \(1.19= 1+r\) donde
\(r=0.19\) es la tasa de crecimiento relativo a la población; es decir
viene de la ecuación \(\Delta P_t = r P_t\); entonces \(19\%\) es el
porcentaje de aumento de la población porque \(t\) está dado por
decadas.

    \hypertarget{c-quuxe9-poblaciuxf3n-predice-la-ecuaciuxf3n-para-el-auxf1o-2050}{%
\subsection{c) ¿Qué población predice la ecuación para el año
2050?}\label{c-quuxe9-poblaciuxf3n-predice-la-ecuaciuxf3n-para-el-auxf1o-2050}}

El año 2050 está dado por \(1940+10t\), entonces
\(t= (2050-1940)/10 = 11\) es la decada correspondiente a 2050. Así la
población predecida para el año 2050 es
\(P_{11} = 2.2 \cdot 1.19^{11}= 14.908682\) millones.

    \begin{tcolorbox}[breakable, size=fbox, boxrule=1pt, pad at break*=1mm,colback=cellbackground, colframe=cellborder]
\prompt{In}{incolor}{14}{\boxspacing}
\begin{Verbatim}[commandchars=\\\{\}]
\PY{n+nb}{print}\PY{p}{(}\PY{p}{(}\PY{l+m+mi}{2050}\PY{o}{\PYZhy{}}\PY{l+m+mi}{1940}\PY{p}{)}\PY{o}{/}\PY{l+m+mi}{10}\PY{p}{)}
\PY{n+nb}{round}\PY{p}{(}\PY{l+m+mf}{2.2}\PY{o}{*}\PY{l+m+mf}{1.19}\PY{o}{*}\PY{o}{*}\PY{l+m+mi}{11}\PY{p}{,}\PY{l+m+mi}{6}\PY{p}{)}
\end{Verbatim}
\end{tcolorbox}

    \begin{Verbatim}[commandchars=\\\{\}]
11
    \end{Verbatim}

            \begin{tcolorbox}[breakable, size=fbox, boxrule=.5pt, pad at break*=1mm, opacityfill=0]
\prompt{Out}{outcolor}{14}{\boxspacing}
\begin{Verbatim}[commandchars=\\\{\}]
14.908682
\end{Verbatim}
\end{tcolorbox}
        

    % Add a bibliography block to the postdoc
    
    
    
\end{document}
